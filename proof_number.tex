\subsubsection{\rProp{tonelli_shanks_algorithm}の証明}
\begin{prProof}{tonelli_shanks_algorithm}
\rProp{sqrt1}より、$(AD^m)^{2^{s-k}}\equiv1\pmod{p}$ならば、$(AD^m)^{2^{s-(k+1)}}\equiv\pm1\pmod{p}$となるのは明らか。
$(AD^m)^{2^{s-(k+1)}}\equiv-1\pmod{p}$のとき、
\begin{align*}
(AD^{m + 2^k})^{2^{s-(k+1)}} &\equiv (AD^m)^{2^{s-(k+1)}}\cdot D^{2^{k+(s-(k+1))}}\pmod{p}\\
&\equiv (-1)\cdot D^{2^{s-1}}\pmod{p}
\end{align*}
となるが、$D$の位数は$2^s$なので、$D^{2^{s-1}}\equiv-1\pmod{p}$となる。
よって、$(AD^{m'})^{2^{s-(k+1)}}\equiv1\pmod{p}$である。
\end{prProof}

\subsubsection{\rTheo{pepin}の証明}
\begin{thProof}{pepin}
必要条件は、Lucasの定理(\rTheo{Lucas_theorem})より明らか。

十分条件を証明する。
\rProp{Legendre_symbol_calc}より
\begin{align*}
\left(\frac{3}{F_k}\right) \equiv 3^{(F_k-1)/2} \pmod{F_k}
\end{align*}
であるから$\left(\frac{3}{F_k}\right)$が$-1$となればよい。
$F_k$が素数のとき、平方剰余の相互法則(\rTheo{legendre_quadratic_reciprocity})より
\begin{align*}
\left(\frac{3}{F_k}\right)\left(\frac{F_k}{3}\right) = (-1)^{\frac{3-1}{2}\frac{F_k-1}{2}}
\end{align*}
が成り立つので、
\begin{align*}
\left(\frac{3}{F_k}\right) = (-1)^{\frac{3-1}{2}\frac{F_k-1}{2}} \bigg/ \left(\frac{F_k}{3}\right)
\end{align*}
が$-1$となれば証明が完了する。

$(F_k-1)/2$は偶数だから$(-1)^{\frac{3-1}{2}\frac{F_k-1}{2}}=1$。

$2^k$が偶数であること、2の偶数乗は3を法として1と合同であることより、$F_k\equiv2\pmod{3}$を得る。
よって、
\begin{align*}
\left(\frac{F_k}{3}\right) = \left(\frac{2}{3}\right) = 2^{(3-1)/2} \equiv -1 \pmod{3}
\end{align*}
となる。

以上より、
\begin{align*}
\left(\frac{3}{F_k}\right) = -1
\end{align*}
となり、証明は完了する。
\end{thProof}
