\documentclass[dvipdfmx]{jarticle}
\usepackage[dvipdfm,bookmarks=true,bookmarksnumbered=true,bookmarkstype=toc]{hyperref}
\ifnum 42146=\euc"A4A2 \AtBeginDvi{\special{pdf:tounicode EUC-UCS2}}\else
\AtBeginDvi{\special{pdf:tounicode 90ms-RKSJ-UCS2}}\fi

%%% 余白の定義 %%%
\paperwidth 597pt
\paperheight 845pt
\hoffset -14.0pt
\voffset 14.5pt
 \oddsidemargin 0.0pt
 \evensidemargin 0.0pt
 \topmargin 0.0pt
 \headheight 0.0pt
 \headsep 0.0pt
\textheight 671.0pt
\textwidth 480.5pt
 \marginparsep 0.0pt
 \marginparwidth 0.0pt
 \footskip 15.0pt

\usepackage{amsmath}
\usepackage{amssymb}
\usepackage{amsthm}
\usepackage{ascmac}
\usepackage{boites}
\usepackage{color}
\usepackage{graphicx}
\usepackage{listings}
\usepackage{misc/jlisting}
\usepackage{makeidx}
\usepackage{multicol}
\usepackage{pxrubrica}
\usepackage{tikz}
\usepackage{url}
\usepackage{wrapfig}

\usetikzlibrary{positioning}

%%% 式番号に節番号を付与 %%%
\makeatletter
 \renewcommand{\theequation}{\thesection.\arabic{equation}}
 \@addtoreset{equation}{section}
\makeatother

%%% 図番号に節番号を付与 %%%
\makeatletter
 \renewcommand{\thefigure}{\thesection.\arabic{figure}}
 \@addtoreset{figure}{section}
\makeatother

%%% 表番号に節番号を付与 %%%
\makeatletter
 \renewcommand{\thetable}{\thesection.\arabic{table}}
 \@addtoreset{table}{section}
\makeatother

%%% 注釈 %%%
% \Notes{内容}
\newcounter{notesNum}
\newcommand{\Notes}[1]{\footnote [\arabic{notesNum}]{#1}\stepcounter{notesNum}}

%%% 索引 %%%
% \IND{項目}{こうもく}
% \Ind{Item}
\newcommand{\IND}[2]{\textbf{#1\index{#2@#1}}}
\newcommand{\Ind}[1]{\IND{#1}{#1}}

%%% theorem %%%
\theoremstyle{definition}
\newtheorem{prop}{命題}[section]
\newtheorem{theo}[prop]{定理}
\newtheorem{defi}[prop]{定義}
\newtheorem{lemm}[prop]{補題}
\newtheorem{coro}[prop]{系}
\newtheorem{conj}[prop]{予想}
\newtheorem{exam}[prop]{例}
\newtheorem{algo}[prop]{アルゴリズム}
\renewcommand{\qedsymbol}{(証明終)}
\renewcommand{\proofname}{\bfseries 証明}

% \begin{Prop}{Title}{label}
\newenvironment{Prop}[2]{\begin{prop}[#1]\label{prop:#2}}{\end{prop}}
\newenvironment{Theo}[2]{\begin{theo}[#1]\label{theo:#2}}{\end{theo}}
\newenvironment{Defi}[2]{\begin{defi}[#1]\label{defi:#2}}{\end{defi}}
\newenvironment{Lemm}[2]{\begin{lemm}[#1]\label{lemm:#2}}{\end{lemm}}
\newenvironment{Coro}[2]{\begin{coro}[#1]\label{coro:#2}}{\end{coro}}
\newenvironment{Conj}[2]{\begin{conj}[#1]\label{conj:#2}}{\end{conj}}
\newenvironment{Exam}[2]{\begin{exam}[#1]\label{exam:#2}}{\end{exam}}

%%% アルゴリズム %%%
% \Algo{Title}{label}{fileName}
\newcommand{\Algo}[3]{\begin{algo}[#1]\label{algo:#2}\footnotesize{#3}\end{algo}\vspace{-3.3mm}\lstinputlisting{src/#2.py}}

% 参照
\newcommand{\rProp}[1]{命題\ref{prop:#1}}
\newcommand{\rTheo}[1]{定理\ref{theo:#1}}
\newcommand{\rDefi}[1]{定義\ref{defi:#1}}
\newcommand{\rLemm}[1]{補題\ref{lemm:#1}}
\newcommand{\rCoro}[1]{系\ref{coro:#1}}
\newcommand{\rExam}[1]{例\ref{exam:#1}}
\newcommand{\rAlgo}[1]{アルゴリズム\ref{algo:#1}}

% proof環境
\newenvironment{prProof}[1]{\begin{proof}[\rProp{#1}の証明]}{\end{proof}}
\newenvironment{thProof}[1]{\begin{proof}[\rTheo{#1}の証明]}{\end{proof}}
\newenvironment{lmProof}[1]{\begin{proof}[\rLemm{#1}の証明]}{\end{proof}}
\newenvironment{crProof}[1]{\begin{proof}[\rCoro{#1}の証明]}{\end{proof}}

%プログラム挿入の設定
\lstset{
    language = Python,
    breaklines = true,
    breakindent = 10pt,
    basicstyle = \ttfamily\footnotesize,
    commentstyle = {\color[cmyk]{1,0.4,1,0}},
    classoffset = 0,
    keywordstyle = {\bfseries \color[cmyk]{0,1,0,0}},
    stringstyle = {\ttfamily \color[rgb]{0,0,1}},
    frame = single,
    framesep = 5pt,
    numberstyle = \tiny,
    tabsize = 4,
    showstringspaces=false,
}
