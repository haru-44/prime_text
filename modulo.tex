$153$は$3$の倍数だろうか? では、$2481924$は? ちょっと\kenten{小技}を知っていれば、割り算をすることなくどちらも$3$の倍数であると分かる。
この小技というのが、「各桁の数を足した数が$3$の倍数ならば元の数も$3$の倍数である」という事実である。
$1+5+3=9$は明らかに$3$の倍数であるし、$2+4+8+1+9+2+4=30$もやはり$3$の倍数である。
この事実は、ちょっとした計算チェックをするときなどに役立つので知っている方も多いと思うが、一方でなぜそうなるかを知っている人は意外と少ない。
実は、割り算をした時の「余り」が関係してくるので、余りについて(もう少し堅苦しい言葉ですれば剰余について)語っていきたいと思う。

小学校では、
\begin{align*}
13 \div 3 = 4 \mbox{余り} 1
\end{align*}
というように書いたと思うが、今注目しているのは余りであるため、新しく余りを求める演算子``$\%$"を導入して、
\begin{align*}
13 \% 3 = 1
\end{align*}
と書くようにしよう。
また、後ほど定義するが、
\begin{align*}
13 \equiv 1 \pmod{3}
\end{align*}
と書いても、ほとんど同じ意味である。
前者は、プログラミング言語(主にCやPythonなど)での書き方であるのに対して、後者は数学書でよく見られる書き方だ。

剰余の性質の一つに、先に剰余してから足し算しても、足し算した後に剰余をしても結果は同じになる、というものがある。
つまり、
\begin{align*}
((a \% c) + (b \% c)) \% c &= (a + b) \% c\\
((a \% c) - (b \% c)) \% c &= (a - b) \% c\\
((a \% c) \times (b \% c)) \% c &= (a \times b) \% c\\
\end{align*}
という関係が成り立つ。

例えば、$14$と$23$をそれぞれ$5$で剰余算しても、足してから剰余算しても結果は一致する。
\begin{align*}
((14 \% 5) + (23 \% 5)) \% 5 &= (4 + 3) \% 5\\
&= 2\\
(14 + 23) \% 5 &= 37 \% 5\\
&= 2
\end{align*}

注意しなければならないのは、足し算、引き算、掛け算では成り立つが、割り算では必ずしも成り立たないということだ。
最初に紹介した「各桁の数を足した数が$3$の倍数ならば元の数も$3$の倍数である」といった算数マジックの裏には、この原理が潜んでいる。
我々が普段使っている数値は、10進法であって、つまり、$1000a + 100b + 10c + d$というふうに分解できるわけである。
ここで、$a,b,c,d$が各桁の数字なのだが、$10, 100, 1000$いずれも$3$で剰余算すると$1$になる\Notes{1つ1つ確認してもよいが、$10 \% 3 = 1$であることが分かれば、以降は、$100 \% 3 = (10 \times 10) \% 3 = 1 \times 1$などとなる}。
つまり、$3$で剰余算する世界では、$1000a + 100b + 10c + d$と$a + b + c + d$は同じ数にたどり着くのだ。

改めて$\bmod$を定義しよう。
$a \equiv b \pmod{n}$とは、$a-b \mid n$と定義する。
つまり、$a-b$は$n$を割り切るとき、$a \equiv b \pmod{n}$と書くことにし、$a$と$b$は$n$を法として\IND{合同}{こうとう}(congruence)と言うことにする。
だが既に説明したように、$a \% n = b$の別表現だと思った方が分かりやすい。

このような定義で上手くいくのか、ということはもう少し精緻に議論しなければならないのだが、先へ進もう。

\begin{Exam}{}{mod}
\begin{align*}
12 &\equiv 2 \pmod{10}\\
-4 &\equiv 5 \pmod{9}\\
\end{align*}
\end{Exam}

書き方も、最初は面食らうと思うが、その内慣れる。
$=$ではなく$\equiv$を使っているのは、「$12$と$2$は本来的に異なるものだけれど、$10$で剰余算した世界では同じと見なせる」というキモチを表している。

さて、$\mathbb{Z}_n$は、$0$から$n-1$までの整数の集合としよう。
既に見たように、足し算、引き算、掛け算が(剰余算を取ることにすれば)この集合の中で実現できる\Notes{$\mathbb{Z}_n$は$\mathbb{Z}/n\mathbb{Z}$と書かれることもあるが、育まれた環境が異なるだけで本質的には同じである。}。

\begin{Exam}{}{congruence}
$\mathbb{Z}_5$は$\{0,1,2,3,4\}$の5つの整数の集合であり、足し算及び掛け算は次のように行われる。
つまり、普通の足し算及び掛け算の後に、$n=5$で剰余算するのである。
\begin{table}[htbp] \label{table:congruence}\caption{演算表}
 \begin{minipage}[t]{.45\textwidth}
  \begin{center}
   (a) 足し算\\
   \begin{tabular}{|c||c|c|c|c|c|} \hline
    $+$ & 0 & 1 & 2 & 3 & 4 \\ \hline\hline
    0   & 0 & 1 & 2 & 3 & 4 \\ \hline
    1   & 1 & 2 & 3 & 4 & 0 \\ \hline
    2   & 2 & 3 & 4 & 0 & 1 \\ \hline
    3   & 3 & 4 & 0 & 1 & 2 \\ \hline
    4   & 4 & 0 & 1 & 2 & 3 \\ \hline
   \end{tabular}
  \end{center}
 \end{minipage}
 %
 \hfill
 %
 \begin{minipage}[t]{.45\textwidth}
  \begin{center}
   (b) 掛け算\\
   \begin{tabular}{|c||c|c|c|c|c|} \hline
    $\times$ & 0 & 1 & 2 & 3 & 4 \\ \hline\hline
    0        & 0 & 0 & 0 & 0 & 0 \\ \hline
    1        & 0 & 1 & 2 & 3 & 4 \\ \hline
    2        & 0 & 2 & 4 & 1 & 3 \\ \hline
    3        & 0 & 3 & 1 & 4 & 2 \\ \hline
    4        & 0 & 4 & 3 & 2 & 1 \\ \hline
   \end{tabular}
  \end{center}
 \end{minipage}
\end{table}
\end{Exam}

このようにして、加・減・乗ができる$\mathbb{Z}_n$という何やらよく分からないものを手に入れた。
この$\mathbb{Z}_n$は、何が凄いかというと、$n$が素数の場合とそうでない場合とで構造が異なるのである。
ということは、$n$が素数かどうかを見極めるには、$\mathbb{Z}_n$の構造を見極めればよいことになる。

それではどんなときに、素数と合成数とで世界が違ってくるのか? それは、「べき乗」である。
掛け算を足し算を複数回繰り返し適用することと定義したように、「べき乗」は掛け算を複数回繰り返し適用することと定義できる。
つまり、
\begin{align*}
g^x = \underbrace{g \times g \times \cdots \times g}_{x \mbox{ 回}}
\end{align*}
というわけだ。

それだけで$\mathbb{Z}_n$の構造の一端が垣間見れる。
その一端とは、Fermatの小定理だ。
\kenten{小}定理の名は、かの有名なFermatの最終定理と比べられたためであって、この定理の有用さは、その名とは大きく異なる。

\begin{Theo}{\IND{Fermatの小定理}{Fermatのしようていり}, Fermat's little theorem}{Fermats-little-theorem}
素数$n$と互いに素な任意の整数$a$について、次の式が成り立つ。
\begin{align*}
a^{n-1} \equiv 1 \pmod{n}
\end{align*}
\end{Theo}

$a^n\equiv a\pmod{n}$の形で紹介されることもあるが、両辺に$a$を掛ければこの形になる。
いずれにせよ、Fermatの小定理は、「$a^{n-1}$を$n$で割った余りは、必ず1になる」ということを主張している。
確かに$n$が素数ならば、$a$がどんな値であろうとも計算結果は1になる。
一方、合成数の場合は、1になることもあれば、1にならないこともある。

\begin{Exam}{}{fermats-little-theorem1}
\begin{align*}
3^4 &\equiv 1 \pmod{5}\\
3^3 &\equiv 3 \not\equiv 1 \pmod{4}
\end{align*}
\end{Exam}

つまり、1になった場合は素数かどうかは分からないが、1にならない場合は合成数と確定するわけだ。
これは、論理学で言うところの対偶である。

\begin{Theo}{Fermatの小定理の対偶}{}
整数$n$と互いに素な整数$a$について、
\begin{align*}
a^{n-1} \not\equiv 1 \pmod{n}
\end{align*}
ならば、$n$は合成数である。
\end{Theo}

さて、このフェルマーの小定理の対偶を用いて、素数判定を行うことができる。
試し割では具体的な素因数を明らかにすることによって、素数/合成数が判明する。
そのようにして素数判定を行っても良いのだが、素因数を明らかにせずとも素数かどうかを判別することも可能で、Fermatの小定理を利用する\IND{Fermatテスト}{Fermat primality test}もその一つである。

\Algo{Fermatテスト}{fermat_test}

一般的には、試行回数$k$を大きくすることによって、素数である確実性が上がることが期待できる。
例えば、$n=15$のとき、$a^{n-1} \bmod{n}$が1になるのは、$a=4,11,14$のときである。
もちろん、15は合成数なのでこれら3つに遭遇してしまうと、正しく合成数と判定できない。
例え運悪く3つのどれかを選んでしまったとしても、もう一度$a$を選んで「合成数でない」と判定できる他の$a$を選ぶことを祈ればよい。
$n=15$で試してみたが、すべての合成数$n$で同じようなことが言えるのだろうか。
つまり、$n$が合成数にも関わらず、 $a^{n−1} \bmod{n}$ の結果が常に1になってしまうような$n$は存在するのだろうか。
そのような数を、Carmichael数と呼ぶ\Notes{\url{https://oeis.org/A002997}}。

\begin{Defi}{\IND{Carmichael数}{Carmichaelすう}, Carmichael number}{Carmichael number}
合成数$n$が、$n$と互いに素な、どんな$a$にも$a^{n−1}\equiv 1 \pmod{n}$を満たすとき、$n$をCarmichael数と呼ぶ。
\end{Defi}

あるいは同値であるが「どんな整数$a$にも$a^{n}\equiv a \pmod{n}$を満たすとき」と定義する場合もある。
最小のCarmichael数は561で、無数に存在することが知られている\cite{Alford1994}。
さらに必要十分条件も知られている。

\begin{Theo}{\IND{Korseltの判定法}{Korseltのはんていほう}, Korselt's criterion}{Korselt's criterion}
合成数$n$の任意の素因数$p$が次を満たすとき、かつそのときのみ、$n$はCarmichael数である。
\begin{itemize}
 \item $p^2 \nmid n$
 \item $(p − 1) \mid (n-1)$
\end{itemize}
\end{Theo}

\begin{Exam}{}{}
561は、$561=3^1\cdot 11^1 \cdot17^1$と素因数分解できる。
\begin{itemize}
 \item 平方因子を持たない。つまり、 3,11,17 いずれの指数部も1である。
 \item $560 = (3−1) \times 280 = (11−1) \times 56 = (17−1) \times 35$
\end{itemize}
となるから、Korseltの判定法の性質を満たしている。
\end{Exam}

意外にも、Carmichaelが具体例として571を発見する前に、Korseltが必要十分条件を与えている。
前者が1910年、後者が1899年だから11年も前である。
Korseltは、そのような数が存在しないだろうと予想したのだろうが、手計算でしらみ潰しに探しても見つけられる場所に存在していたのである。

なお、Pythonでは、pow関数がべき乗剰余を計算してくれるため気にする必要はないのだが、後々のためにどのようにべき乗を計算しているのかも解説しておこう。
素朴に考えれば、べき乗は$a$を$n$回掛けるのだから$n$回の演算が必要になる。
それでは実用に耐えないので、もう少し効率化したい。
一般に、$a$に対してある演算子$\dagger$を$n$回繰り返す\Notes{例えば、$\dagger=+$なら$na$は掛け算、$\dagger=\times$なら$na$はべき乗に一致する。}、つまり、
\begin{align*}
na = \underbrace{a \dagger a \dagger \cdots \dagger a }_{n\text{ times}}
\end{align*}
という計算は、$\log{n}$回程度の$\dagger$演算で済む。
これは、$(n+m)a = na \dagger ma$が成り立つため\Notes{厳密には$\dagger$が結合法則を満たしているという前提がある。}、$n$を適切に分割すれば(具体的には2のべき乗の足し算に分割すれば)計算回数が削減できるためである。
つまり、$a, 2a, 2^2a, 2^3a, \ldots$は、それぞれ直前の値から$2^{k+1}a = 2^ka \dagger 2^ka$と計算できるし、例えば、$n=13=1+4+8$なら、$a \dagger 2^2a \dagger 2^3a$を計算すれば$13a$が得られる。

\Algo{演算の$n$回適用}{n_times}{}
