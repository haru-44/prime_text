Fibonacci数列を使って素数判定ができた。
Lucas数列はFibonacci数列の一般化であるが、ならば同様にLucas数列でも素数判定ができるのではないか。
そういう思考を働かせて、Lucas数列を導入する。
ちなみにLucas\kenten{数}というのもあって、混乱しやすいこと甚だしい。
ここで使うのは、Lucas\kenten{数列}だからお間違いのないように。

\begin{Defi}{\IND{Lucas数列}{Lucasすうれつ}, Lucas sequence}{Lucas_sequence}
Lucas数列$\{U_n\},\{V_n\}$は、次のように定義される数列である。
\begin{align*}
U_n =
\begin{cases}
0, &\mbox{if } n = 0\\
1, &\mbox{if } n = 1\\
aU_{n - 1} - bU_{n - 2}, &\mbox{if } n \ge 2
\end{cases}
\end{align*}
\begin{align*}
V_n =
\begin{cases}
2, &\mbox{if } n = 0\\
a, &\mbox{if } n = 1\\
aV_{n - 1} - bV_{n - 2}, &\mbox{if } n \ge 2
\end{cases}
\end{align*}
あるいは、同値であるが、
\begin{align*}
U_n &= \frac{\alpha^n - \beta^n}{\alpha - \beta}\\
V_n &= \alpha^n - \beta^n
\end{align*}
と定義される数列である。ここで、$\alpha, \beta$は、2次方程式$x^2-ax+b=0$の解
\begin{align*}
\alpha &= \frac{a+\sqrt{\Delta}}{2}\\
\beta &= \frac{a-\sqrt{\Delta}}{2}
\end{align*}
である。ただし、$\Delta=a^2 - 4b$は平方数でないとする。
\end{Defi}

一応、後々のために$\{V_n\}$も一緒に述べたが、一旦$\{U_n\}$のみで考えることにする。

\Algo{Lucas数列}{lucas_sequence}{}

Lucas数列はFibonacci数列の一般化であると言ったが、$a=1, b=-1$のとき、$U_n$はFibonacci数列に一致する。
\begin{align*}
\alpha &= \frac{1 + \sqrt{1^2 - 4 \times (-1)^2}}{2}\\
&= \frac{1 + \sqrt{5}}{2}\\
&= \phi
\end{align*}

Lucas数列にも、Fibonacci数列が持っていた素数に関する性質がある。
ここで、$\left(\frac{a}{p}\right)$はLegendre記号である。

\begin{Theo}{}{lucas_sequence}
$\Delta=a^2-4b$は平方数でないとする。$n$が$\gcd(n, 2b\Delta)=1$を満たす素数ならば、
\begin{align*}
U_{n - \left(\frac{\Delta}{n}\right)} \equiv 0 \pmod{n}
\end{align*}
が成り立つ。
\end{Theo}

Lucas数列がFibonacci数列の一般化だというのだから、定理もまた一般化されているはずである。
Fibonacci数列の場合を考えてみる。
$\Delta = 1^2-4\times(-1)^2=5$であるから、$\left(\frac{\Delta}{n}\right)$は、$\left(\frac{5}{n}\right)$となる。
平方剰余の相互法則(\rTheo{legendre_quadratic_reciprocity})より、分子と分母(分数ではないので、このような呼称は適切ではないが)が反転可能である。

$n$は素数と仮定しているから、5も$n$も素数であるとしてよい。
$n=2, 5$の場合を棚上げにして、平方剰余の相互法則を適用してみる。
\begin{align*}
\left(\frac{5}{n}\right) \bigg(\frac{n}{5}\bigg) &= (-1)^{\frac{5-1}{2}\cdot\frac{n-1}{2}}\\
&= (-1)^{n-1}\\
&= 1
\end{align*}
$n-1$は常に偶数になるから、$(-1)^{n-1}$は常に$1$になる。
よって、$\left(\frac{5}{n}\right)$と$\left(\frac{n}{5}\right)$は常に同じ値になる(Legendre記号は$0$を除き$-1,1$しかとらないことに注意)。
$n=2, 5$のときも、個別に値を調べれば$\left(\frac{2}{5}\right) = \left(\frac{5}{2}\right) = -1$であるから一致する。
つまり、Lucas数列での定理は、Fibonacci数列の定理の一般化である。

Fibonacci数列のときと同じように、定理から素数判定アルゴリズムを作ることができる。

\Algo{Lucas数列テスト}{lucas_sequence_test}{c.f., \rAlgo{legendre_symbol}, \rAlgo{lucas_sequence}}

やはりこの素数判定法も、合成数を誤って「素数」と判定してしまうことがある。
例えば、$323=17\times19$は、合成数であるが、$a=-10,b=-5$のとき``probable prime"を返す。
