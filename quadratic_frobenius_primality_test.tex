Frobenius写像を使った素数判定法は、Fibonacci数列やLucas数列を使った判定法の一般化と言える。
このFrobenius写像とはどのようなものなのか。
定義から入ろう。

\begin{Defi}{\IND{Frobenius写像}{Frobeniusしやそう}, Frobenius map}{Frobenius_map}
素数$p$を標数とする可換環$R$のFrobenius写像$\sigma:R\to R$を、次のように定義する。
\begin{align*}
\sigma(a) = a^p
\end{align*}
\end{Defi}

難しい言葉が並んだが、結局Frobenius写像とは、元を$p$乗するに過ぎない。
例えば、$\mathbb{F}_p$の場合は、
\begin{align*}
\sigma(a) = a^p = a
\end{align*}
となり、恒等関数となる。
たったそれだけのことである。
しかし、次のような都合の良い性質がある。

\begin{Prop}{}{Frobenius}
素数$p$を標数とする可換環$R$と、その元$a,b\in R$とする。
\begin{itemize}
\item $\sigma(a) + \sigma(b) = \sigma(a + b)$
\item $\sigma(a) \times \sigma(b) = \sigma(a \times b)$
\item $\sigma(a) = a$ならば、$a$は$R$の部分体$\mathbb{F}_p$の元である。
\end{itemize}
\end{Prop}

どんなふうに都合が良いかは、これから説明していく。
今、Lucas数列を$p$で割った余り$U_n \bmod{p}$がどのような構造を持っているかを解明したい。
構造というのは、もっと具体的に言えば$n$の変化によって、$U_n \bmod{p}$がどのように変わるのかということに興味がある。
先に答えを言ってしまうと、$p$を素数としたとき$U_n \bmod{p}$は、$\mathbb{F}_{p^2}$あるいは、$\mathbb{F}_p\times\mathbb{F}_p$と同型になる。
$\left(\frac{\Delta}{n}\right)$が$-1$のとき$\mathbb{F}_{p^2}$と同型になり、$1$のとき$\mathbb{F}_p\times\mathbb{F}_p$と同型となるのだ。
しかし、これは$U_n \bmod{p}$の「内部」で起こっている本質的な事柄であって、人類にとって容易に確認できる代物ではない。
そこで、Frobenius写像が登場する。
$\mathbb{F}_p\times\mathbb{F}_p$にFrobenius写像を適用してもどの元も動かないのに対して、$\mathbb{F}_{p^2}$では根$\alpha,\beta$の置換を引き起こす。
つまり、$\mathbb{F}_{p^2}$と$\mathbb{F}_p\times\mathbb{F}_p$とでは挙動が異なるのだ。

根$\alpha,\beta$だけに注目すれば、$\mathbb{F}_p\times\mathbb{F}_p$のとき(つまりは$\left(\frac{\Delta}{n}\right)=-1$のとき)
\begin{align*}
\begin{cases}
\sigma(\alpha) &= \beta\\
\sigma(\beta) &= \alpha
\end{cases}
\end{align*}
である一方、$\mathbb{F}_{p^2}$のとき(つまりは$\left(\frac{\Delta}{n}\right)=1$のとき)
\begin{align*}
\begin{cases}
\sigma(\alpha) &= \alpha\\
\sigma(\beta) &= \beta
\end{cases}
\end{align*}
となる。
この違い故に、Fibonacci数列、Lucas数列の素数判定にLegendre記号が現れたのである。
思い出して欲しい。
Lucas数列$\{U_n\}$は
\begin{align*}
U_n = \frac{\alpha^n - \beta^n}{\alpha - \beta}
\end{align*}
と表さるのであった。
$p$が素数で$\left(\frac{\Delta}{n}\right)=1$のとき、$U_{p-1} \equiv 0 \pmod{n}$となるのは、$\alpha^p,\beta^p$がそれぞれ$\alpha,\beta$に移るため、
\begin{align*}
U_{p-1} &= \frac{\alpha^{p-1} - \beta^{p-1}}{\alpha - \beta}\\
&= \frac{\alpha^p\beta - \beta^p\alpha}{(\alpha-\beta)\alpha\beta}\\
&= \frac{\alpha\beta - \beta\alpha}{(\alpha-\beta)\alpha\beta}\\
&= 0
\end{align*}
となるためである。
一方、$p$が素数で$\left(\frac{\Delta}{n}\right)=-1$のとき、$U_{p+1} \equiv 0 \pmod{n}$となるのは、$\alpha^p,\beta^p$がそれぞれ$\beta,\alpha$に移るため、
\begin{align*}
U_{p+1} &= \frac{\alpha^{p+1} - \beta^{p+1}}{\alpha - \beta}\\
&= \frac{\alpha^p\alpha - \beta^p\beta}{\alpha-\beta}\\
&= \frac{\beta\alpha - \alpha\beta}{\alpha-\beta}\\
&= 0
\end{align*}
となるためである。

以上がFibonacci数列とLucas数列の裏に隠れていたカラクリである。

\begin{Theo}{\IND{2次Frobeniusテスト}{2しFrobeniusてすと}, quadratic frobenius primality test}{quadratic_frobenius_primality_test}
$a,b$を$\Delta=a^2-4b$が平方数にならないような整数とする。
また、$n$は$\gcd(n,2b\Delta)=1$を満たすとする。
$n$が素数ならば、次を満たす。
\begin{align*}
U_{n - \left(\frac{\Delta}{n}\right)} &\equiv 0 \pmod{n}\\
V_{n - \left(\frac{\Delta}{n}\right)} &\equiv
\begin{cases}
2b, &\mbox{if } \left(\frac{\Delta}{n}\right) = -1\\
2,  &\mbox{if } \left(\frac{\Delta}{n}\right) = 1
\end{cases}
\end{align*}
\end{Theo}
