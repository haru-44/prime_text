LucasテストやPepinテストは、$n-1$の素因数分解を利用した素因数判定法であった。
一般に$n-1$の素因数を求めることは難しいが、特殊な形の数\Notes{例えば、Fermat数$F_k=2^{2^k}+1$}に対しては都合よく適用できる。
同じように、$n+1$の素因数分解を利用した素数判定法を、Lucas数列テストから構成する。
$n-1$と同様に$n+1$の素因数を求めることは難しいが、特殊な形の数(具体的には\IND{Mersenne数}{Mersenneすう}$M_n=2^n-1$)に対しては都合よく適用できる。
それがLucas-Lehmerテストである。

Lucas数列テストでは、
\begin{align*}
f(x)=x^2-ax+b, \Delta = a^2-4b
\end{align*}
と置き、方程式の根を$\alpha,\beta$としたとき
\begin{align*}
U_n &= \frac{\alpha^n - \beta^n}{\alpha - \beta}\\
V_n &= \alpha^n - \beta^n
\end{align*}
を考えたのであった。

ここで、$\left(\frac{\Delta}{n}\right)=-1$の場合に限定して考える。
すると、LucasテストやPepinテスト、Pocklingtonテストが$n-1$の素因数分解を利用したのに対して、$n+1$の素因数分解を利用する素数判定法を得られる。

\begin{Theo}{}{p_plus_1_u}
$f(x)=x^2-ax+b, \Delta = a^2-4b$とし、$n$は$\gcd(n,2b)=1$および$\left(\frac{\Delta}{n}\right)=-1$を満たすとする。
$n+1=FR$としたとき、$q$は$F$の任意の素因数とする。
次を満たすとき、$n$の任意の素因数$p$は$p\equiv\left(\frac{\Delta}{p}\right)\pmod{F}$を満たす。
特に、$F>\sqrt{n}+1$ならば$n$は素数である。
\begin{align*}
\begin{cases}
U_{n+1} \equiv 0 \pmod{n}\\
\gcd(U_{(n+1)/q}, n) = 1
\end{cases}
\end{align*}
\end{Theo}

$V_n$でも同様の定理が考えられる。

\begin{Theo}{}{p_plus_1_v}
$f(x)=x^2-ax+b, \Delta = a^2-4b$とし、$n$は$\gcd(n,2b)=1$および$\left(\frac{\Delta}{n}\right)=-1$を満たすとする。
$n+1=FR$としたとき(ただし、$F$は偶数)、$q$は$F$の$2$以外の任意の素因数とする。
次を満たすとき、$n$の任意の素因数$p$は$p\equiv\left(\frac{\Delta}{p}\right)\pmod{F}$を満たす。
特に、$F>\sqrt{n}+1$ならば$n$は素数である。
\begin{align*}
\begin{cases}
V_{F/2} \equiv 0 \pmod{n}\\
\gcd(V_{F/2q}, n) = 1
\end{cases}
\end{align*}
\end{Theo}

Lucas-Lehmerテストは、Mersenne数$M_n=2^n-1$に対して\rTheo{p_plus_1_v}を適用した素数判定法である。
ただし、Lucas-Lehmerテストは必要十分条件であり、素数か合成数かを確実に判定する。

\begin{Theo}{\IND{Lucas-Lehmerテスト}{Lucas-Lehmerてすと}, Lucas-Lehmer test}{Lucas_Lehmer_test}
数列$\{v_k\}$を、$v_0=4,v_{k+1}=v_k^2-2$と定める。
奇素数$q$に対して、$v_{q-2} \equiv 0 \pmod{M_q}$を満たすとき、かつそのときのみ、Mersenne数$M_q$は素数である。
\end{Theo}

証明の流れを見てみよう。
$a=4,b=1$とするLucas数列$\{V_k\}$は、特性多項式$f(x)=x^2-4x+1$であり、$\Delta=4^2-4\cdot1=12$である。
特性方程式$f(x)$の根を$\alpha,\beta$とするとき、$x^2-4x+1=(x-\alpha)(x-\beta)=x^2-(\alpha+\beta)x+\alpha\beta$である\Notes{多項式の係数と根の関係という高校数学の知識だ。}から、
\begin{align*}
\alpha + \beta &= 4\\
\alpha\beta &= 1
\end{align*}
が得られる。
このことから、$\alpha\beta=\alpha(4-\beta)=1$が分かり、また$f(\alpha) = \alpha^2 - 4\alpha + 1 = 0$であることも併せて
\begin{itemize}
\item $x(4-x)\equiv1\pmod{f(x)}$
\item $x^2 \equiv 4x - 1 \pmod{f(x)}$
\end{itemize}
という2本の式が得られる。

さらに$M_q$が素数のとき、Frobenius写像(\rDefi{Frobenius_map}\Notes{元を$M_q$乗するのであった})は根の置換を引き起こす。
雰囲気的には$\alpha^{M_q}=\beta,\beta^{M_q}=\alpha$が起きるのだが、丁寧に書けば、
\begin{itemize}
\item $x^{M_q} \equiv (4 - x) \pmod{f(x),M_q}$
\item $(4-x)^{M_q} \equiv x \pmod{f(x),M_q}$
\end{itemize}
となる。
2次Frobeniusテストでは、Legendre記号の値によって根の置換か恒等写像かが分かれることを見たが、ここで考えるLegendre記号の値は常に$-1$であると主張する\rProp{ll_sub_2}のお陰で根の置換のみを考えればよい。

\begin{Prop}{}{ll_sub_2}
$n\ge3$が奇数のとき、次が成り立つ。
\begin{align*}
\left(\frac{12}{M_n}\right)=-1
\end{align*}
\end{Prop}

そして、$\{V_k\}$と$\{v_k\}$をつなぐ次の命題は重要である。

\begin{Prop}{}{ll_sub_1}
任意の$k\ge1$について
\begin{align*}
V_{2^k} = v_k
\end{align*}
\end{Prop}

前提条件が揃ったので、$F=2^{q-1}$と置いて、\rTheo{p_plus_1_v}を適用する。
すると、$V_{F/2} = V_{2^{q-2}} \equiv 0 \pmod{M_q}$という簡単な条件となる\Notes{一応述べておくが、2つ目の最大公約数の条件は、$F$の素因数が$2$しかないことから、考える必要はない。}。
さらに\rProp{ll_sub_1}を適用すると$v_{q-2} \equiv 0 \pmod{M_q}$が得られる。
以上より、$v_{q-2} \equiv 0 \pmod{M_q}$ならば$M_q$は素数である。

そして、$M_q$が素数であるとして、$V_{2^{q-2}}\equiv0\pmod{M_q}$を導出しよう。
以下の命題では、$M_q$が素数であるという前提や、特性方程式が$f(x)=x^2-4x+1$であることの明言を端折っているため注意する。

\begin{Prop}{}{ll_sub_3}
$x^{2^{q-1}}\equiv-1\pmod{f(x),M_q}$
\end{Prop}

これより、次の命題が言える。

\begin{Prop}{}{ll_sub_4}
$U_{2^{q-1}}\equiv0\pmod{M_q}$
\end{Prop}

最後に、次の命題を証明することで$M_q$が素数ならば$v_{q-2} \equiv 0 \pmod{M_q}$であると言える。

\begin{Prop}{}{ll_sub_5}
$V_{2^{q-2}}\equiv0\pmod{M_q}$
\end{Prop}

アルゴリズムの実装を見てみよう。

\Algo{Lucas-Lehmerテスト}{lucas_lehmer_primality_test}{}

その裏で必要となる整数論の難解さは別にして、アルゴリズム自体は驚くほど単純である。
$p-2$回の2乗算を高速化する余地はあるが、ここでは言及しない。

Mersenne素数が大きな素数の上位を独占しているのは、Lucas-Lehmerテストに拠るところが大きいが、Mersenne数自体にも都合の良い性質がいくつもある。

\begin{Prop}{}{}
$M_q$が素数ならば$q$は素数である。
\end{Prop}

残念ながら逆は正しくない。
しかし、Mersenne素数を探すなら$q$が素数のみで良いことが分かる。
更に、

\begin{Prop}{}{}
$q$を奇素数とする。
$M_q$の任意の素因数$r$は、$r\equiv1\pmod{q}$かつ$r\equiv\pm1\pmod{8}$を満たす。
\end{Prop}

ということが分かっているので、$M_q$の任意の素因数$r$は、$r=2qt+1$の形で表される。
試し割をするにも、$p-1$法を実行するにも都合がいいことが了解されよう。

また、Catalan予想は、1844年Catalanが予想し、2002年Mihăilescuによって証明された\Notes{そのため、「予想」と呼ばず、Mihăilescuの定理と呼ぶこともある。}\cite{Catalan's_conjecture}。

\begin{Theo}{\IND{Catalan予想}{Catalanよそう}, Catalan's conjecture}{Catalan's conjecture}
次を満たす1より大きい正整数$x,n,y,m>1$は、$(x,n,y,m)=(3,2,2,3)$のみである。
\begin{align*}
x^n - y^m = 1
\end{align*}
\end{Theo}

この事実を用いれば、

\begin{Coro}{}{Catalan_Mersenne}
$n$が1より大きい正整数であるとすると、$M_n$は累乗数ではない。
\end{Coro}

ということも分かる。
もっと言えば、$M_q$は平方因子を持たないようにも見えるが、未証明である。

これらの事実から、Mersenne素数の探索するには、素数$q$のみを考えればよく、試し割や$p-1$法によって篩にかけ、残った$M_q$に対してのみLucas-Lehmerテストを実施してやればよい。
