Fibonacci数列は、1202年Leonardo Fibonacciによって著わされた『算盤の書(Liber Abaci)』に記載された。
$0,1,1,2,3,5,8,\ldots$と続く数列\Notes{\url{https://oeis.org/A000045}}で、定義は次の通りである。

\begin{Defi}{\IND{Fibonacci数列}{Fibonacciすうれつ}, Fibonacci sequence}{Fibonacci_sequence}
Fibonacci数列$\{F_n\}$は、整数$n\ge0$について
\begin{align*}
F_n =
\begin{cases}
0, &\mbox{if } n = 0\\
1, &\mbox{if } n = 1\\
F_{n - 1} + F_{n - 2}, &\mbox{if } n \ge 2
\end{cases}
\end{align*}
で定義される数列である。
\end{Defi}

Fibonacciは、ウサギの増え方に関する問題として扱った。
自然現象に数多く出現する数列として、数学を専門にしない人でも「Fibonacci数列」という名称は比較的聞いたことがあるかも知れない。
しかし、自然界に出現することはFibonacci数列の一面でしかない。

\Algo{Fibonacci数列}{fibonacci_sequence}{}

上記のように、漸化式を地道に計算していく以外にも、一本の式でも求められる。
それがFibonacci数列の一般項と呼ばれるものだ。
\begin{align*}
F_n = \frac{\phi^n - (1 - \phi)^n}{\phi - (1 - \phi)}
\end{align*}
ただし、
\begin{align*}
\phi = \frac{1 + \sqrt{5}}{2} = 1.618\ldots
\end{align*}
である。
この$\phi$は\IND{黄金数}{おうこんすう}(golden number)とも呼ばれる。
ちなみに、$F_n$の分母はあえて整理していない。
驚くべきは、黄金数が整数(あるいは有理数)ではなく無理数であるということだ。
右辺は無理数を含んでいるが、左辺は整数になるということは一見すると不可解である。

実は、$\phi$が現れる裏には、特性方程式と呼ばれる方程式が存在する。
導出過程には触れないが、Fibonacci数列の特性方程式は、$x^2-x-1=0$であり、この方程式の2つの解を$\alpha, \beta$と置くとFibonacci数列の一般項は
\begin{align*}
F_n = \frac{\alpha^n - \beta^n}{\alpha - \beta}
\end{align*}
とも書ける。
実は、$\alpha=\phi,\beta=(1-\phi)$となるのであり、「単に$\phi$から$\alpha, \beta$に文字を書き換えただけではないか」という感覚はまったく正しくて、現段階で何ら益はない。
後々で一般化するにあたり必要な考え方であり、とりあえずそういうものだ、と頭の片隅に置いておけばよい。

Fibonacci数列の一般項は分かったが、面白みに欠ける。
一般項はこうです、ふーんそうなんだ、というのではやはりお仕着せ感がある。
また、素数とも関わらないように思える。
そういうわけだから、ちょっと手を動かしてみよう。
何をするかというと、Fibonacci数列を素数$p$で割った余りを求めてみるのである。
ちょっとやる気になった御仁はそのまま進めて、恐らく答えを知りたい大多数の人向けに、$p=5,7,11$での表を示しておく(表\ref{table:fibonacci_example})。

\begin{table}[htb]\label{table:fibonacci_example}
 \begin{center}
    \caption{Fibonacci数列を$p$で割った余り}
  \begin{tabular}{|l|l|l|l|l|l|l|l|l|l|l|l|l|l|l|l|l|l|}\hline
    $p$ & $F_0$ & $F_1$ & $F_2$ & $F_3$ & $F_4$ & $F_5$ & $F_6$ & $F_7$ & $F_8$ & $F_9$ & $F_{10}$ & $F_{11}$ & $F_{12}$ & $F_{13}$ & $F_{14}$ & $F_{15}$ & $F_{16}$ \\ \hline\hline
    5   & 0     & 1     & 1     & 2     & 3     & 0     & 3     & 3     & 1     & 4     & 0        & 4        & 4        & 3        & 2        & 0        & 2        \\ \hline
    7   & 0     & 1     & 1     & 2     & 3     & 5     & 1     & 6     & 0     & 6     & 6        & 5        & 4        & 2        & 6        & 1        & 0        \\ \hline
    11  & 0     & 1     & 1     & 2     & 3     & 5     & 8     & 2     & 10    & 1     & 0        & 1        & 1        & 2        & 3        & 5        & 8        \\ \hline
  \end{tabular}
  \end{center}
\end{table}

この観察から何が得られるか。
ゼロの出現する箇所を詳しく見てみると、$F_0=0$なので、最初は必ずゼロになる。
その後、どのようにゼロが出現するかであるが、何かしらの法則性が隠れていそうな雰囲気を醸し出している。
\begin{itemize}
 \item $p = 5$のとき、$F_0, F_5, F_{10}, F_{15}$ $\Rightarrow$ 5間隔でゼロが現れる。
 \item $p = 7$のとき、$F_0, F_8, F_{16}$ $\Rightarrow$ 8間隔でゼロが現れる。
 \item $p = 11$のとき、$F_0, F_{10}$ $\Rightarrow$ 10間隔でゼロが現れる。
\end{itemize}

なるほど、0が周期的に現れている。
しかし、$p=5$のとき5間隔だから、$p=7$のときも7間隔かと思いきや、8間隔である。
$p+1$間隔になることもあるのかと思ったら、$p=11$のときは10間隔である。
これを明らかにするのが次の定理である。

\begin{Theo}{}{fibonacci_prime}
$n$が素数ならば、
\begin{align*}
F_{n - \left(\frac{n}{5}\right)} \equiv 0 \pmod{n}
\end{align*}
が成り立つ。
ここで$\left(\frac{n}{5}\right)$はLegendre記号である。
\end{Theo}

$\left(\frac{n}{5}\right)$はLegendre記号だと言うのだから、次のように扱ってもよい。
\begin{align*}
\left(\frac{n}{5}\right) =
\begin{cases}
1, &\mbox{if }n \equiv \pm 1 \pmod{5}\\
-1, &\mbox{if }n \equiv \pm 2 \pmod{5}\\
0, &\mbox{otherwise}
\end{cases}
\end{align*}

例えば、$n=5$のとき、$n$は素数だから$F_{5 - 0} = 0 \pmod{5}$
が得られるが、これは既に計算されているように正しい。
同様に、$n=7$のとき、$n$は素数だから
$F_{7 - (-1)} = F_{8} = 0 \pmod{7}$
も正しいし、$n=11$のとき、$n$は素数だから
$F_{11 - 1} = F_{10} = 0 \pmod{11}$も正しい。

「素数ならば\ruby{斯}{か}くなる数式を満たす」という形の定理は、そのまま素数判定に利用できることは既に見た。
Fibonacci数列もまた、(実用的かはともかく)素数判定に用いることができる。
この素数判定も、合成数を誤って「素数」と判定することがあり得る。

\Algo{Fibonacci数列テスト}{fibonacci_pseudoprime_test}{c.f., \rAlgo{fibonacci_sequence}, \rAlgo{legendre_symbol}}

\begin{thProof}{fibonacci_prime}
Fibonacci数列の一般項は、
\begin{align*}
F_n = \frac{\alpha^n - \beta^n}{\sqrt{5}}
\end{align*}
とも書ける\Notes{$\alpha-\beta=\sqrt{5}$に注意}。
両辺に$2^n\sqrt{5}$を掛けて、$\alpha,\beta$を展開すると、
\begin{align*}
2^n\sqrt{5}F_n = (1+\sqrt{5})^n - (1-\sqrt{5})^n
\end{align*}
となる。
右辺の各項に二項定理を適用すると、右辺はそれぞれ
\begin{align*}
{n \choose 0}\sqrt{5}^0       &+& {n \choose 1}\sqrt{5}^1       &+ \cdots +& {n \choose n-1}\sqrt{5}^{n-1}           &+& {n \choose n}\sqrt{5}^n\\
{n \choose 0}(-1)^0\sqrt{5}^0 &+& {n \choose 1}(-1)^1\sqrt{5}^1 &+ \cdots +& {n \choose n-1}(-1)^{n-1}\sqrt{5}^{n-1} &+& {n \choose n}(-1)^n\sqrt{5}^n
\end{align*}
となって、上から下を引くと奇数乗の項のみが残る。
つまり、
\begin{align}
\label{catalan_equation_fibonacci_prime}
2^n F_n = 2 \left[ {n \choose 1} + {n \choose 3}5 + {n \choose 5}5^2 + \cdots \right]
\end{align}
を得る。

式(\ref{catalan_equation_fibonacci_prime})に$n=p-1$を代入すると、
\begin{align*}
2^{p-1}F_{p-1} = 2 \left[ {p-1 \choose 1} + {p-1 \choose 3}5 + {p-1 \choose 5}5^2 + \cdots \right]
\end{align*}
となるが、$p$が素数のとき、$0 < k < p$において
\begin{align*}
{p-1 \choose k} \equiv (-1)^k \pmod{p}
\end{align*}
であることと、等比数列の和の公式より、
\begin{align*}
2^{p-1}F_{p-1} &\equiv -2 (1 + 5 + \cdots + 5^{(p-3)/2}) \pmod{p}\\
&\equiv -2 \frac{5^{(p-1)/2} - 1}{5-1}\pmod{p}
\end{align*}
を得る。
$5^{(p-1)/2}\pmod{p}$は、$p\equiv\pm1\pmod{5}$のとき、1となるから、このとき$F_{p-1}\equiv 0\pmod{p}$を得る。

同様に、式(\ref{catalan_equation_fibonacci_prime})に$n=p+1$を代入すると、
\begin{align*}
2^{p+1}F_{p+1} = 2 \left[ {p+1 \choose 1} + {p+1 \choose 3}5 + {p+1 \choose 5}5^2 + \cdots \right]
\end{align*}
となるが、$p$が素数のとき、$1 < k < p$において
\begin{align*}
{p+1 \choose k} \equiv 0 \pmod{p}
\end{align*}
であるから、
\begin{align*}
2^{p+1}F_{p+1} \equiv 2 (1 + 5^{(p-1)/2}) \pmod{p}
\end{align*}
を得る。
$5^{(p-1)/2}\pmod{p}$は、$p\equiv\pm2\pmod{5}$のとき、$-1$となるから、このとき$F_{p+1}\equiv 0\pmod{p}$を得る。
\end{thProof}
