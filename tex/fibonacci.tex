\subsubsection{Fibonacci数列}
Fibonacci数列は、1202年Leonardo Fibonacciによって著わされた『算盤の書(Liber Abaci)』に記載された。
$0,1,1,2,3,5,8,\ldots$と続く数列\Notes{\url{https://oeis.org/A000045}}で、定義は次の通りである。

\begin{Defi}{\IND{Fibonacci数列}{Fibonacciすうれつ}, Fibonacci sequence}{Fibonacci_sequence}
Fibonacci数列$\{F_n\}$は、整数$n\ge0$について
\begin{align*}
F_n =
\begin{cases}
0, &\mbox{if } n = 0\\
1, &\mbox{if } n = 1\\
F_{n - 1} + F_{n - 2}, &\mbox{if } n \ge 2
\end{cases}
\end{align*}
で定義される数列である。
\end{Defi}

Fibonacciは、ウサギの増え方に関する問題として扱った。
自然現象に数多く出現する数列として、数学を専門にしない人でも「Fibonacci数列」という名称は比較的聞いたことがあるかも知れない。
しかし、自然界に出現することはFibonacci数列の一面でしかない。

\Algo{Fibonacci数列}{fibonacci_sequence}{}

上記のように、漸化式を地道に計算していく以外にも、一本の式でも求められる。
それがFibonacci数列の一般項と呼ばれるものだ。
\begin{align*}
F_n = \frac{\phi^n - (1 - \phi)^n}{\phi - (1 - \phi)}
\end{align*}
ただし、
\begin{align*}
\phi = \frac{1 + \sqrt{5}}{2} = 1.618\ldots
\end{align*}
である。
この$\phi$は\IND{黄金数}{おうこんすう}(golden number)とも呼ばれる。
ちなみに、$F_n$の分母はあえて整理していない。
驚くべきは、黄金数が整数(あるいは有理数)ではなく無理数であるということだ。
右辺は無理数を含んでいるが、左辺は整数になるということは一見すると不可解である。

実は、$\phi$が現れる裏には、特性方程式と呼ばれる方程式が存在する。
導出過程には触れないが、Fibonacci数列の特性方程式は、$x^2-x-1=0$であり、この方程式の2つの解を$\alpha, \beta$と置くとFibonacci数列の一般項は
\begin{align*}
F_n = \frac{\alpha^n - \beta^n}{\alpha - \beta}
\end{align*}
とも書ける。
実は、$\alpha=\phi,\beta=(1-\phi)$となるのであり、「単に$\phi$から$\alpha, \beta$に文字を書き換えただけではないか」という感覚はまったく正しくて、現段階で何ら益はない。
後々で一般化するにあたり必要な考え方であり、とりあえずそういうものだ、と頭の片隅に置いておけばよい。

ちなみに、地道な計算はすぐに限界を来す。
一般に、線形漸化式
\begin{align*}
\theta_n = a_1\theta_{n-1} + a_2\theta_{n-2} + \cdots + a_d\theta_{n-d}
\end{align*}
は、
\begin{align*}
\begin{pmatrix}
\theta_{n}\\
\theta_{n+1}\\
\vdots\\
\theta_{n+d-1}
\end{pmatrix}
=
\begin{pmatrix}
0 & 1 & 0 & \cdots & 0\\
0 & 0 & 1 & \cdots & 0\\
\vdots & \vdots & \ddots& \ddots & \vdots\\
0 &0&\cdots&0& 1\\
a_d & a_{d-1} & \cdots & a_{2} & a_{1}
\end{pmatrix}^n
\begin{pmatrix}
\theta_0\\
\theta_1\\
\vdots\\
\theta_{d-1}
\end{pmatrix}
\end{align*}
が成り立つ\Notes{$n=1$の場合を考えてみると、$d-1$行目まで右辺の縦ベクトルの要素を1つ上にスライドさせ、$d$行目では定義式を使って左辺縦ベクトルの$d$行目を作っていることが分かる。}。
よって、Fibonacci数列の場合は、
\begin{align*}
\begin{pmatrix}
F_{n} \\
F_{n+1}
\end{pmatrix} = 
\begin{pmatrix}
0 & 1 \\
1 & 1
\end{pmatrix}^n
\begin{pmatrix}
0 \\
1
\end{pmatrix}
\end{align*}
となるので、
\begin{align*}
\begin{pmatrix}
F_{n-1} & F_n \\
F_{n} & F_{n+1}
\end{pmatrix} = 
\begin{pmatrix}
0 & 1 \\
1 & 1
\end{pmatrix}^n
\end{align*}
あるいは同じことだが
\begin{align*}
\begin{pmatrix}
F_{n+1} & F_n \\
F_{n} & F_{n-1}
\end{pmatrix} = 
\begin{pmatrix}
1 & 1 \\
1 & 0
\end{pmatrix}^n
\end{align*}
が得られる\Notes{線形漸化式から行列を構成した方法を思い返せば、行の要素を上下逆転させても結果は変わらないことが分かる。}。
よって、行列のべき乗演算ができればFibonacci数列が求められるわけで、それを実現したのが次の実装だ。

\Algo{Fibonacci数列(行列版)}{fibonacci_sequence_matrix}{c.f., \rAlgo{n_times}}

行列を使ったFibonacci数列の計算は、Fibonacci数列の計算する上では有用だが、素数判定という文脈においてはさほど触れなくてもよい知識なので、このくらいで次に進む。

\subsubsection{Fibonacci数列と素数}
Fibonacci数列の一般項は分かったが、面白みに欠ける。
一般項はこうです、ふーんそうなんだ、というのではやはりお仕着せ感がある。
また、素数とも関わらないように思える。
そういうわけだから、ちょっと手を動かしてみよう。
何をするかというと、Fibonacci数列を素数$p$で割った余りを求めてみるのである。
ちょっとやる気になった御仁はそのまま進めて、恐らく答えを知りたい大多数の人向けに、$p=5,7,11$での表を示しておく(表\ref{table:fibonacci_example})。

\begin{table}[htb]\label{table:fibonacci_example}
 \begin{center}
    \caption{Fibonacci数列を$p$で割った余り}
  \begin{tabular}{|l|l|l|l|l|l|l|l|l|l|l|l|l|l|l|l|l|l|}\hline
    $p$ & $F_0$ & $F_1$ & $F_2$ & $F_3$ & $F_4$ & $F_5$ & $F_6$ & $F_7$ & $F_8$ & $F_9$ & $F_{10}$ & $F_{11}$ & $F_{12}$ & $F_{13}$ & $F_{14}$ & $F_{15}$ & $F_{16}$ \\ \hline\hline
    5   & 0     & 1     & 1     & 2     & 3     & 0     & 3     & 3     & 1     & 4     & 0        & 4        & 4        & 3        & 2        & 0        & 2        \\ \hline
    7   & 0     & 1     & 1     & 2     & 3     & 5     & 1     & 6     & 0     & 6     & 6        & 5        & 4        & 2        & 6        & 1        & 0        \\ \hline
    11  & 0     & 1     & 1     & 2     & 3     & 5     & 8     & 2     & 10    & 1     & 0        & 1        & 1        & 2        & 3        & 5        & 8        \\ \hline
  \end{tabular}
  \end{center}
\end{table}

この観察から何が得られるか。
ゼロの出現する箇所を詳しく見てみると、$F_0=0$なので、最初は必ずゼロになる。
その後、どのようにゼロが出現するかであるが、何かしらの法則性が隠れていそうな雰囲気を醸し出している。
\begin{itemize}
 \item $p = 5$のとき、$F_0, F_5, F_{10}, F_{15}$ $\Rightarrow$ 5間隔でゼロが現れる。
 \item $p = 7$のとき、$F_0, F_8, F_{16}$ $\Rightarrow$ 8間隔でゼロが現れる。
 \item $p = 11$のとき、$F_0, F_{10}$ $\Rightarrow$ 10間隔でゼロが現れる。
\end{itemize}

なるほど、0が周期的に現れている。
しかし、$p=5$のとき5間隔だから、$p=7$のときも7間隔かと思いきや、8間隔である。
$p+1$間隔になることもあるのかと思ったら、$p=11$のときは10間隔である。
これを明らかにするのが次の定理である。

\begin{Theo}{}{fibonacci_prime}
$n$が素数ならば、
\begin{align*}
F_{n - \left(\frac{n}{5}\right)} \equiv 0 \pmod{n}
\end{align*}
が成り立つ。
ここで$\left(\frac{n}{5}\right)$はLegendre記号である。
\end{Theo}

$\left(\frac{n}{5}\right)$はLegendre記号だと言うのだから、次のように扱ってもよい。
\begin{align*}
\left(\frac{n}{5}\right) =
\begin{cases}
1, &\mbox{if }n \equiv \pm 1 \pmod{5}\\
-1, &\mbox{if }n \equiv \pm 2 \pmod{5}\\
0, &\mbox{otherwise}
\end{cases}
\end{align*}

例えば、$n=5$のとき、$n$は素数だから$F_{5 - 0} \equiv 0 \pmod{5}$
が得られるが、これは既に計算されているように正しい。
同様に、$n=7$のとき、$n$は素数だから
$F_{7 - (-1)} = F_{8} \equiv 0 \pmod{7}$
も正しいし、$n=11$のとき、$n$は素数だから
$F_{11 - 1} = F_{10} \equiv 0 \pmod{11}$も正しい。

「素数ならば\ruby{斯}{か}くなる数式を満たす」という形の定理は、そのまま素数判定に利用できることは既に見た。
Fibonacci数列もまた、(実用的かはともかく)素数判定に用いることができる\Notes{\rAlgo{fibonacci_pseudoprime_test}の実装は\rTheo{fibonacci_prime}を愚直に実装しているが、後で別の実装も紹介する。}。

\Algo{Fibonacci数列テスト}{fibonacci_pseudoprime_test}{c.f., \rAlgo{fibonacci_sequence}, \rAlgo{legendre_symbol}}

この素数判定も、合成数を誤って「素数」と判定することがあり得る\Notes{小さい方から、$25, 60, 120, 125,\ldots$と続く。\url{https://oeis.org/A241505}}が、他の素数判定法と組み合わせることが可能である。
例えば、
\begin{itemize}
 \item $n \equiv \pm 2 \pmod{5}$
 \item 底$a=2$のFermatテスト(\rAlgo{fermat_test})で素数と判定される。
 \item Fibonacci数列テスト(\rAlgo{fibonacci_pseudoprime_test})で素数と判定される。
\end{itemize}
となるような合成数$n$は見つかっていない。
これは、素数判定に関する\IND{Selfridge予想}{Selfridgeよそう}(Selfridge's conjecture)あるいは\IND{PSW予想}{PSWよそう}として知られ、真であることまたは反例を見つけた者に620ドルの賞金が送られることになっている\cite{A_Computational_Perspective}。

\subsubsection{Lucas数列}
Fibonacci数列を使って素数判定ができた。
Lucas数列はFibonacci数列の一般化であるが、ならば同様にLucas数列でも素数判定ができるのではないか。
そういう思考を働かせて、Lucas数列を導入する。
ちなみにLucas\kenten{数}というのもあって、混乱しやすいこと甚だしい。
ここで使うのは、Lucas\kenten{数列}だからお間違いのないように。

\begin{Defi}{\IND{Lucas数列}{Lucasすうれつ}, Lucas sequence}{Lucas_sequence}
Lucas数列$\{U_n\},\{V_n\}$は、次のように定義される数列である。
\begin{align*}
U_n =
\begin{cases}
0, &\mbox{if } n = 0\\
1, &\mbox{if } n = 1\\
aU_{n - 1} - bU_{n - 2}, &\mbox{if } n \ge 2
\end{cases}
\end{align*}
\begin{align*}
V_n =
\begin{cases}
2, &\mbox{if } n = 0\\
a, &\mbox{if } n = 1\\
aV_{n - 1} - bV_{n - 2}, &\mbox{if } n \ge 2
\end{cases}
\end{align*}
あるいは、同値であるが、
\begin{align*}
U_n &= \frac{\alpha^n - \beta^n}{\alpha - \beta}\\
V_n &= \alpha^n + \beta^n
\end{align*}
と定義される数列である。ここで、$\alpha, \beta$は、2次方程式$x^2-ax+b=0$の解
\begin{align*}
\alpha &= \frac{a+\sqrt{\Delta}}{2}\\
\beta &= \frac{a-\sqrt{\Delta}}{2}
\end{align*}
である。ただし、$\Delta=a^2 - 4b$は平方数でないとする。
\end{Defi}

一応、後々のために$\{V_n\}$も一緒に述べたが、一旦$\{U_n\}$のみで考えることにする。

\Algo{Lucas数列}{lucas_sequence}{}

Lucas数列はFibonacci数列の一般化であると言ったが、$a=1, b=-1$のとき、$U_n$はFibonacci数列に一致する\Notes{なお、$a=1, b=-1$のときの$V_n$がLucas数と呼ばれる。}。
\begin{align*}
\alpha &= \frac{1 + \sqrt{1^2 - 4 \times (-1)^2}}{2}\\
&= \frac{1 + \sqrt{5}}{2}\\
&= \phi
\end{align*}

Lucas数列にも、Fibonacci数列が持っていた素数に関する性質がある。
ここで、$\left(\frac{a}{p}\right)$はLegendre記号である。

\begin{Theo}{}{lucas_sequence}
$\Delta=a^2-4b$は平方数でないとする。$n$が$\gcd(n, 2b\Delta)=1$を満たす素数ならば、
\begin{align*}
U_{n - \left(\frac{\Delta}{n}\right)} \equiv 0 \pmod{n}
\end{align*}
が成り立つ。
\end{Theo}

Lucas数列がFibonacci数列の一般化だというのだから、定理もまた一般化されているはずである。
Fibonacci数列の場合を考えてみる。
$\Delta = 1^2-4\times(-1)^2=5$であるから、$\left(\frac{\Delta}{n}\right)$は、$\left(\frac{5}{n}\right)$となる。
平方剰余の相互法則(\rTheo{legendre_quadratic_reciprocity})より、分子と分母(分数ではないので、このような呼称は適切ではないが)が反転可能である。

$n$は素数と仮定しているから、5も$n$も素数であるとしてよい。
$n=2, 5$の場合を棚上げにして、平方剰余の相互法則を適用してみる。
\begin{align*}
\left(\frac{5}{n}\right) \bigg(\frac{n}{5}\bigg) &= (-1)^{\frac{5-1}{2}\cdot\frac{n-1}{2}}\\
&= (-1)^{n-1}\\
&= 1
\end{align*}
$n-1$は常に偶数になるから、$(-1)^{n-1}$は常に$1$になる。
よって、$\left(\frac{5}{n}\right)$と$\left(\frac{n}{5}\right)$は常に同じ値になる(Legendre記号は$0$を除き$-1,1$しかとらないことに注意)。
$n=2, 5$のときも、個別に値を調べれば$\left(\frac{2}{5}\right) = \left(\frac{5}{2}\right) = -1$であるから一致する。
つまり、Lucas数列での定理は、Fibonacci数列の定理の一般化である。

Fibonacci数列のときと同じように、定理から素数判定アルゴリズムを作ることができる。

\Algo{Lucas数列テスト}{lucas_sequence_test}{c.f., \rAlgo{legendre_symbol}, \rAlgo{lucas_sequence}}

やはりこの素数判定法も、合成数を誤って「素数」と判定してしまうことがある。
例えば、$323=17\times19$は、合成数であるが、$a=-10,b=-5$のとき``probable prime"を返す。

