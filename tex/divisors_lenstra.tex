「$n$の因数は、$s$を法として$r$と合同」という事実が判明したとする。
このことは、$n$の素数判定あるいは素因数分解に役立つだろうか? すぐに、試し割は$xs+r$の形のみを行えば良いことが分かる\Notes{$x$を動かせばよい}。
他にも$r=1$なら$p-1$法が使えるし、$r=s-1$なら$p+1$法が使える。

ここでは、Lenstraの結果\cite{divisors_lenstra}を紹介しよう。
このアルゴリズムは、$s$を法として$r$と合同な$n$の非自明な約数を列挙することができる。
残念ながら、このアルゴリズムには人口に膾炙した名前がない\Notes{ふつう「Lenstra法」と言えば楕円曲線を使った素因数分解アルゴリズムを指す。}。

仮定より、
\begin{align*}
n = (xs+r)(ys+r')
\end{align*}
と書ける、$x,y\ge0$と$0<r'<s$が存在する。
数列$a,b,c$を次のように定義しよう。
ただし、$r^{-1}$は、$s$を法とした$r$の逆元で、$q_i$は、$a_i$を計算する過程で一意に決まる。

\begin{align*}
a_i &= 
\begin{cases}
s, &\mbox{if } i = 0\\
r^{-1}r' \bmod{s}, \; 0< a_i \le s, &\mbox{if } i = 1\\
a_{i-2} - q_ia_{i-1}, \; 0< a_i \le s, &\mbox{if } 2 \le i \mbox{ かつ } i \mbox{ は偶数}\\
a_{i-2} - q_ia_{i-1}, \; 0\le a_i < s, &\mbox{if } 2 \le i \mbox{ かつ } i \mbox{ は奇数}
\end{cases}
\\
b_i &=
\begin{cases}
0, &\mbox{if } i = 0\\
1, &\mbox{if } i = 1\\
b_{-2} - q_ib_{i-1}, &\mbox{otherwise}
\end{cases}
\\
c_i &=
\begin{cases}
0, &\mbox{if } i = 0\\
\frac{n-rr'}{s}\cdot r^{-1} \bmod{s}, &\mbox{if } i = 1\\
c_{i-2}-q_ic_{i-1} \bmod{s}, &\mbox{otherwise}
\end{cases}
\end{align*}
この数列は、$a_t=0$となる$t$まで続ける。
すると、次が成り立つ。

\begin{Prop}{}{}
$0\le i \le t$なる任意の$i$について、
\begin{align*}
xa_i + yb_i \equiv c_i \pmod{s}
\end{align*}
を満たす。
\end{Prop}

我々の方針は、それぞれの$i$において$c' \equiv c_i \pmod{s}$を満たす$c'$をしらみつぶしに探す。
$c'$が\kenten{あたり}なら、連立方程式
\begin{align*}
\begin{cases}
xa_i + yb_i = c'\\
(xs+r)(ys+r')=n
\end{cases}
\end{align*}
を満たす$x,y$が求められ、目的が達成される。
これを実現するための2つの障害は、どの範囲の$c'$を探せばいいかということと、連立方程式を解くにはどうすればよいかということだ。

後者については、$u=a_i(xs+r), v=b_i(ys+r')$と置き、$u,v$が解となる2次方程式を作る。
$uv=a_ib_in, u+v=c's+a_ir+b_ir'$であることに注意すると、そのような2次方程式は、
\begin{align*}
T^2 - (c's+a_ir+b_ir')T + a_ib_in
\end{align*}
であることが分かる\Notes{2次方程式の解と係数の関係は、高校数学の範囲である(たぶん)。}。
この方程式に整数解があるかは、判別式
\begin{align*}
\Delta = (c's+a_ir+b_ir')^2 - 4a_ib_in
\end{align*}
が平方数であるかと等価である。
$\Delta$が平方数であれば、2次方程式の解の公式より$((c's+a_ir+b_ir')\pm\sqrt{\Delta})/2$が$u,v$である。
こうすることで、$xs+r$が求められるわけであるが、コーナーケースに注意しよう。
$b_0=0, a_t=0$では、2次方程式を利用するこの方法では上手くいかない。
幸いなことに、$i=0,t$のときに約数が見つかるというのは、$x=0$または$y=0$のときだけなので、実装上は別で判定し探索範囲は$0<i<t$としてよい。

次に、前者の問題を考えよう。
答えから言えば、$c'$が動く範囲は
\begin{align*}
\begin{cases}
-s < c' < s, &\mbox{if } i \mbox{ は偶数}\\
2a_ib_i < c' < a_ib_i + n/s^2, &\mbox{if } i \mbox{ は奇数}
\end{cases}
\end{align*}
で良い。
もちろん、$c' \equiv c_i \pmod{s}$という条件があるので、$i$が偶数の場合は高々2個のチェックで済む。

以上で実装に必要な知識は揃った。

\Algo{剰余類に含まれる因数}{divisors_lenstra}{c.f., \rAlgo{inverse_mod}, \rAlgo{is_square_number}, \rAlgo{sqrt_int}}

直感的に$s$が大きい方がすぐ終わるだろうと予想できるが、それは正しい。
一般に、\rAlgo{divisors_lenstra}が平方数であるかを求める回数は$O(\frac{n^{1/3}}{s} \log{n})$と評価できる。
なので、$n<s^3$であれば$O(\log{n})$回で済む。

