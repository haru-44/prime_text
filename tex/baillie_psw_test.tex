\subsubsection{擬素数の整理}
これまでに紹介してきた擬素数($+\alpha$)を整理してみよう。

まずは、Fermatテストをすり抜けるFermat擬素数があった。
Solovay-Strassenテストをすり抜けるのはEuler-Jacobi擬素数で、Miller-Rabinテストに対応するのが強擬素数だった。
これらの関係は次のようになる。

\begin{Theo}{}{}
任意のパラメータ$a$に対して
\begin{align*}
\mbox{強擬素数} \subseteq \mbox{Euler-Jacobi擬素数} \subseteq \mbox{Euler擬素数} \subseteq \mbox{Fermat擬素数}
\end{align*}
が成り立つ。
\end{Theo}

もちろん、擬素数は少ない方が良いので、強擬素数の方が人類にとっては最も嬉しい。
Euler擬素数は初めて登場したので、定義だけ述べておく。

\begin{Defi}{\IND{Euler擬素数}{Eulerきそすう}, Euler pseudoprime}{euler_pseudoprime}
奇数の合成数$n$は$a$と互いに素であるとする。
$n$が
\begin{align*}
a^{(n-1)/2} \equiv \pm 1 \pmod{n}
\end{align*}
を満たすとき、$n$を底$a$のEuler擬素数と呼ぶ。
\end{Defi}

他に、Lucas擬素数やFrobenius擬素数が登場したが、
\begin{align*}
\mbox{Frobenius擬素数} &\subseteq \mbox{Lucas擬素数}\\
\mbox{強Lucas擬素数} &\subseteq \mbox{Lucas擬素数}\\
\mbox{強Frobenius擬素数} &\subseteq \mbox{Frobenius擬素数}
\end{align*}
という半ば自明の関係がある。

\subsubsection{ニコイチ}
複数の素数判定法を組み合わせれば良くなるというのは、安直ながら強力な方法である。
例えば、
\begin{itemize}
 \item $n \equiv \pm 2 \pmod{5}$
 \item 底$a=2$のFermatテスト(\rAlgo{fermat_test})で素数と判定される。
 \item Fibonacci擬素数テストで素数と判定される。
\end{itemize}
となるような合成数$n$は見つかっていない。
これは、素数判定に関する\IND{Selfridge予想}{Selfridgeよそう}(Selfridge's conjecture)あるいは\IND{PSW予想}{PSWよそう}として知られ、真であることまたは反例を見つけた者に620ドルの賞金が送られることになっている\cite{A_Computational_Perspective}。

\IND{Baillie-PSWテスト}{Baillie-PSWてすと}(Baillie-PSW primality test)は、大雑把に言えばMiller-Rabinテストと強Lucas擬素数テストを組み合わせた素数判定法である。
Baillie-PSWテストという名称は、発明者であるRobert Baillie, Carl Pomerance, John Selfridge, Samuel Wagstaffからきている。
元々、3人が発明したPSWテストが存在し、Baillieが改良を加えたため、このような名称となっている。

Miller-RabinテストもLucas数列テストも、誤って素数と判定されてしまう合成数が存在するが、両方を突破する合成数は感覚的は少ないと推定できるだろう。
もちろん、無闇矢鱈にテストを繰り返せば合成数であるという疑いは徐々に晴れていくものの、それでは芸がない。
Miller-Rabinテストにおいて、$a=2$の場合に誤って素数と判定されてしまう合成数は、$2047, 3277, 4033,\ldots$等である\Notes{\url{https://oeis.org/A001262}}。
一方、Selfridgeの方法Aで決められたパラメータ$a,b$を設定された強Lucas擬素数テストにおいて、誤って素数と判定されてしまう合成数は、$5459, 5777, 10877,\ldots$等である\Notes{\url{https://oeis.org/A217255}}。
この2つを試して間違うときというのは、この2つの数列に共通の数が含まれていた場合のみである。
(現在検証されている範囲では)$2^{64}$までなら確実に素数か合成数かを判定できるし、これ以上についても、そのような合成数は未だ発見されていない。

\Algo{Baillie-PSWテスト}{baillie_psw_test}{c.f., \rAlgo{is_square_number}, \rAlgo{miller_rabin_test}, \rAlgo{selfridge_method_a}, \rAlgo{strong_lucas_sequence_test}}

