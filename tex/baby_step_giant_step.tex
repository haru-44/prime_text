Shanksの \Ind{baby-step giant-step} は一般には、離散対数問題を解くアルゴリズムとして知られている。
離散対数問題とは、群$G$の元$\alpha,\beta$を与えられたとき、$\alpha^x=\beta$を満たす$x$を求める問題である。
しかし、$\beta=1$とすれば、$\alpha$の位数を求める問題にも利用できるし、元の位数が分かるということは群の位数を求めることにも応用できると期待できることは想像に難くない\Notes{厳密な議論をするならば、$\alpha^x=1$を満たす$x$が分かったからと言って、$\alpha$の位数が求まったわけではないことに注意する。あくまで$\alpha$の位数の倍数が$x$なのである。}。

$h(D)$を知ろうとしたとき、

\begin{Prop}{}{}
任意の$D<0$において、
\begin{align*}
0 < h(D) < \sqrt{|D|}\ln{|D|}
\end{align*}
を満たす。
\end{Prop}

という事実が知られているので、元を虱潰しに枚挙していく方法だと、少なくとも元の個数分のステップは必要である。
そこまで理解したうえでbaby-step giant-step を眺めてみよう。

まずは、baby-step giant-step の本家本元、離散対数問題を解くことを考える。
ここでの群とは巡回群である。
今、位数$n$の巡回群$G$の元$\alpha,\beta$が与えられ、$\alpha^x=\beta$となる$x$を求めたい。
$m$を$m=\left\lceil \sqrt{n} \right\rceil$とし、$x=im+j$と置く。
これは、$x$を$m$進2桁の数と考えたようなものだ。
すると、
\begin{align*}
\alpha^x &= \beta\\
\alpha^{im+j} &= \beta\\
\alpha^j &= \beta(\alpha^{-m})^i
\end{align*}
と変形できる。
つまらない式変形に見えて、既に結論を得ている。
まず、$\alpha^j$を$0\le j<m$の範囲で動かした値をテーブルとして作成する。
この作業は$O(m)$だけの時間とメモリ空間を使う。
そして、$\beta(\alpha^{-m})^i$を$0\le i<m$の範囲で計算するたびに、テーブル上に同じ値がないかを探し、存在すればその$i,j$から$x$を得られるという仕組みだ。
テーブルに存在するか照会するコストを無視すれば、やはり$O(m)$のステップで済むので全体としての計算時間は$O(\sqrt{n})$と評価できる。
\Algo{baby-step giant-step}{baby_step_giant_step}{c.f., \rAlgo{inverse_mod}, \rAlgo{sqrt_int}}

baby-step giant-step の考え方を使えば、群の位数を探索することができる。

\begin{Prop}{}{}
群$G$の位数$\sharp G$は不明だが、$L\le\sharp G\le U$を満たすとする。
$1$を$G$の単位元、$a$を$G$の元とするとき、集合$H_a=\{i: L \le i \le U, a^i = 1 \}$の要素は$\sharp G$の候補である。
特に$\sharp H_a=1$のとき、その元が群の位数$\sharp G$である。
\end{Prop}

よって、ある程度$h(D)$の範囲を絞ったら、その範囲で baby-step giant-step を適用すればよい。
特別なことをしなくとも$h(D) < \sqrt{|D|}\ln{|D|}$より、$\sqrt{\sqrt{|D|}\ln{|D|}}$程度で$h(D)$を求められることが分かる。
しかも、$D$が決まっていれば$a,b$が決まった時点で$c$は確定するので、アルゴリズム中で$c$を記憶しておく必要はない。
さらに、$(a,b,c)$の逆元は
\begin{align*}
(a,b,c)^{-1} =
\begin{cases}
(a, -b, c), &\mbox{if } 0 < b < a \neq c\\
(a, b, c), &\mbox{otherwise}
\end{cases}
\end{align*}
であるから、$b$は絶対値のみを記憶すればよい。
これで、プレーンな baby-step giant-step よりも2倍早くなる。

今、$L\le\sharp G\le U$ であることが分かっているとして$\alpha^{L+x}=1$となる$x$を求めたい。
$x$の範囲は$0\le x \le U-L$である。
ここで、$\beta=\alpha^{-L}$と置くと、$\alpha^x=\beta$を解く問題に置き換えられる。
後は、baby-step giant-step を適用すればよく、しかも、$i$は$2\le i <m$で偶数のみを調べればよい\Notes{$\beta\alpha^{-2m}$がテーブル上にないということは、$m$から$3m$の間に位数が存在しないということであることに注意。}。

