\rTheo{fibonacci_prime}や\rTheo{lucas_sequence}は、素数に関する定理だが、素数判定にしか使えないわけではない。
$n$を合成数、$p$を$n$の素因数としよう。
$U_{p-\epsilon} \equiv 0 \pmod{p}$が成り立つということは、$\gcd(U_{p-\epsilon} \bmod{n}, n)$は$n$の非自明な因数であると期待できる、というのは、$p-1$法と同じ理屈である。
ここで、$\epsilon$と仮に置いたが、これは$+1$か$-1$のどちらかであり、$p$やLucas数列のパラメータ$a,b$によって変わるのであった。
$+1$のときは$p-1$法が因数分解できる数が素因数分解できるだけで面白みがない。
\IND{Williamsの$p+1$法}{Williamsのp+1ほう}(以下、$p+1$法)は、$n$の素因数$p$について$p+1$が$B$-smoothであるという仮定を置いた素因数分解アルゴリズムである\cite{Williams1982}。
無論、$p$がそもそも不明であることからも分かる通り、意図して$\epsilon=1$となるようにすることはできない。
Lucas数列のパラメータ$a$をランダムに選んで、見込みがなければ別の$a$で試す、ということも必要となる。

$p+1$法では、扱いやすいため主にLucas数列$\{V_n\}$を使う。
これまで主にLucas数列と言えば$\{U_n\}$しか扱ってこなかったが、$\{V_n\}$でも$\{U_n\}$とほぼ同じことが言える。
ただし、$V_0=2$のため周回して戻ってくる先も0ではなく2である。

これまで紹介したFibonacci数列とLucas数列を求めるアルゴリズム(\rAlgo{fibonacci_sequence}, \rAlgo{lucas_sequence})は、愚直に計算するだけで効率が悪い。
悪いだけでなく、ちょっと大きい数を与えると落ちてしまう。
しかも、$p-1$法の類似物を考えるということは、$V_j$から$V_{jk}$を計算できて欲しい。
そこで、\rAlgo{lucas_sequence_v}だ。

\Algo{Lucas数列$\{V_n\}$}{lucas_sequence_v}

このアルゴリズムは、$\{V_n\}$の
\begin{align*}
V_{j+k} = V_jV_k - b^jV_{k-1}
\end{align*}
という性質の特別の場合($b=1$で、$k=j$及び$j=Nj,k=N(j+1)$)を用いる。
そうすると、
\begin{align*}
V_{2j} &= V_j^2 - 2\\
V_{N(2j+1)} &= V_{Nj}V_{N(j+1)} - V_N
\end{align*}
を得るので、適切に適用していけば$V_j$から、$V_{jk}$を効率的に計算できるというわけだ。

最後に、$p+1$法のアルゴリズムを紹介して終わろう。
\rAlgo{p_minus_1_method}との類似にも注目して見て欲しい。

\Algo{Williamsの$p+1$法}{p_plus_1_method}{\rAlgo{lucas_sequence_v}}
