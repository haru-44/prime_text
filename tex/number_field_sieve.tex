近年、大きな合成数が素因数分解できたという成果の多くは、数体篩法に拠ることがほとんどだ。
RSA社が実施する\Ind{RSA Factoring Challenge}は、素因数分解屋に挑戦しがいのある合成数を示す一方、RSA暗号などの暗号がどれくらい安全であるかを実証するのにも一役買っている。
2021年現在、RSA Factoring Challengeが提示する合成数の中で素因数分解が判明している最大のものは、RSA-250と呼ばれる10進250桁の合成数で、
\begin{align*}
&2140324650240744961264423072839333563008614715144755017797754920881418023447\\
&1401366433455190958046796109928518724709145876873962619215573630474547705208\\
&0511905649310668769159001975940569345745223058932597669747168173806936489469\\
&9871578494975937497937
\end{align*}
である。
RSA-250は、2020年2月、Fabrice Boudotらによって
\begin{align*}
&641352894770715802787901901705773890848250147429434472081168596\\
&32024532344630238623598752668347708737661925585694639798853367\\
\times&\\
&333720275949781565562260106053551142279407603447675546667845209\\
&87023841729210037080257448673296881877565718986258036932062711
\end{align*}
という素因数分解が発見されている。
このときに使用されたのが、\Ind{CADO-NFS}\Notes{\url{http://cado-nfs.gforge.inria.fr/}}という数体篩法を行うソフトウェアで、RSA-240(2019年)、RSA-232(2020年)、RSA-230(2018年)の素因数分解もこのCADO-NFSを使用している\Notes{カッコ内は成功年。}。

