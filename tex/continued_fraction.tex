例えば、$\sqrt{3}$は$1.73205\cdots$と続く無理数だ。
つまり、整数/整数という形に表すことができない。
それでも近似を考えることはできて、安直には
\begin{align*}
\frac{17}{10} \; \frac{173}{100} \; \frac{1732}{1000}
\end{align*}
は有理数で$\sqrt{3}$を近似していると言える。
少々不満なのは、近似の精度を上げるにつれて分子分母の数が大きくなり過ぎではないかという点だ。
そうすると、もっと小さな数で良い近似は存在しないのだろうか? という疑問が自然と生まれてくる。

そこで、実数$\omega$に対して
\begin{align*}
\omega = a_0 + \cfrac{1}{a_1 + \cfrac{1}{a_2 + \cfrac{1}{a_3 + \cfrac{1}{\ddots}}}}
\end{align*}
というような\IND{連分数}{れんふんすう}(continued fraction)を考えよう。
分子がすべて1である場合には\IND{正則連分数}{せいそくれんふんすう}(regular continued fraction)と呼ぶこともあるが、今回扱う連分数はすべて正則連分数なので、ここで連分数といった場合には正則連分数を指す。
また、紙幅の関係で、
\begin{align*}
\omega=a_0 +\frac{1}{a_1 +}\, \frac{1}{a_2 +}\, \frac{1}{a_3 +}\cdots
\end{align*}
と書いたり、$[a_0;a_1,a_2,a_3,\ldots]$と書いたりする。
次は、単なる用語の定義だ。

\begin{Defi}{\IND{全商}{せんしよう}(complete quotient), \IND{部分商}{ふふんしよう}(partial quotient)}{complete_quotient}
実数$\omega$の連分数展開$[a_0;a_1,a_2,a_3,\ldots]$に対して、
\begin{align*}
x_i = [a_i; a_{i+1}, a_{i+2},\ldots]
\end{align*}
を$i$次全商、$a_i$を$i$次部分商と呼ぶ。
\end{Defi}

明らかに$x_0=\omega$である。
また、
\begin{align*}
x_i = a_i + \frac{1}{x_{i+1}}
\end{align*}
であるから$\left\lfloor x_i \right\rfloor$は$a_i$である。
つまり、$x_i$の整数部は$a_i$であることが分かる。
あるいは少し変形することで、
\begin{align*}
x_{i+1} = \frac{1}{x_i - a_i}
\end{align*}
も得られる。

さて、連分数の何が凄いのかというと、無理数に対して良く近似する有理数を作れることができるのだ。
例えば、$\sqrt{3}$は無理数なので当然、連分数も無限に続くのだが、適当なところで打ち切って整理してやれば、
\begin{align*}
\sqrt{3} &\approx 1 + \cfrac{1}{1 + \cfrac{1}{2 + \cfrac{1}{1 + \cfrac{1}{2 + \cfrac{1}{1}}}}}\\
&=\frac{26}{15} = 1.7333\ldots
\end{align*}
となって、3桁の数字を使わなくとも小数点以下第2位まで正しい有理数が得られた。

実数を連分数表示に変換する方法も、機械的にできて、$\sqrt{3}=1.73\ldots$の整数部は1で、少数部の逆数は$1/0.73\ldots=1.366\ldots$だから、
\begin{align*}
\sqrt{3} &= 1 + 0.73\ldots\\
&= 1 + \frac{1}{1.366\ldots}\\
&= 1 + \frac{1}{1 + 0.366\ldots}\\
&= 1 + \cfrac{1}{1 + \cfrac{1}{2.732\ldots}}
\end{align*}
というように、再帰的に計算できる。
ただし、$\sqrt{3}$のようなケースではもう少し工夫ができて、$w=\sqrt{3}-1$と置き、$w^2+2w=2$より$w=\frac{2}{2+w}$が得られる。
よって、
\begin{align*}
w + 1 &= 1 + \frac{2}{2+w} = 1 + \frac{1}{1 + w/2}\\
&= 1 + \cfrac{1}{1 + \cfrac{1}{1 + w/2}}\\
&= 1 + \cfrac{1}{1 + \cfrac{1}{1 + \cfrac{1}{2+w}}}
\end{align*}
というようにして求められる。

再三例示しているように、$\sqrt{3}$は連分数では$[1;1,2,1,2,\ldots]$と表されるが、この先どこまで計算していっても$1,2$が繰り返されるだけである。
これを$[1;\overline{1,2}]$と書いて、$1,2$が循環していることを表す。
少数にも循環小数が存在したが、連分数にも循環する場合があって、それは2次の無理数であることが必要にして十分である。

\begin{Prop}{}{}
2次無理数の連分数展開は循環する。
逆に、循環連分数は2次無理数を表す。
\end{Prop}

ここまでのアルゴリズムをきちんと書き下してみよう。

\begin{Prop}{}{continued_faraction_1}
連分数$[a_0;a_1,a_2,\ldots,a_n]$に対する有理数$p_n/q_n$は次の漸化式で求められる。
\begin{align*}
p_n = 
\begin{cases}
a_0, &\mbox{if } n = 0\\
p_0a_1 + 1, &\mbox{if } n = 1\\
p_{n-1}a_n + p_{n-2}, &\mbox{otherwise}
\end{cases}
\\
q_n = 
\begin{cases}
1, &\mbox{if } n = 0\\
q_0a_1, &\mbox{if } n = 1\\
q_{n-1}a_n + q_{n-2}, &\mbox{otherwise}
\end{cases}
\end{align*}
\end{Prop}

なお、\rProp{continued_faraction_1}で得られる$p_n/q_n$は既約分数である。
つまり、$\gcd(p_n,q_n)=1$なのである。

\begin{Prop}{}{continued_faraction_2}
平方数でない$N$に対して$\sqrt{N}$の連分数表示$[a_0;a_1,a_2,\ldots]$は次の漸化式で求められる。
\begin{align*}
a_n &= \left \lfloor x_n \right \rfloor\\
x_n &= \frac{\sqrt{N} + P_n}{Q_n}\\
P_n &= 
\begin{cases}
0, &\mbox{if } n = 0\\
a_{n-1} Q_{n-1} - P_{n-1}, &\mbox{otherwise}
\end{cases}
\\
Q_n &= 
\begin{cases}
1, &\mbox{if } n = 0\\
N - a_0^2, &\mbox{if } n = 1\\
Q_{n-2} + a_{n-1}(P_{n-1} - P_n), &\mbox{otherwise}
\end{cases}
\end{align*}
さらに、任意の$n=1,2,\ldots$において、
\begin{align*}
p_{n-1}^2 - N q_{n-1}^2 = (-1)^n Q_n\\
N = P_n^2 + Q_{n-1}Q_n
\end{align*}
が成り立つ。
\end{Prop}

\rProp{continued_faraction_2}では、全商$x_n$の求め方についても示しているが、具体的に計算するとなると必然的に実数(厳密には$\mathbb{Q}(\sqrt{N})$)を扱わなくてはならなくなる。
部分商$a_n$を求めたいだけなら、
\begin{align*}
a_n = \left \lfloor \frac{\left \lfloor \sqrt{N} \right \rfloor + P_n}{Q_n} \right \rfloor
\end{align*}
で事足りる。

\Algo{連分数を計算する}{continued_fraction}{c.f., \rAlgo{sqrt_int}}

\begin{Exam}{}{}
$N=41$とする。
このとき、$\omega=\sqrt{41}$の連分数展開を求めよう。
\begin{table}[htb]
\centering
\caption{$\sqrt{41}$の連分数展開}
\label{tb:sqrt41_continued_fraction}
\begin{tabular}{|r|r|r|c|r|r|r|}
\hline
$n$ & $P_n$ & $Q_n$ & $x_n$                   & $a_n$ & $p_n$ & $q_n$ \\ \hline
0   & 0     & 1     & $\frac{\sqrt{41}+0}{1}$ & 6     & 6     & 1     \\ \hline
1   & 6     & 5     & $\frac{\sqrt{41}+6}{5}$ & 2     & 13    & 2     \\ \hline
2   & 4     & 5     & $\frac{\sqrt{41}+4}{5}$ & 2     & 32    & 5     \\ \hline
3   & 6     & 1     & $\frac{\sqrt{41}+6}{1}$ & 12    & 397   & 62    \\ \hline
4   & 6     & 5     & $\frac{\sqrt{41}+6}{5}$ & 2     & 826   & 129   \\ \hline
5   & 4     & 5     & $\frac{\sqrt{41}+4}{5}$ & 2     & 2049  & 320   \\ \hline
6   & 6     & 1     & $\frac{\sqrt{41}+6}{1}$ & 12    & 25414 & 3969  \\ \hline
7   & 6     & 5     & $\frac{\sqrt{41}+6}{5}$ & 2     & 52877 & 8258  \\ \hline
\end{tabular}
\end{table}
表\ref{tb:sqrt41_continued_fraction}のように計算できるから、$\sqrt{41}=[6;\overline{2,2,12}]$である。
\end{Exam}

連分数表示を求める漸化式もさることながら、各変数間の関係式も重要である。
唐突に思われるかもしれないが、ここでPell方程式を考えよう。

\begin{Defi}{\IND{Pell方程式}{Pellほうていしき}, Pell's equation}{Pell's_equation}
平方数でない自然数$N$に対して、
\begin{align*}
x^2 - Ny^2 = 1
\end{align*}
の形の方程式をPell方程式と呼ぶ\Notes{イングランドの数学者 John Pell の名が冠されているが、これは名付け親の Leonhard Euler が Pell の業績と誤解したためだと言われている。}。
\end{Defi}

場合によっては、右辺が$-1$の場合もPell方程式と呼ぶことがあるが、ここでは$1$のときのみを考える。
さて、Pell方程式は\rProp{continued_faraction_2}で得た、
\begin{align*}
p_{n-1}^2 - N q_{n-1}^2 = (-1)^n Q_n
\end{align*}
という関係式に似ていないだろうか。
実は、Pell方程式は連分数展開のアルゴリズムによって解ける。
$n$が偶数で$Q_n$が1ならば、$p_{n-1},q_{n-1}$が方程式の解であることは明らかである。
この方法で必ず解が見つかること、しかもそれが最小解であることも知られている。

\Algo{Pell方程式を解く}{pell_equation}{c.f., \rAlgo{continued_fraction}}

さて、連分数を素因数分解に利用する1つの手は
\begin{align*}
p_{n-1}^2 - N q_{n-1}^2 = (-1)^n Q_n
\end{align*}
である。
両辺を$N$で剰余算すると
\begin{align*}
p_{n-1}^2 \equiv (-1)^n Q_n \pmod{N}
\end{align*}
となる。
このような関係式をいくつも集めて$Q_n$が平方数になる組み合わせを探せば、$x^2\equiv y^2\pmod{N}$を満たす$x,y$が得られる。
この手法は一般に\IND{連分数法}{れんふんすうほう}として知られている。
2次篩に似た手法だが、歴史的には連分数法が先輩だ。
個々の$Q_n$は(2次篩での場合に比べて)比較的小さいことはメリットだが、篩が使えないというデメリットもあって、総合的には2次篩に劣る。
