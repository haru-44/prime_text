もし、$n$の2種類の異なる平方数の和で表すことができれば、素因数分解ができる。
例えば、$n=1000009$は、$n=1000^2+3^2$と$n=972^2+235^2$という2種類の表し方がある。
これを見つけることができれば、$\gcd(n, 1000\cdot235 - 3\cdot972)$を計算することによって、非自明な約数$3413$を発見できる。
このように、$n$について2つの平方数の和を2種類得ることによって、素因数分解をするのが\IND{Euler法}{Eulerほう}(Euler's factorization method)である。

ここで、2つの疑問が湧き起こる。
\begin{enumerate}
\item すべての合成数$n$についてこのような表現があるのか。
\item 存在するとして簡単に見つけることができるのか。
\end{enumerate}
実の所、どちらも否定的だ。
つまり、2つの平方数の和が2種類存在しない合成数があるし、存在したとしても簡単に見つける方法は知られていない。
では、Euler法は使い物にならないのだろうか。
実は、Euler法のアイディアを2元2次形式まで広げることによって、$O(n^{1/3+\epsilon})$のアルゴリズムを得られる。
そのためには、まず2元2次形式が何者なのかを理解しなければならない。

\begin{Defi}{\IND{2元2次形式}{2けん2しけいしき}, binary quadratic form}{binary quadratic form}
\begin{align*}
f(x,y) = ax^2 + bxy + cy^2
\end{align*}
という形の2変数多項式を2元2次形式、あるいは単に2次形式(quadratic form)と呼ぶ。
\end{Defi}

2元2次形式という所から察せられる通り、$n$元2次形式も存在し、$n$を一般化した理論を2次形式と呼ぶこともあるが、ここでは2元2次形式のことを指して2次形式と呼ぶ。
また、$ax^2 + 2bxy + cy^2$を2次形式と呼ぶ流儀があるが、ここでは$ax^2 + bxy + cy^2$の形をそう呼ぶ。
さらに、簡単のため$f=(a,b,c)$と表すこともある。
何の変哲もない多項式に見えるが、数学の様々な分野において非常に重要な概念である。

今、我々は$x,y$に整数を与えたとき、$f(x,y)$がどの整数を取るのか知りたい。
例えば、$4x^2+3xy+2y^2$は、$x=y=1$のとき$9$になるが、$x,y$を整数の範囲でどんなに動かしても$1$にはならない。

\begin{Defi}{}{qf_representation}
整数$n$が2次形式$f(x,y)=ax^2+bxy+cy^2$で表現可能であるとは、整数$x,y\in\mathbb{Z}$が存在して、$n=f(x,y)$を満たすことである。
\end{Defi}

\begin{Defi}{}{qf_iff}
2次形式$f=(a,b,c),g=(a',b',c')$が同値(equivalent)であるとは、表現可能な整数の集合が一致することを言う。
\end{Defi}

$f=(a,b,c),g=(a',b',c')$が同値であるときとはどのような時か、ということを考えたい。
こういうことを考えるということは、異なる表し方でも同値となるような2次形式が存在するということでもある。
例えば、$(1,0,5)$と$(1,2,6)$は見た目こそ異なるが、表現可能な整数の集合は一致する(つまり、2つの2次形式は同値である)。
同値か同値でないかを一目で分かるようにすることはできないだろうか? そのためまずは、2次方程式にも判別式があったように、2次形式にも判別式を導入する。

\begin{Defi}{}{qf_det}
$D=b^2-4ac$を$f=(a,b,c)$の判別式と呼ぶ。
\end{Defi}

\begin{Prop}{}{qf_det}
2次形式$f=(a,b,c),g=(a',b',c')$が同値ならば、2つの判別式は一致する。
\end{Prop}

しかし、注意が必要なのは、逆は成り立たないことだ。
例えば、$(1,1,4)$と$(2,1,2)$は共に$D=-15$だが、前者は1を表現可能であるのに対して後者は表現できないので、これらは同値ではない。
このように、判別式は同値か同値でないかを見分けるには今一つ役に立たないように思える。
それでも判別式は、2次形式を語る上では欠かせない、重要な数値であるから最初に導入した。
特に、$D$が正か負かで話はまったく変わってくる。
ここでは$D$が負の場合について扱うので誤解がないようにしたい。
さらに$D$が負の場合、$(a,b,c)$と$(-a,-b,-c)$は符号を反転しただけに過ぎないので、$a>0$のみを考える。

さて、同値か同値でないかを一目で判別するアイディアとして、同値な2次形式は1つの形で表すことはできないだろうか。
例えば分数では、$2/4$も$10/20$も同じ数を表して、$1/2$とするのが良いように、2次形式でも標準的な形のようなものがあれば嬉しい。
それが簡約形式と呼ばれるものだ。
分数で言うところの既約分数にあたると言えるだろう。

\begin{Defi}{}{qf_reduction}
負の判別式で$a>0$である2次形式$(a,b,c)$が簡約形式であるとは、次を満たすことを言う。
\begin{align*}
-a < b \le a < c \mbox{ または } 0 \le b \le a = c
\end{align*}
\end{Defi}

では、どのように簡約形式を見つければ良いのか?

\Algo{2次形式の簡約}{quadratic_form_reduction}{}

簡約形式の嬉しさは、$a,b,c$の大きさが$D$によって制限されることである。
$(a,b,c)$が簡約形式であるとき、次を満たす。
\begin{align*}
|a| \le \sqrt{\frac{|D|}{3}}
\end{align*}
というのも、
\begin{align*}
|D| &= |b^2 - 4ac|\\
&\ge \big| |b|^2 - 4|a||c| \big|\\
&\ge 4|a||c| - |b|^2\\
&\ge 4|a|^2 - |b|^2\\
&\ge 4|a|^2 - |a|^2\\
&=3|a|^2
\end{align*}
だからである。

$D<0$かつ$D\equiv0,1\pmod{4}$なる$D$について、$\mathcal{C}(D)$は次のような2次形式$(a,b,c)$の集合と定義する。
\begin{itemize}
 \item 判別式は$D$
 \item $a>0$
 \item $(a,b,c)$は簡約形式
 \item $\gcd(a,b,c)=1$
\end{itemize}

念のため補足するが、簡約形式であることと$\gcd(a,b,c)=1$は同値ではない。
例えば、$(4,4,6)$は$D=-60$の簡約形式だが、$\gcd(a,b,c)=1$ではないので、$\mathcal{C}(-60)$には属さない。
また、$D\equiv0,1\pmod{4}$に限定しているのは、$D\equiv3,4\pmod{4}$になるような2次形式がそもそも存在しないためである\Notes{$D=b^2-4ac$であるが、$b^2\equiv0,1\pmod{4}$かつ$4ac\equiv0\pmod{4}$}。

ごちゃごちゃ言うよりも実例を見た方が早いだろう。

\begin{align*}
\mathcal{C}(-3) &= \{(1,1,1)\}\\
\mathcal{C}(-4) &= \{(1,0,1)\}\\
\mathcal{C}(-15) &= \{(1,1,4), (2,1,2)\}\\
\mathcal{C}(-23) &= \{(1,0,6), (2,1,3), (2,-1,3)\}\\
\end{align*}

$D<0$のとき$\mathcal{C}(D)$は有限集合である。
後で詳細に論じるが、$|D|$が小さければ$\mathcal{C}(D)$の元を枚挙できる。
簡約形式は$a$が$|D|$で抑えられていること、$a,b$が決まれば(既に$D$は決まっているので)$c$が決まることに注意すれば、ナイーブなアルゴリズムを次のように構成できるからだ。

\Algo{$\mathcal{C}(D)$の元を枚挙する}{quadratic_form_class_number_naive}{c.f., \rAlgo{sqrt_int}}

古くから考えられてきた問題として、$D$と$n$を与えられたとき$D$を判別式とする2次形式で$n$が表現可能か? というものがある。
これに対する答えとして次の定理が知られている。

\begin{Theo}{}{quadratic_form_and_quadratic_residue}
\begin{align*}
h^2 \equiv D \pmod{4n}
\end{align*}
を満たすような$h$が存在すれば、$D$を判別式とするある2次形式が存在し、$n$を表現可能である。
\end{Theo}

\begin{thProof}{quadratic_form_and_quadratic_residue}
$h^2 \equiv D \pmod{4n}$を満たすような$h$が存在するとする。
すると、$h^2 = D + 4nA$を満たす$A$が存在する。
$(A,h,n)$は判別式$D$の2次形式であり、$x=0,y=1$のとき$n$を表示する。
\end{thProof}

2次形式を通して、Euler法をもう一度眺めてみよう。
Euler法は、分解したい数$n$の2次形式$(1,0,1)$による異なる表示を求めている。
一般に、$i=1,2$について
\begin{align*}
n = x_i^2 + y_i^2\\
x_i \ge y_i \ge 0\\
x_1 > x_2
\end{align*}
であれば、$1<\gcd(x_1y_2 - y_1x_2, n) < n$となり、素因数分解が成功する。

既に述べたように、必ずしも合成数は2つの平方数の和として2通りに表されるわけではない。
むしろ、稀と言っても良い。
そこで、$n$の2次形式$(a,b,c)$による異なる表示を求めることによって、素因数分解することを考える。
$n$がある2次形式で表示できるかは、
\begin{align*}
h^2 \equiv D \pmod{4n}
\end{align*}
が可解であるかにかかっているのであった。
結論から述べると、$h_1 \not\equiv \pm h_2 \pmod{n}$である、$h^2\equiv D\pmod{4n}$の解が2つ見つかれば$n$の非自明な因数を見つけることができる。

