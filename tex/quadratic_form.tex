\subsubsection{2次形式の導入}
2次形式とは、次数が2の斉次多項式である(ここでは変数が2つの2元2次形式を単に2次形式と呼ぶ)。
つまり、
\begin{align*}
ax^2 + bxy + cy^2
\end{align*}
という形の多項式に関する理論であり、数学の広範な分野において重要な概念である。
もちろん、本稿で取り上げるからには素因数分解にも利用できる。
例えば$n=1000009$は、$n=1000^2+3^2$と$n=972^2+235^2$という2種類の表し方があるが、これを見つけることができれば、$\gcd(n, 1000\cdot235 - 3\cdot972)$を計算することによって、非自明な約数$3413$を発見できる。
つまり、$n$の2種類の異なる平方数の和で表すことができれば、素因数分解ができるのである。
このような素因数分解法は、\IND{Euler法}{Eulerほう}(Euler's factorization method)と呼ばれている。
$x^2+y^2$は、$a=c=1,b=0$であるような2次形式であるから、Euler法と2次形式は関係があるように思える。

改めて、2次形式を定義しよう。

\begin{Defi}{\IND{2元2次形式}{2けん2しけいしき}, binary quadratic form}{binary quadratic form}
\begin{align*}
f(x,y) = ax^2 + bxy + cy^2
\end{align*}
という形の2変数多項式を2元2次形式、あるいは単に2次形式(quadratic form)と呼ぶ。
\end{Defi}

\begin{Defi}{}{qf_det}
$D=b^2-4ac$を$f=(a,b,c)$の判別式と呼ぶ。
\end{Defi}

2元2次形式という所から察せられる通り、$n$元2次形式も存在し、$n$を一般化した理論を2次形式と呼ぶこともあるが、ここでは2元2次形式のことを指して2次形式と呼ぶ。
また、$ax^2 + 2bxy + cy^2$を2次形式と呼ぶ流儀があるが、ここでは$ax^2 + bxy + cy^2$の形をそう呼ぶ。
さらに、簡単のため$f=(a,b,c)$と表すこともある。
そして、2次形式を語る上では欠かせない、重要な数値である判別式を導入した。
特に、$D$が正か負かで話はまったく変わってくるのだが、本稿では$D<0$のみを扱う。
さらに$D$が負の場合、$(a,b,c)$と$(-a,-b,-c)$は符号を反転しただけに過ぎないので、$a>0$のみを考える。

Euler法は、$n$から$n=x^2+y^2$となる$x,y$を見つけることを試みたが、すべての合成数$n$について、そのような$x,y$は存在するのだろうか? この疑問を考察するために、我々は$x,y$に整数を与えたとき、$f(x,y)$がどの整数を取るのかを考えよう。
例えば、$4x^2+3xy+2y^2$は、$x=y=1$のとき$9$になるが、$x,y$を整数の範囲でどんなに動かしても$1$にはならない。

\begin{Defi}{}{qf_representation}
整数$n$が2次形式$f(x,y)=ax^2+bxy+cy^2$で表現可能であるとは、整数$x,y\in\mathbb{Z}$が存在して、$n=f(x,y)$を満たすことである。
\end{Defi}

一方で、見た目が異なる2次形式でも、表現可能な整数の集合は一致する場合がある。

\begin{Defi}{}{qf_iff}
2次形式$f=(a,b,c),g=(a',b',c')$が同値(equivalent)であるとは、表現可能な整数の集合が一致することを言う。
\end{Defi}

例えば、$(1,0,5)$と$(1,2,6)$は見た目こそ異なるが、表現可能な整数の集合は一致する(つまり、2つの2次形式は同値である)。
同値か同値でないかを一目で分かるようにすることはできないだろうか? 判別式は、部分的な結論を与える。

\begin{Prop}{}{qf_det}
2次形式$f=(a,b,c),g=(a',b',c')$が同値ならば、2つの判別式は一致する。
\end{Prop}

注意が必要なのは、逆は成り立たないことだ。
例えば、$(1,1,4)$と$(2,1,2)$は共に$D=-15$だが、前者は1を表現可能であるのに対して後者は表現できないので、これらは同値ではない。
このように、判別式は同値か同値でないかを見分けるには今一つ役に立たないように思える。
それならば、同値な2次形式は1つの形で表すことはできないだろうか。
例えば分数では、$2/4$も$10/20$も同じ数を表して、$1/2$とするのが良いように、2次形式でも標準的な形のようなものがあれば嬉しい。
それが簡約形式と呼ばれるもので、分数で言うところの既約分数にあたると言えるだろう。

\begin{Defi}{}{qf_reduction}
負の判別式で$a>0$である2次形式$(a,b,c)$が簡約形式であるとは、次を満たすことを言う。
\begin{align*}
-a < b \le a < c \mbox{ または } 0 \le b \le a = c
\end{align*}
\end{Defi}

簡約形式にするのも、次のようにしてできる。

\Algo{2次形式の簡約}{quadratic_form_reduction}{}

簡約形式の嬉しさは、$a,b$の大きさが$D$によって制限されることである。
つまり、$(a,b,c)$が簡約形式であるとき、
\begin{align*}
|a| \le \sqrt{\frac{|D|}{3}}
\end{align*}
を満たす。

次に、$D$毎に2次形式を分類することを試みる。
$D<0$かつ$D\equiv0,1\pmod{4}$なる$D$について、$\mathcal{C}(D)$は次のような2次形式$(a,b,c)$の集合と定義する。
\begin{itemize}
 \item 判別式は$D$
 \item $a>0$
 \item $(a,b,c)$は簡約形式
 \item $\gcd(a,b,c)=1$
\end{itemize}

念のため補足するが、簡約形式であることと$\gcd(a,b,c)=1$は同値ではない。
例えば、$(4,4,6)$は$D=-60$の簡約形式だが、$\gcd(a,b,c)=1$ではないので、$\mathcal{C}(-60)$には属さない。
また、$D\equiv0,1\pmod{4}$に限定しているのは、$D\equiv3,4\pmod{4}$になるような2次形式がそもそも存在しないためである\Notes{$D=b^2-4ac$であるが、$b^2\equiv0,1\pmod{4}$かつ$4ac\equiv0\pmod{4}$}。

ごちゃごちゃ言うよりも実例を見た方が早いだろう。

\begin{align*}
\mathcal{C}(-3) &= \{(1,1,1)\}\\
\mathcal{C}(-4) &= \{(1,0,1)\}\\
\mathcal{C}(-15) &= \{(1,1,4), (2,1,2)\}\\
\mathcal{C}(-23) &= \{(1,0,6), (2,1,3), (2,-1,3)\}\\
\end{align*}

$|a| \le \sqrt{\frac{|D|}{3}}$であることから、$D$が小さければ$\mathcal{C}(D)$の元を枚挙することは可能だ。
ナイーブなアルゴリズムを次のように構成できる。

\Algo{$\mathcal{C}(D)$の元を枚挙する}{quadratic_form_class_number_naive}{c.f., \rAlgo{sqrt_int}}

ここまでの準備によって、改めて2次形式で表現可能な$n$について考えたい。
$D$と$n$を与えられたとき$D$を判別式とする2次形式で$n$が表現可能か? という問題に対しては、次の定理が知られている。

\begin{Theo}{}{quadratic_form_and_quadratic_residue}
\begin{align*}
h^2 \equiv D \pmod{4n}
\end{align*}
を満たすような$h$が存在すれば、$D$を判別式とするある2次形式が存在し、$n$を表現可能である。
\end{Theo}

\begin{thProof}{quadratic_form_and_quadratic_residue}
$h^2 \equiv D \pmod{4n}$を満たすような$h$が存在するとする。
すると、$h^2 = D + 4nA$を満たす$A$が存在する。
$(A,h,n)$は判別式$D$の2次形式であり、$x=0,y=1$のとき$n$を表示する。
\end{thProof}

定理中では「ある2次形式」となっているが、特に$\sharp\mathcal{C}(D)=1$のときは1つの2次形式に限定されて、必要十分条件となる。
例えば、有名な定理としてFermatの二平方定理\Notes{有名な問題であるが、定まった名前がない。}もこの定理から証明できる。

\begin{Theo}{\IND{Fermatの二平方定理}{Fermatのにへいほうていり}, Fermat's theorem on sums of two squares}{Fermat's_theorem_on_sums_of_two_squares}
奇素数$p$が整数$x,y$によって
\begin{align*}
x^2 + y^2 = p
\end{align*}
と表されるとき、かつそのときのみ、$p\equiv1\pmod{4}$を満たす。
\end{Theo}

これは奇素数の場合の結果だが、$n$が合成数の場合でも、常に$n=x^2+y^2$となるような$x,y$が存在するとは限らない。
つまり、Fermat法はどんな合成数にも適用できる素因数分解法ではないのだ。
しかし、がっかりするのはまだ早い。
2次形式を通してEuler法をもう一度眺めてみると、Euler法は分解したい数$n$の2次形式$(1,0,1)$による異なる表示を求めている。
$(1,0,1)$に限定することに無理があったのであって、2次形式一般に同様の議論ができないだろうか。

(混乱しないように明言しておくが)$n,D$は固定して話しを進める。
我々は既に、$h^2\equiv D \pmod{4n}$なる$h$を見つければ、$n$を表示する2次形式$(a,b,c)$を得ることができる。
これは、イメージだけで言えば「$h$の世界」と「2次形式の世界」とに橋が架かっている状態と言える。
更に、「2次形式の世界」と
\begin{align*}
u^2 \equiv Dv^2 \pmod{4n}
\end{align*}
を満たす「解$(u,v)$の世界」に橋を架ければ、「$h$の世界」から「$(u,v)$の世界」までの経路が完成する。

\begin{Prop}{}{quadratic_form_1}
判別式$D$の2次形式$(a,b,c)$が、整数$n$を$ax^2+bxy+cy^2=n$と表示するとき、
\begin{align*}
u^2 \equiv Dv^2 \pmod{4n}
\end{align*}
を満たす解$u,v$が存在する。
\end{Prop}

\begin{prProof}{quadratic_form_1}
$ax^2+bxy+cy^2=n$の両辺に$4a$を掛けて整理すると、
\begin{align*}
(2ax + by)^2 -Dy^2 = 4an
\end{align*}
が得られる。
$u=2ax + by, v=y$と置くと、$u^2 - Dv^2\equiv0\pmod{4n}$の解$(u,v)$が得られる。
\end{prProof}

1つの結論は、「$h$の世界」で本質的に異なる2つの$h$さえ見つければ、それぞれの$h$から求められる$(u,v)$を使って$n$の非自明な約数を見つけられるということだ。

\begin{Prop}{}{sq_div}
$n$は$\sqrt{|D|}$以下の素数では割り切れないとする。
$h_i(i=1,2)$を$h_i^2\equiv D\pmod{4n}$の解で、$h_1\not\equiv\pm h_2\pmod{n}$を満たすとする。
上記の設定の下、$h_i$から得られる$u_i^2 \equiv Dv_i^2 \pmod{4n}$の解を$(u_i,v_i)$とすると、$\gcd(u_1v_2 - u_2v_1, n)$は$n$の非自明な約数である。
\end{Prop}

\subsubsection{$\mathcal{C}(D)$が成す群}
$\mathcal{C}(D)$は前節まで単なる集合として扱ってきたが、適切な演算を入れることによって群になる。
前節までを振り返ると、2つの2次形式$(a,b,c),(a',b',c')$が同値であるとは、$f(x,y)=ax^2+bxy+cy^2$と$g(x,y)=a'x^2+b'xy+c'y^2$の生成する数が一致することであった。
そして、同じ数を生成するのに複数の表し方があるのは都合が悪いので、同値な2次形式たちを「代表」する2次形式を定めたかった。
それが簡約形式で、2つの2次形式が同値であることと、それぞれの簡約形式が一致することは必要十分条件であった。

最初に、途中で挫折しないように素因数分解までのアウトラインを示しておこう。
判別式$D<0$の2次形式全体を考えると無限に存在するが、簡約形式で考えれば有限個に収まって、しかもそれは群を成す。
この群は素因数分解に応用できて、奇数$n$を素因数分解したい数だとして、$D=-n$と置く(ただし、$n\equiv3\pmod{4}$)。
どんな群にも言えることだが、群の位数が分かると、位数2の元が比較的簡単に見つけられる。
さて、この群の位数2の元というのは、$(a,0,c),(a,a,c),(a,b,a)$のいずれかの形をしている。
例えば、位数2の元として$(a,a,c)$を見つけたとすると、実は$a$は$n$の約数である。

整理し直すと、
\begin{enumerate}
 \item 群の位数を求める。
 \item 位数2の元を求める。
 \item $n$の因数分解を得る。
\end{enumerate}
というステップである。
このアプローチによる最大の障害は、群の位数を求めることにあるから、この群について詳細に考えたい。
というのが、本節以降での流れだ。

では改めて、これが群になるということを説明する。
群であるということは、適切な演算が定義されなければならない。
それが合成と呼ばれる操作で、2つの2次形式から新たな2次形式を作る。
このとき、単位元$1_D$となるのは
\begin{align*}
1_D = \begin{cases}
(1, 0, -D/4),& \mbox{if $D$は偶数}\\
(1, 1, (1-D)/4),& \mbox{if $D$は奇数}
\end{cases}
\end{align*}
である。

\begin{table}[htb]
\caption{まとめ}
\centering
\begin{tabular}{|c|c|c|c|c|c|}\hline
\multirow{2}{*}{$n \bmod{4}$} & \multirow{2}{*}{$D$}   & \multicolumn{2}{c|}{自明な分解を与える特異形式}                     & \multirow{2}{*}{特異形式} & \multirow{2}{*}{対応する分解} \\ \cline{4-4}
                              &                        &                                  & $1_D$                            &                           &                               \\ \hline\hline
\multirow{2}{*}{$3$}          & \multirow{2}{*}{$-n$}  &                                  & \multirow{2}{*}{$(1,1,(1+n)/4)$} & $(a,b,a)$                 & $2a + b \mid n$               \\ \cline{5-6}
                              &                        &                                  &                                  & $(a,a,c)$                 & $a \mid n$                    \\ \hline
\multirow{3}{*}{$1$}          & \multirow{3}{*}{$-4n$} & \multirow{3}{*}{$(2,2,(1+n)/2)$} & \multirow{3}{*}{$(1,0,n)$}       & $(a,b,a)$                 & $(2a + b)/2 \mid n$           \\ \cline{5-6}
                              &                        &                                  &                                  & $(a,a,c)$                 & $a/2 \mid n$                  \\ \cline{5-6}
                              &                        &                                  &                                  & $(a,0,c)$                 & $a \mid n$                    \\ \hline
\end{tabular}
\end{table}

\subsubsection{類数公式}
一般に、Dirichletの類数公式は、2次体の類数に関する公式として知られる。
今、2次形式の話しをしていたはずなのに、なぜ2次体のことを持ち出すのか。
それは、2次形式を発展させたものが2次体だからである。
ここでは2次体の理論を詳しく説明しないが、2次体の類数と2次形式の類数$\sharp\mathcal{C}(D)$は一致するのでその結果だけを流用する。
以降、$h(D)=\sharp\mathcal{C}(D)$と書くことにしよう。

\begin{Theo}{}{}
2次体$K=\mathbb{Q}(\sqrt{m})$、その判別式$D$、Kronecker指標$\chi_D$, $L$関数
\begin{align*}
L(s, \chi_D) = \sum_{n=0}^{\infty}\frac{\chi_D(n)}{n^s}
\end{align*}
と置く。
$K$の類数$h(D)$は
\begin{align*}
h(D) = \frac{1}{\kappa_0}L(1, \chi_D)
\end{align*}
と表せる。
ただし、$w$は$K$に含まれる1のべき根の個数、$\varepsilon_0$は$K$の基本単数で$\varepsilon_0>1$としたとき、
\begin{align*}
\kappa_0 = 
\begin{cases}
\frac{2\pi}{w\sqrt{|D|}}, & \mbox{if } D < 0\\
\frac{2 \log{\varepsilon_0}}{\sqrt{D}}, & \mbox{if } D > 0
\end{cases}
\end{align*}
とする。
\end{Theo}

何も説明していなので、何を言っているか分からないと思う。
それでも$h(D)$を求めたいという目的意識で眺めてみると、無限和が出てくるため機械での計算が困難であることが了解できるだろう。
そこで、Gauss和を用いて有限和の形に直す。
さらに、$D<0$のみを考えていたので、$D>0$の式は省略する。

\begin{Theo}{\IND{Dirichletの類数公式}{Dirichletのるいすうこうしき}, Dirichlet class number formula}{dirichlet_class_number_formula}
虚2次体の類数$h(D)$は、次の式で表される。
\begin{align*}
h(D) = \frac{w}{2}\cdot\frac{1}{D}\sum_{n=1}^{|D|}\chi_D(n)n
\end{align*}
\end{Theo}

$|D|$回の計算が必要であるから、$D$がそれなりに小さくないと計算できない。
それでも、無限が有限に変わったのは大きい。
これを具体的に計算するために、説明を加えていこう。

$w$は「$K$に含まれる1のべき根の個数」であったが、それほど難しい値ではない。
$D$の値によって、
\begin{align*}
w = 
\begin{cases}
4, & \mbox{if } D = -3\\
6, & \mbox{if } D = -4\\
2, & \mbox{otherwise}
\end{cases}
\end{align*}
と定まる数で、$D=-3,-4$が例外で基本的に$2$であることが分かる。

次にKronecker指標についてであるが、これまでLegendre記号、Jacobi記号を学んだ。
Legendre記号は奇素数について、Jacobi記号は奇数について定義されていたが、Kronecker記号は整数全体に対して定義される。

\begin{Defi}{\IND{Kronecker記号}{Kroneckerきこう}, Kronecker symbol}{kronecker_symbol}
整数$a,n$とし、$n\neq0$のとき$n=u2^sm$と分解できるとする($u=\pm1$, $m$は正の奇数)。
Kronecker記号$\left(\frac{a}{n}\right)$は次のように定義される。
当然ながら、$n$が奇数のときにはKronecker記号$\left(\frac{a}{n}\right)$はJacobi記号$\left(\frac{a}{n}\right)$に一致する。

\begin{align*}
\left(\frac{a}{n}\right) = \left(\frac{a}{u}\right)\cdot \left(\frac{a}{2}\right)^s \cdot\left(\frac{a}{m}\right)
\end{align*}
ここで、
\begin{align*}
\left(\frac{a}{u}\right) = 
\begin{cases}
1, & \mbox{if } u = 1 \mbox{ or } a \ge 0\\
-1, & \mbox{otherwise}
\end{cases}
\end{align*}
\begin{align*}
\left(\frac{a}{2}\right) = 
\begin{cases}
0, & \mbox{if } a \mbox{ は偶数}\\
1, & \mbox{if } a \equiv \pm 1 \pmod{8}\\
-1, & \mbox{if } a \equiv \pm 3 \pmod{8}
\end{cases}
\end{align*}
$\left(\frac{a}{m}\right)$は、Jacobi記号とする。
また、$n=0$のときは
\begin{align*}
\left(\frac{a}{0}\right) = 
\begin{cases}
1, & \mbox{if } a = \pm 1\\
0, & \mbox{otherwise}
\end{cases}
\end{align*}
とする。
\end{Defi}

\Algo{Kronecker記号}{kronecker_symbol}{c.f., \rAlgo{jacobi_symbol}, \rAlgo{split_int}}

Kronecker指標$\chi_D(n)$は、Kronecker記号を使って$\chi_D(n)=\left(\frac{D}{n}\right)$と書けるので、Dirichletの類数公式を計算する素地が整った。
定理をそのまま実装したバージョンと、若干の式変形をして簡単化したバージョンを載せる\Notes{変形の詳細については、\cite[p236]{iwanami_number_theory_2}を参照のこと。}。

\Algo{Dirichletの類数公式}{dirichlet_class_number_formula}{c.f., \rAlgo{kronecker_symbol}}

