もし、$n$の2種類の異なる平方数の和で表すことができれば、素因数分解ができる。
例えば、$n=1000009$は、$n=1000^2+3^2$と$n=972^2+235^2$という2種類の表し方がある。
これを見つけることができれば、$\gcd(n, 1000\cdot235 - 3\cdot972)$を計算することによって、非自明な約数$3413$を発見できる。
このように、$n$について2つの平方数の和を2種類得ることによって、素因数分解をするのが\IND{Euler法}{Eulerほう}(Euler's factorization method)である。

ここで、2つの疑問が湧き起こる。
\begin{enumerate}
\item すべての合成数$n$についてこのような表現があるのか。
\item 存在するとして簡単に見つけることができるのか。
\end{enumerate}
実の所、どちらも否定的だ。
つまり、2つの平方数の和が2種類存在しない合成数があるし、存在したとしても簡単に見つける方法は知られていない。
では、Euler法は使い物にならないのだろうか。
実は、Euler法のアイディアを2元2次形式まで広げることによって、$O(n^{1/3+\epsilon})$のアルゴリズムを得られる。
そのためには、まず2元2次形式が何者なのかを理解しなければならない。

\begin{Defi}{\IND{2元2次形式}{2けん2しけいしき}, binary quadratic form}{binary quadratic form}
\begin{align*}
f(x,y) = ax^2 + bxy + cy^2
\end{align*}
という形の2変数多項式を2元2次形式、あるいは単に2次形式(quadratic form)と呼ぶ。
\end{Defi}

$ax^2 + 2bxy + cy^2$を2次形式と呼ぶ流儀があるが、ここでは$ax^2 + bxy + cy^2$の形をそう呼ぶ。
また、簡単のため$f=(a,b,c)$と表すこともある。
何の変哲もない多項式に見えるが、数学の様々な分野において非常に重要な概念である。

今、我々は$x,y$に整数を与えたとき、$f(x,y)$がどの整数を取るのか知りたい。
例えば、$4x^2+3xy+2y^2$は、$x=y=1$のとき$9$になるが、$x,y$を整数の範囲でどんなに動かしても$1$にはならない。

\begin{Defi}{}{qf_representation}
整数$n$が2次形式$f(x,y)=ax^2+bxy+cy^2$で表現可能であるとは、整数$x,y\in\mathbb{Z}$が存在して、$n=f(x,y)$を満たすことである。
\end{Defi}

\begin{Defi}{}{qf_iff}
2次形式$f=(a,b,c),g=(a',b',c')$が同値(equivalent)であるとは、表現可能な整数の集合が一致することを言う。
\end{Defi}

つまり、$f=(a,b,c),g=(a',b',c')$が同値であるときとはどのような時か、ということを知ることが当面の目標となる。
そこで、2次方程式にも判別式があったように、2次形式にも判別式を導入する。

\begin{Defi}{}{qf_det}
$D=b^2-4ac$を$f=(a,b,c)$の判別式と呼ぶ。
\end{Defi}

判別式$D$が$0$のときは、ある1次形式の2乗となるため、1次形式の簡単な場合に帰着される。
判別式$D>0$が平方数のときは、数学的に面白い問題ではない。
などということが分かる以外にも、判別式には次の重要な性質がある。

\begin{Prop}{}{qf_det}
2次形式$f=(a,b,c),g=(a',b',c')$が同値ならば、2つの判別式は一致する。
\end{Prop}

しかし、注意が必要なのは、逆は成り立たないことだ。

次に、同値な2次形式は1つの形で表したい。
分数では、$2/4$も$10/20$も同じ数を表して、$1/2$とするのが良いように、2次形式でも標準的な形のようなものがあれば嬉しい。
それが簡約形式と呼ばれるものだ。
分数で言うところの既約分数にあたる。

\begin{Defi}{}{qf_reduction}
2次形式$(a,b,c)$が簡約形式であるとは、次を満たすことを言う。
\begin{align*}
|b| \le |a| \le |c|
\end{align*}
\end{Defi}

では、どのように簡約形式を見つければ良いのか? 

\Algo{2次形式の簡約}{quadratic_form_reduction}{}

簡約形式の嬉しさは、$a,b,c$の大きさが$D$によって制限されることである。
$(a,b,c)$が簡約形式であるとき、次を満たす。
\begin{align*}
|a| \le \sqrt{\frac{|D|}{3}}
\end{align*}
というのも、
\begin{align*}
|D| &= |b^2 - 4ac|\\
&\ge \big| |b|^2 - 4|a||c| \big|\\
&\ge 4|a||c| - |b|^2\\
&\ge 4|a|^2 - |b|^2\\
&\ge 4|a|^2 - |a|^2\\
&=3|a|^2
\end{align*}
だからである。

2次形式を通して、Euler法をもう一度眺めてみよう。
Euler法は、分解したい数$n$の2次形式$(1,0,1)$による異なる表示を求めている。
一般に、$i=1,2$について
\begin{align*}
n = x_i^2 + y_i^2\\
x_i \ge y_i \ge 0\\
x_1 > x_2
\end{align*}
であれば、$1<\gcd(x_1y_2 - y_1x_2, n) < n$となり、素因数分解が成功する。

既に述べたように、必ずしも合成数は2つの平方数の和として2通りに表されるわけではない。
むしろ、稀と言っても良い。
そこで、$n$の2次形式$(a,b,c)$による異なる表示を求めることによって、素因数分解することを考える。
実際には任意の2次形式を考えるわけではなく、ある程度制限されたものを考えるのだが、$(1,0,1)$で考えていたEuler法とはどのように違うのだろうか。

まずは、何やらよく分からない集合$\mathcal{S}(D,n)$を定義する。

\begin{Defi}{}{sq_S}
集合$\mathcal{S}(D,n)$を次のように定義する。
\begin{align*}
\mathcal{S}(D,n) = \{(u,v) \mid u^2 - Dv^2 \equiv 0\pmod{4n}\}
\end{align*}
\end{Defi}

ここで、2次形式$f=(a,b,c)$は正整数$n$を表現可能であるとする。
つまり、$ax^2+bxy+cy^2=n$を満たす$(x,y)$が存在するわけで、この両辺に$4a$を掛けて整理すると、
\begin{align*}
(2ax + by)^2 -Dy^2 = 4an
\end{align*}
が得られる。
$u=2ax + by, v=y$と置くと、$u^2 - Dv^2\equiv0\pmod{4n}$の解$(u,v)$が得られる。
つまり、適当な$a,b,c,x,y$は、$(u,v)\in\mathcal{S}(D,n)$と対応することが分かる。

\begin{Prop}{}{sq_S}
任意の$(u,v),(u',v')\in\mathcal{S}(D,n)$について、$uv'\equiv u'v\pmod{2n}$を満たす。
\end{Prop}

そこで、次のように$h,A$を設定する。
\begin{itemize}
\item $h$は、$h^2\equiv D\pmod{4n}$の解とする。
\item $A$は、$h^2=D+4An$を満たすとする。
\end{itemize}
これを用いて2次形式$f=(A,h,n)$を作る。
$x=0,y=1$のとき$n$を表示し、対応する$\mathcal{S}(D,n)$の元は$(h,1)$である。

\begin{Prop}{}{sq_2n}
2次形式$f=(A,h,n)$を簡約して$g=(a,b,c)$を得たとし、対応する$\mathcal{S}(D,n)$の元は$(u,v)$とする。
このとき、$u\equiv vh \pmod{2n}$かつ、$\gcd(v,n)=1$である。
\end{Prop}

\begin{Prop}{}{sq_div}
$h_1,h_2$を$h^2\equiv D\pmod{4n}$の解で、$h_1\not\equiv\pm h_2\pmod{n}$を満たすとする。
上記の設定の下、$\gcd(u_1v_2 - u_2v_1, n)$は$n$の非自明な約数である。
\end{Prop}

以上より、$4n$を法とする$D$の平方根$h_1,h_2$を見つけることができれば、非自明な約数を得られることが分かった。
$h_1,h_2$が存在することも、既に前節で見ている。
では、$h_1,h_2$を探索すれば良いのかというと、話はそう単純ではない。
McKeeは、逆に$(u,v)$を探すことで目的を達成しようとした。
簡約形式は、$|a|\le\sqrt{|D|/3}$を満たすから、すべての$a$で$u^2\equiv4an\pmod{|D|}$を満たす$u$を$u\le2\sqrt{an}$の範囲で探せばよい。

この計算量に関しては次のように見積もることができる。
\begin{itemize}
\item ある$a$に対して、$u$の候補は高々$|D|^{\epsilon}+2\sqrt{an}/|D|^{1-\epsilon}$
\item すべての$a$では、$O(|D|^{1/2+\epsilon} + \sqrt{n}/|D|^{1/4-\epsilon})$
\item $|D|$が$n^{2/3}$程度であれば、$O(n^{1/3+\epsilon})$
\end{itemize}

$D$の決め方について説明していないが、上手く設定できる。

