\subsubsection{Jacobi signature}
一般に、$n$の素因数$q$に対して、$x_i \not\equiv x_j \pmod{n}$かつ$x_i \equiv x_j \pmod{q}$なる$x_i,x_j$を見つけることができれば、$\gcd(x_i-x_j, n)$は非自明な因数となる。
では、次のような素因数分解アルゴリズムを考えてみよう。

\begin{enumerate}
 \item ランダムに$m$個の$x\in\mathbb{Z}_n^*$をとる。
 \item この中で、$x_i \not\equiv x_j \pmod{n}$かつ$x_i \equiv x_j \pmod{q}$なる$x_i,x_j$を見つける。
 \item $\gcd(x_i-x_j, n)$を非自明な因数として答える。
\end{enumerate}

誕生日のパラドックスより、$m=\sqrt{q}$程度であれば素因数分解が成功すると期待できる。
しかし、条件に適う$x_i,x_j$のペアを見つけることは一筋縄ではいかない。
$m$個の中から2個の組み合わせをすべて調べるというのは現実的ではないし、$x_i \equiv x_j \pmod{q}$であることは傍目からでは分からないからだ。
けれど、この考え方が全然ダメというわけではない。
もし、$x_i \equiv x_j \pmod{q}$かどうかを比べられるような量が存在すれば、我々はすべての組み合わせをチェックすることはなくなる。
そんな都合の良い量が、特に$n=p^2q$型の合成数の場合存在する。

\begin{Defi}{Jacobi signature\cite{PeraltaOkamoto1996}}{jacobi_signature}
奇素数$p,q$、合成数$n=p^2q$とする。
長さ$k$のJacobi signature$J^{(k)}(x)$はJacobi記号$\left(\frac{\cdot}{\cdot} \right)$を使って次のように定義される。
\begin{align*}
J^{(k)}(x) = \left( \left(\frac{x + 1}{n} \right), \ldots, \left(\frac{x + k}{n} \right) \right)
\end{align*}
\end{Defi}

もし、$x_i \equiv x_j \pmod{q}$ならば、$J^{(k)}(x_i)=J^{(k)}(x_j)$である。
Jacobi記号の性質より、
\begin{align*}
\left(\frac{x}{n} \right) = \left(\frac{x}{p^2q} \right) = \left(\frac{x}{p^2} \right)\left(\frac{x}{q} \right) = \left(\frac{x}{q} \right)
\end{align*}
となるから当然のことであるが、逆に言えば$p^2q$型のときのみ有効な手法であることも了解できると思う。
また、長さ$k$を十分大きくとれば$x_i \equiv x_j \pmod{q}$でないときに$J^{(k)}(x_i)=J^{(k)}(x_j)$となる確率は低くなる。

これで、当初の青写真を実現できる目途が立った。
実装は次のようになる。

\Algo{Jacobi signature}{JacobiSignature}{c.f., \rAlgo{jacobi_symbol}}

だが、$\mathbb{Z}_n^*$からランダムに$x$を選んだのでは、$O(\sqrt{q})$程度かかってしまい、それではPollardの$\rho$法に劣る。
Jacobi signatureの真価は、ECM(楕円曲線法)の後段に実行する「部品」とすることができる点にある。
ECMで考える前に、簡単な例で考えてみよう。
Pollardの$p-1$法(\rAlgo{p_minus_1_method})を実行すると、$w=a^{p_1^{l_1}\cdot p_2^{l_2}\cdot\cdots\cdot p_r^{l_r}}$が得られるわけだが、これを利用する。
例えば、
\begin{align*}
p&=82694431355027=2\cdot43\cdot961563155291+1
q&=71694431354881=2^{16}\cdot3^7\cdot5\cdot100043+1
\end{align*}
のとき、$n=p^2q$を因数分解したいとする。
$p-1$法で$B=5$まで計算した結果\Notes{もちろん、分解するまで分からないが、仮にそれ以上までべき乗を計算しても結果に影響しない}得られる$w$の位数は$a$に位数よりも格段に小さい。
ならば、$\mathbb{Z}_n^*$からランダムに選ぶより$\langle w\rangle$からランダムに選んだ方\Notes{理屈の上では$r$をランダムに選んで$w^r$を計算することによって実現できるが、現実的にはべき乗の計算コストは馬鹿にできない。1つの巧いやり方として、$w^1,w^2,w^3,\ldots,w^k$を計算してその中からランダムに選んだ2つについて$w^x\cdot w^y$を計算すれば乗法演算だけで済む。}が当たりやすそうだ。
具体的には、$\sqrt{100043}$程度で当たりが見つかると見積もれる。
一般には$R$が$B$-smoothで$q=R\cdot S+1$と分解できるとき、$B$まで$p-1$法を適用した後、$O(\sqrt{S})$個のJacobi signatureを集めれば分解できる\Notes{もちろん$p-1$が$B$-smoothなら$p-1$法の時点で因数が見つかるが、そうでないならJacobi signatureの段階では$p$の特徴は無視されることに注意する。}。
この結果は$O(\sqrt{p})$であった$\rho$法よりも、多くの場合で勝る。
$(q-1)/2$も素数であるような意地の悪い素数に出くわすと、その優位性は崩れてしまうがそれなら$p-1$法ではなく、ECMを使えばよい。
ECMならそういう都合の悪い位数なら、パラメータを変えることで回避できる。
以上が\cite{PeraltaOkamoto1996}のアイディアである。
この提案手法を便宜上、\IND{Peralta-岡本法}{Peraltaおかもとほう}と呼ぶが、人口に膾炙した名称ではない。

\Algo{Peralta-岡本法}{peralta_okamoto_method}{c.f., \rAlgo{EllipticCurveMontgomery}, \rAlgo{inverse_mod}, \rAlgo{JacobiSignature}}

雰囲気は分かったが、本当に動くのだろうか? それを理解するためには楕円曲線のおさらいをしよう。
$E(\mathbb{Z}_n)$では特に関係のなさそうに見える2点$P=(x_1,y_1),Q=(x_2,y_2)$がもし、$x_1\equiv x_2\pmod{p}, y_1+y_2\equiv0\pmod{p}$だとすると、$E(\mathbb{Z}_p)$の世界では$P+Q=O$が成り立つ。
一方、$E(\mathbb{Z}_n)$の世界で$P+Q$を計算しようとすると、逆元計算が出来ず、結果的に$n$の非自明な因数が見つかる。
少し思考を進めると、$[s]P=O$となる正整数$s$が存在して、$t\equiv u\pmod{s}$を満たすとき、$X([a]P)\equiv X([b]P)\pmod{p}$を満たすことが分かる。
ただし、$X(P)$は$P$の$x$座標を表すものとする。
$P$の$x$座標が$\pmod{p}$で合同かはJacobi signatureを使えば分かるから、Peralta-岡本法は期待する通りに動作する。

\subsubsection{Peralta法}
さて、ここまでの話しで疑問が生じなかっただろうか。
「Jacobi signature って本当に必要なのか?」と。
Jacobi signature を毎回保存する必要があるが、それは誕生日のパラドックスのためには必要だ。
一方で$\rho$法も誕生日のパラドックスの原理は使いつつ、省メモリである。
当然、Jacobi signature を$\rho$法のようなアルゴリズムに変更するという改良が思いつく。
これを実現したのが\IND{Peralta法}{Peraltaほう}\cite{10.1007/978-3-0348-8295-8_11}である\Notes{これもまた、一般的な呼称ではない。}。

\Algo{Peralta法}{peralta_method}{c.f., \rAlgo{EllipticCurveMontgomery}, \rAlgo{inverse_mod}, \rAlgo{jacobi_symbol}}

