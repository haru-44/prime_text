\subsubsection{Fermat法}
例えば、$n=9951$を素因数分解するとしよう。
$x^2-y^2 = (x-y)(x+y)$という公式と、直感によって、
\begin{align*}
9951 &= 10000 - 49\\
&= 100^2 - 7^2 = (100 - 7)(100 + 7)\\
&= 93 \times 107
\end{align*}
を見つけるだろう。

この原理を一般の合成数にも適用できないだろうか。
理屈を知れば当たり前のことかもしれないが、どんな奇数の合成数$n$でも$n=x^2-y^2$を満たす自然数$x,y$が存在する。
というのも、奇数の合成数は$n=rs$と分解できる\Notes{$r<s$とする。}が、「$r$と$s$の中間地点」を$x$、「$x$から$r$(あるいは$s$)までの距離」を$y$と置けば$r=x-y,s=x+y$と書ける。
$r$と$s$は奇数なのだから「$r$と$s$の距離」は偶数であり、$x,y$は整数であることが保証される。
数式を使って言い直せば、$r=2a+1,s=2b+1$と表せられるとき、$x=a+b+1,y=b-a$と書ける。

つまり、この$x,y$を見つけられることができれば、素因数分解できたと言うことになる。
$n=x^2-y^2$となる$x,y$を見つけることによって、$n$を素因数分解数しようとするアルゴリズムを一般に\IND{平方差法}{へいほうさほう}と呼ぶが、\IND{Fermat法}{Fermatほう}は、その中でも最も愚直な方法である。
つまり、$x$を動かして$y=\sqrt{x^2-n}$が整数になれば良い。

$x$を調べる範囲は、$x,y$がどのような数であったかを思い出せば簡単だ。
$y=0$のとき、これは2つの因数$r,s$の距離がゼロだというのだから、$n$は$x^2$と表される平方数である。
このとき、$x$は$\sqrt{n}$が整数かをチェックすればよい。
逆に2つの因数が最も離れているパターンは$n=3r$のときで、2点の距離は$r-3=n/3-3$、中点は$3+\frac{\frac{n}{3}-3}{2}=\frac{n+9}{6}$である。
よって、$x$は$\left \lceil\sqrt{n}\right \rceil\le x \le (n + 9) / 6$の範囲を調べればよい。

以上の考察より、Fermat法は素因数が$\sqrt{n}$の近くにある場合に素早く解くことができて、Fermat法の得手不得手と、試し割法のそれは真逆の関係である。
Fermat法が素早く解ける整数は、試し割法では苦戦し、試し割法が素早く解ける整数は、Fermat法では苦戦する、といった様子だということが分かる。

以上を踏まえて、Fermat法を実装しよう。
既に述べたように、$\left \lceil\sqrt{n}\right \rceil \le x \le (n + 9) / 6$の範囲の$x$をすべてチェックするのである。

\Algo{Fermat法}{fermat_factorization_method}{c.f., \rAlgo{sqrt_int}}

Fermat法の高速化については、\ref{sssec:quadratic_residue_square}で改めて考察する。

\subsubsection{Lehman法}
Fermat法の挙動を観察すると、ループ回数は、単純に$n$が大きくなれば多くなるという訳ではないことに気付く。
例えば、$4343=101\times43$は7回の反復を要するが、その3倍($3\times4343=13029$)は、1回目で非自明な約数の発見に至っている。
より一般には$n=pq$のとき、$|q-p|$ が小さい方が良いので、$|uq-vp|$が最小となるような$k=uv$を使って$kn$に対してFermat法を実施するというアイディアが得られる。
もちろん、$n$の因数分解を知らずして$k$を知ることはできないので、$k$は考え得るすべての値で試してみる。
この考えを実現したのが、\IND{Lehman法}{Lehmanほう}である。

Lehman法では、新たに$k$というパラメータを用意し、$n$ではなく$4kn$について$4kn=x^2-y^2$となる$x,y$を探す。
これだけでは$k$が増えた分、純粋に探索範囲が増えてしまうが、$k,x$の範囲は、
\begin{itemize}
\item $k \le \left \lceil n^{1/3}\right \rceil$
\item $2\sqrt{kn}\le x < 2\sqrt{kn}+n^{1/6}/(4\sqrt{k})$
\end{itemize}
で済むことが分かっているので、この範囲を調べ尽くせば良い。
ただし、$n>21$かつ$n^{1/3}$以下に非自明な約数が存在しないことを前提としている。

\Algo{Lehman法}{lehman_method}{c.f., \rAlgo{sqrt_int}}

あまり興味がないだろうから軽く触れるだけにするが、探索範囲がなぜこれで済むのかの証明は次の流れで行う。
\begin{enumerate}
 \item $n=pq$と分解できるとする。ただし、試し割を実行するため$n^{1/3}<p\le q$を仮定してよい。
 \item $k \le \left \lceil n^{1/3}\right \rceil$, $\mid uq - vp \mid < n^{1/3}$, $k = uv$を満たす$k, u, v$が存在する。
 \item $x = uq + vp$, $y = \mid uq - vp \mid$と置くと、$4kn=x^2-y^2$及び$2\sqrt{kn}\le x < 2\sqrt{kn}+n^{1/6}/(4\sqrt{k})$を得る。
\end{enumerate}

なお、2つ目のステップには、DirichletのDiophantus近似定理を使用する。
この定理は、ある実数が無理数であることを証明するときによくお目にかかるのだが、数論を勉強していたと思ったら、初歩的とは言え超越数論の教科書を引っ張ってこなければならなくなった。
どんな分野にでも言えることだが、局所の知識がその分野のみに留まるということは中々ない。

\begin{Theo}{\IND{DirichletのDiophantus近似定理}{DirichletのDiophantusきんしていり}, Dirichlet's approximation theorem}{dirichlets_approximation_theorem}
実数$\alpha$, 正整数$\mathcal{B}>1$に対し、
\begin{align*}
v &\le \mathcal{B}\\
\mid \frac{u}{v} - \alpha \mid &< \frac{1}{v\mathcal{B}}
\end{align*}
を満たす正整数$u,v$が存在する。
\end{Theo}

この定理に$\alpha=p/q,\mathcal{B}=n^{1/6}\sqrt{q/p}$と代入することによって、
\begin{align*}
\mid uq - vp \mid < n^{1/3}
\end{align*}
を得る。
