\subsubsection{\rProp{tonelli_shanks_algorithm}の証明}
\begin{prProof}{tonelli_shanks_algorithm}
\rProp{sqrt1}より、$(AD^m)^{2^{s-k}}\equiv1\pmod{p}$ならば、$(AD^m)^{2^{s-(k+1)}}\equiv\pm1\pmod{p}$となるのは明らか。
$(AD^m)^{2^{s-(k+1)}}\equiv-1\pmod{p}$のとき、
\begin{align*}
(AD^{m + 2^k})^{2^{s-(k+1)}} &\equiv (AD^m)^{2^{s-(k+1)}}\cdot D^{2^{k+(s-(k+1))}}\pmod{p}\\
&\equiv (-1)\cdot D^{2^{s-1}}\pmod{p}
\end{align*}
となるが、$D$の位数は$2^s$なので、$D^{2^{s-1}}\equiv-1\pmod{p}$となる。
よって、$(AD^{m'})^{2^{s-(k+1)}}\equiv1\pmod{p}$である。
\end{prProof}

\subsubsection{\rTheo{pepin}の証明}
\begin{thProof}{pepin}
必要条件は、Lucasの定理(\rTheo{Lucas_theorem})より明らか。

十分条件を証明する。
\rProp{Legendre_symbol_calc}より
\begin{align*}
\left(\frac{3}{F_k}\right) \equiv 3^{(F_k-1)/2} \pmod{F_k}
\end{align*}
であるから$\left(\frac{3}{F_k}\right)$が$-1$となればよい。
$F_k$が素数のとき、平方剰余の相互法則(\rTheo{legendre_quadratic_reciprocity})より
\begin{align*}
\left(\frac{3}{F_k}\right)\left(\frac{F_k}{3}\right) = (-1)^{\frac{3-1}{2}\frac{F_k-1}{2}}
\end{align*}
が成り立つので、
\begin{align*}
\left(\frac{3}{F_k}\right) = (-1)^{\frac{3-1}{2}\frac{F_k-1}{2}} \bigg/ \left(\frac{F_k}{3}\right)
\end{align*}
が$-1$となれば証明が完了する。

$(F_k-1)/2$は偶数だから$(-1)^{\frac{3-1}{2}\frac{F_k-1}{2}}=1$。

$2^k$が偶数であること、2の偶数乗は3を法として1と合同であることより、$F_k\equiv2\pmod{3}$を得る。
よって、
\begin{align*}
\left(\frac{F_k}{3}\right) = \left(\frac{2}{3}\right) = 2^{(3-1)/2} \equiv -1 \pmod{3}
\end{align*}
となる。

以上より、
\begin{align*}
\left(\frac{3}{F_k}\right) = -1
\end{align*}
となり、証明は完了する。
\end{thProof}

\subsubsection{\rTheo{fibonacci_prime}の証明}
\begin{Lemm}{}{binomial_coefficients_mod}
$p$が素数のとき、次が成り立つ。
\begin{align}
{p \choose k} &\equiv 0 \pmod{p} \quad &(0 < k < p)\\
\label{eq:binomial_coefficients_mod_p1}
{p+1 \choose k} &\equiv 0 \pmod{p} \quad &(1 < k < p)\\
\label{eq:binomial_coefficients_mod_m1}
{p-1 \choose k} &\equiv (-1)^k \pmod{p} \quad &(0 < k < p)
\end{align}
\end{Lemm}

\begin{lmProof}{binomial_coefficients_mod}
二項係数の公式より、$0<k<p$のとき
\begin{align*}
{p \choose k} = \frac{p!}{k!(p-k)!} = \frac{p(p-1)\cdots(p-k+1)}{k(k-1)\cdots 1}
\end{align*}
であるが、$p$が素数で$k<p$という条件より、分子の$p$は分母では割り切れない。
よって、${p \choose k}$は$p$の倍数である。
$p+1$の場合も同様に、分子の$p$を分母では割り切れないため${p+1 \choose k}$は$p$の倍数である。

$p-1$の場合、二項係数の公式より、$0<k<p$のとき
\begin{align*}
{p-1 \choose k} &= \frac{(p-1)(p-2)\cdots(p-k)}{k(k-1)\cdots 1}\\
 &\equiv \frac{(-1)(-2)\cdots(-k)}{k(k-1)\cdots 1} \pmod{p}\\
 &\equiv (-1)^k \pmod{p}
\end{align*}
となる。
\end{lmProof}

\begin{thProof}{fibonacci_prime}
Fibonacci数列の一般項は、
\begin{align*}
F_n = \frac{\alpha^n - \beta^n}{\sqrt{5}}
\end{align*}
とも書ける\Notes{$\alpha-\beta=\sqrt{5}$に注意}。
両辺に$2^n\sqrt{5}$を掛けて、$\alpha,\beta$を展開すると、
\begin{align*}
2^n\sqrt{5}F_n = (1+\sqrt{5})^n - (1-\sqrt{5})^n
\end{align*}
となる。
右辺の各項に二項定理を適用すると、右辺はそれぞれ
\begin{align*}
{n \choose 0}\sqrt{5}^0       &+& {n \choose 1}\sqrt{5}^1       &+ \cdots +& {n \choose n-1}\sqrt{5}^{n-1}           &+& {n \choose n}\sqrt{5}^n\\
{n \choose 0}(-1)^0\sqrt{5}^0 &+& {n \choose 1}(-1)^1\sqrt{5}^1 &+ \cdots +& {n \choose n-1}(-1)^{n-1}\sqrt{5}^{n-1} &+& {n \choose n}(-1)^n\sqrt{5}^n
\end{align*}
となって、上から下を引くと奇数乗の項のみが残る。
つまり、
\begin{align}
\label{catalan_equation_fibonacci_prime}
2^n F_n = 2 \left[ {n \choose 1} + {n \choose 3}5 + {n \choose 5}5^2 + \cdots \right]
\end{align}
を得る。

式(\ref{catalan_equation_fibonacci_prime})に$n=p-1$を代入すると、
\begin{align*}
2^{p-1}F_{p-1} = 2 \left[ {p-1 \choose 1} + {p-1 \choose 3}5 + {p-1 \choose 5}5^2 + \cdots \right]
\end{align*}
となるが、式(\ref{eq:binomial_coefficients_mod_m1})と、等比数列の和の公式より、
\begin{align*}
2^{p-1}F_{p-1} &\equiv -2 (1 + 5 + \cdots + 5^{(p-3)/2}) \pmod{p}\\
&\equiv -2 \frac{5^{(p-1)/2} - 1}{5-1}\pmod{p}
\end{align*}
を得る。
$5^{(p-1)/2}\pmod{p}$は、$p\equiv\pm1\pmod{5}$のとき、1となるから、このとき$F_{p-1}\equiv 0\pmod{p}$を得る。

同様に、式(\ref{catalan_equation_fibonacci_prime})に$n=p+1$を代入すると、
\begin{align*}
2^{p+1}F_{p+1} = 2 \left[ {p+1 \choose 1} + {p+1 \choose 3}5 + {p+1 \choose 5}5^2 + \cdots \right]
\end{align*}
となるが、式(\ref{eq:binomial_coefficients_mod_p1})より
\begin{align*}
2^{p+1}F_{p+1} \equiv 2 (1 + 5^{(p-1)/2}) \pmod{p}
\end{align*}
を得る。
$5^{(p-1)/2}\pmod{p}$は、$p\equiv\pm2\pmod{5}$のとき、$-1$となるから、このとき$F_{p+1}\equiv 0\pmod{p}$を得る。
\end{thProof}

\subsubsection{2つのGauss和の定義が一致することの証明}\label{sss:proof_gauss_sum_1}
Gauss和$\tau_p$の2つの定義、式\ref{eq:tau_p1}と式\ref{eq:tau_p2}が奇素数において同値であることを示す。

式\ref{eq:tau_p1}は
\begin{align*}
\tau_p = \sum_{u \in \mathbf{QR}_p} \zeta_p^u - \sum_{v \in \mathbf{QNR}_p} \zeta_p^v
\end{align*}
と書き直せることがLegendre記号の定義から分かる。
ところで、\rProp{cyc_zeta}の4を適切に並び替えることによって、
\begin{align*}
1 + \sum_{u \in \mathbf{QR}_p} \zeta_p^u + \sum_{v \in \mathbf{QNR}_p} \zeta_p^v = 0
\end{align*}
を得る。
このことから、
\begin{align*}
\tau_p = 1 + 2\sum_{u \in \mathbf{QR}_p} \zeta_p^u
\end{align*}
と書けることが分かった。
一方、式\ref{eq:tau_p2}も、$n$が1から$p-1$を走るとき$n^2$には$\mathbf{QR}_p$の元がちょうど2つずつ現れるから、
\begin{align*}
\tau_p = 1 + 2\sum_{u \in \mathbf{QR}_p} \zeta_p^u
\end{align*}
が得られ、式\ref{eq:tau_p1}と式\ref{eq:tau_p2}は奇素数の場合において一致する。

