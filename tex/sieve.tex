\subsubsection{Eratosthenesの篩}
素数を列挙するためのアルゴリズムとして、\IND{Eratosthenesの篩}{Eratosthenesのふるい}(sieve of Eratosthenes)は有名だ。
Eratosthenesは紀元前のギリシャ人学者であり、他に地球の大きさを初めて測定した業績が知られている。

長さ$n$のリストを用意して、
\begin{itemize}
 \item 2の倍数は素数を素数でないとして消し込む
 \item 3の倍数は素数を素数でないとして消し込む
 \item 5の倍数は素数を素数でないとして消し込む……
\end{itemize}
とすることで、素数のみが残る。
実装としては、長さ$n+1$の配列を用意\Notes{オリジンの差異を吸収するための$+1$だ。}し、配列の$k$番目が$k$が素数かを管理する。
2以上をすべて素数として、アルゴリズムを開始し、$\sqrt{n}$まで動けば、$n$未満の素数が明らかになる。
配列の書き換え回数は2のときは$n/2$回、3のときは$n/3$回というようになるから、すべて足すと、
\begin{align*}
\frac{n}{2} + \frac{n}{3} + \frac{n}{5} + \cdots + \frac{n}{\sqrt{n}} = n\left(\frac{1}{2}+\frac{1}{3}+\frac{1}{5}+\cdots+\frac{1}{\sqrt{n}}\right)
\end{align*}
となり、$\sqrt{n}$までの素数の逆数和が現れる。
素数の逆数和
\begin{align*}
\sum_p \frac{1}{p}
\end{align*}
は発散することは非常に有名で、素数が無限に存在することの別証明にもなる\Notes{もしも有限なら発散しない。逆に、収束するからといって有限であることは言えない。双子素数の逆数の和は\IND{Brun定数}{Brunていすう}(Brun's constant)に収束するが、双子素数が有限個であるという証明にはならない。}。
発散するにはするのだが、その速度はとても遅い。
具体的には
\begin{align*}
\sum_{p\le n}\frac{1}{p} = O(\log{\log{n}})
\end{align*}
くらいだ。
以上より、Eratosthenesの篩の計算量は$O(n\log{\log{n}})$であることが分かる。
1つに対する素数判定を行うには大仰だが、複数回の素数判定を行う場合には効果を発揮する。
1度リストを作ってしまえば、1回の素数判定は$O(1)$で可能だからだ。

\Algo{Eratosthenesの篩}{sieve_of_eratosthenes}{c.f., \rAlgo{sqrt_int}}

\subsubsection{Sundaramの篩}
素数を列挙するためのアルゴリズムとして、Eratosthenesの篩が紹介されることは多いが、それ以外のアルゴリズムが紹介されることは少ない。
Eratosthenesの篩でも十分実用的であり、マニアックな改良まで紹介する必要がないからであろう。
そういう所に突っ込んでいくべきであろう。

\IND{Sundaramの篩}{Sundaramのふるい}(sieve of Sundaram)の基本的なアイディアは、奇数の積を篩う。
最初から偶数は無視することにすれば、リストのサイズは半分に減って、インデックスは$2x+1$の読み替えを必要とする。
任意の奇数の合成数は、$(2i+1)(2j+1)=2(i+j+2ij)+1$の形で表されることに注意しよう。
$i,j$が$(2i+1)(2j+1)\le n$を満たす範囲を動けば、$n$以下のすべての合成数をチェックすることができる。
一般性を失うことなく$i\le j$を仮定して良いし、$(2i+1)$は$\sqrt{n}$まで動けば十分であることはすぐ分かる。
実装は、オリジンの問題と変数の使い方の問題によって複雑に見えるが、実態は上記を忠実にコードに落とし込んでいるだけである。

\Algo{Sundaramの篩}{sieve_of_sundaram}{c.f., \rAlgo{sqrt_int}}

\subsubsection{Atkinの篩}
\IND{Atkinの篩}{Atkinのふるい}(sieve of Atkin)\cite{atkin_sieve}は、次の定理を使う。

\begin{Theo}{}{atkin}
$n$を無平方数とする。
\begin{enumerate}
 \item $n \equiv 1\pmod{4}$ならば、$n$が素数である必要十分条件は、$\sharp\{(x,y): x>0,y>0,4x^2+y^2=n\}$が奇数になることである。
 \item $n \equiv 1\pmod{6}$ならば、$n$が素数である必要十分条件は、$\sharp\{(x,y): x>0,y>0,3x^2+y^2=n\}$が奇数になることである。
 \item $n \equiv 11\pmod{12}$ならば、$n$が素数である必要十分条件は、$\sharp\{(x,y): x>y>0,3x^2-y^2=n\}$が奇数になることである。
\end{enumerate}
\end{Theo}

まず、$n$が無平方数であるという前提を無視して、この定理を使う。
$x,y$が自然数を走れば、$n$になるような個数が判明し、「素数っぽい」ものがチェックできる。
このときの「素数っぽい」とは、無平方数なら確実に素数だが、平方因子を持つ数が誤って素数と扱われているかもしれない。
そこで、最後の仕上げとして平方因子を持つ数を篩う。
具体的には、リストを小さい方から見て、$p$が「素数」ならば真に素数で問題なく、そこから$k$を動かして、$kp^2$を「合成数」とする。
最初に「素数っぽい」とチェックした合成数は、より小さい真の素数によって既に篩われているので、誤って「素数」と判定されることはない。

これでアルゴリズムの方針が分かった。
$n>3$が素数ならば、$n \% 12$ は1,5,7,11のいずれかであるので、
\begin{itemize}
 \item $n \% 12 = 1,5$ なら\rTheo{atkin}の1を使う。
 \item $n \% 12 = 7$ なら\rTheo{atkin}の2を使う。
 \item $n \% 12 =11$ なら\rTheo{atkin}の3を使う。
\end{itemize}
というように、$n$の種類によって使用する式の役割分担ができる。

これを実装するだけでもAtkinの篩を名乗って差し支えないのだが、$x,y$について範囲を限定することによって高速化を試みよう。

\begin{Prop}{}{atkin_cond}\;
\begin{enumerate}
 \item $4x^2+y^2 \equiv 1,5 \pmod{12}$を満たすとき、$y$は奇数である。
 \item $3x^2+y^2 \equiv 7 \pmod{12}$を満たすとき、$x$は奇数で$y$は偶数である。
 \item $3x^2-y^2 \equiv 11 \pmod{12}$を満たすとき、$x$と$y$の偶奇は異なる。
\end{enumerate}
\end{Prop}

さらに、剰余が1か5を判定するのもエレガントではないので、次の命題より9でないことを判定することに代替しよう。

\begin{Prop}{}{atkin_cond1}
$x$は正整数、$y$は正の奇数とする。
$4x^2+y^2 \bmod{12}$は1,5,9のいずれかをとる。
\end{Prop}

実装を考えたとき、加算よりも乗算の方がコストがかかるので$x^2,y^2$というような計算も極力少なくしたい。
そこで、加算で2乗を計算したいのだが、奇数の2乗、$1^2,3^2,5^2,\ldots$は次の漸化式で計算できる。
\begin{align*}
a_{k} = (2k+1)^2 =
\begin{cases}
a_{k-1} + 8k, &\mbox{if } k > 0\\
1, &\mbox{if } k = 0
\end{cases}
\end{align*}
同様に偶数の2乗、$2^2,4^2,6^2,\ldots$は次の漸化式で計算できる。
\begin{align*}
a_k = (2(k+1))^2 =
\begin{cases}
a_{k-1} + (12 + 8k), &\mbox{if } k > 0\\
4, &\mbox{if } k = 0
\end{cases}
\end{align*}

\Algo{Atkinの篩}{sieve_of_atkin}{}

Atkinの篩の計算量は$O(n/\log{\log{n}})$でEratosthenesの篩よりも優れる。
それでもその簡明さ故か、Eratosthenesの篩の応用例は多く素数列挙といえばEratosthenesの篩なのである。
Eratosthenesの篩の考え方を応用すれば、篩う対象は素数でなくともよい。
例えば、与えられた数の素因数分解も可能だ。
リストに記録するのは、「割られた最小の数」である。
この表は完全な素因数分解を提供する。
例えば、表の105番目には3が記録されているので、次に$105/3=35$番目を参照すると5が記録されている。
このように辿っていくと、105の素因数分解は$3\times5\times7$であることが分かる。

\Algo{素因数を篩う}{factor_sieve}{c.f., \rAlgo{sqrt_int}}

もう少し工夫すると、B-smooshな数を篩うことも可能だ。
リストには、書き込むのは、$p^a$の倍数のとき$p$を掛ける。
$B$-smoothな数ならば、インデックスと値が同じになっているはずだ。

\Algo{smoothな数を篩う}{b_smooth_sieve}{}

smoothな数を篩うことは結構重要で、後で紹介する2次篩法などがどうして「篩」という名を冠しているかというと、$B$-smoothな数を篩うからだ。
そして、実際上にしろ理論上にしろ最も時間がかかるのが篩うステップである。
よって、篩の高速化はそのまま素因数分解アルゴリズムの高速化にもつながる。
対数表は画期的な発明だったとされるが、現代の感覚からすると中々理解しづらい。
コンピュータが計算することが当たり前になって、手計算で近似値を求める必要性が乏しいからだ。
対数表の本質的な考えは、$x$と$a^x$とを対応させることにある。
「掛け算が足し算になる」というのは、単純な計算コストという面でも有利だ。
つまり、\rAlgo{b_smooth_sieve}で配列に保持するのは$\log$を取った値にすれば、毎回掛け算をする必要がなく、足し算で済む。
こうすると、実数で扱うため、計算機上では誤差が避けられないが、「$B$-smoothかどうかを(ほぼ正しく)判別したい」という目的において大きな問題ではない。

さて、これらEratosthenesの篩の類似物を扱う一般的な枠組みは考えられないだろうか。
実際、数え上げ問題など威力を発揮する定式化は存在するのだが、本稿の目的を外れすぎるのも良くない。
程よい一般化として次の問題を考えよう。

正整数から整数への関数$g$が与えられたとき、$f$を
\begin{align*}
f(n) = \sum_{d \mid n}g(d)
\end{align*}
と定義する。
$f(1),f(2),\ldots,f(n)$をすべて求めよ、という問題は、$g(1),g(2),\ldots,g(n)$の計算コストを無視すれば、定義通り実装したとしても$O(n\log{n})$で済む。
しかし、Eratosthenesの篩と同様に考えれば、$O(\log{\log{n}})$で済むので、より早い。

\Algo{}{divisor_transform}{c.f., \rAlgo{sieve_of_eratosthenes}}

このアルゴリズムはどのようなときに使えばよいのだろうか。
あからさまにそういう形をしている、約数関数から見てみよう。

\begin{Defi}{\IND{約数関数}{やくすうかんすう}, divisor function}{divisor_function}
自然数$n$に対して約数関数$\sigma_x(n)$を次のように定義する。
\begin{align*}
\sigma_x(n) = \sum_{d \mid n} d^x
\end{align*}
特に$\sigma_0(n)$は$n$の約数の個数を、$\sigma_1(n)$は$n$の約数の総和を、それぞれ意味する。
\end{Defi}

よって、$f(n)=1$とすれば、$g(n)=\sigma_0(n)$となるし、$f(n)=n$とすれば$g(n)=\sigma_1(n)$となる。

ところで、アルゴリズムを見れば分かる通り、別に総和である必要はない。
結合法則と交換法則さえ備わっていればいいので、$\max$や$\min$などで考えてもアルゴリズムは適用できる。

例えば、
\begin{align*}
g(n) &=
\begin{cases}
n, &\mbox{if } n \mbox{ is prime}\\
1, &\mbox{oterwise}
\end{cases}
\\
f(n) &= \max_{d \mid n}g(d)
\end{align*}
と定義すれば、$f(n)$は$n$を割り切る最大の素数が得られるし、
\begin{align*}
g(n) &=
\begin{cases}
n, &\mbox{if } n \mbox{ is prime}\\
\infty, &\mbox{oterwise}
\end{cases}
\\
f(n) &= \min_{d \mid n}g(d)
\end{align*}
なら、$f(n)$は$n$を割り切る最小の素数が得られる。
これらも$O(\log{\log{n}})$で列挙できる。
これら自体は既に具体的な示しているので驚くに当たらないが、包括的なアルゴリズムで表現できる点が目を見張る。

