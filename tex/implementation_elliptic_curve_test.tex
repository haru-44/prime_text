\subsubsection{楕円曲線の実装}
数学上の様態は理解が進んだかと思うが、これをプログラムに落とし込もうと思うと無限遠点$O$をどう扱うかなど、考える要素はいくつもある。
ここで紹介する実装方法は4種類である。
\begin{description}
 \item [アフィン座標]\mbox{}\\
       $(x,y,z)$の3つ組で管理し、楕円曲線上の点$(x,y)$は$(x,y,1)$、無限遠点は$(0,\pm1,0)$とする。
 \item [射影座標]\mbox{}\\
       $(X,Y,Z)$の3つ組で管理し、無限遠点は$(0,1,0)$とする。
       無限遠点でない射影座標$(X,Y,Z)$は、アフィン座標$(X/Z,Y/Z,1)$に対応する。
 \item [修正射影座標]\mbox{}\\
       $(X,Y,Z)$の3つ組で管理し、無限遠点は$(0,1,0)$とする。
       無限遠点でない修正射影座標$(X,Y,Z)$は、アフィン座標$(X/Z^2,Y/Z^3,1)$に対応する。
 \item [Montgomery座標]\mbox{}\\
       $(X,Z)$の2つ組で管理し、無限遠点は$(0,0)$とする。
       無限遠点でないMontogomery座標$(X,Z)$は、アフィン座標$(X/Z,?,1)$に対応する。
\end{description}
アフィン座標は、上記数式の素朴な実装だ。
それに対して射影座標、修正射影座標は、逆元計算をなるべくしないようにしたいという意図がある。
Montgomery座標は、そもそも$y$座標を管理しない。
$y$座標に興味がない場合に使えて、素因数分解の高速化の際に触れる。

まずは、アフィン座標での実装例を見てみよう。

\Algo{アフィン座標による楕円曲線の実装}{EllipticCurveAffine}{c.f., \rAlgo{inverse_mod}, \rAlgo{n_times}}

実装でネタバレをしているのだが、この楕円曲線に与える$p$は\textbf{素数でなくともよい}。
もちろん、$p$が素数でなければ群にはならないのだが、大抵の点で計算は問題なく行える。
計算が失敗し、群でないことが露わになる瞬間、うれしいことに一緒に素因数まで判明する。
素因数分解をしたい我々にとって願ってもない結果だ。
というわけで、これを素因数分解に使わない手はない。

\Algo{楕円曲線法}{elliptic_curve_method}{c.f., \rAlgo{EllipticCurveAffine}}

楕円曲線法は、$p-1$法(\rAlgo{p_minus_1_method})に類似していることに気付いただろうか? $p-1$法は$p-1$が$B$-smoothであるとき、素因数分解に成功したが、楕円曲線法も同じようなことが言える。
$p-1$法がなぜ$p-1$であったかと言えば、それが群の位数だったからである。
同様に、楕円曲線法も群の位数が$B$-smoothであれば素因数分解できる。
既に見たように、楕円曲線の群の位数はパラメータによって変化する。
つまり、楕円曲線のパラメータを動かすことによって、\kenten{たまたま}smoothな位数に巡り合えることができればよい。
また、楕円曲線を決めてから点を選ぼうとすると考えることが多くなるが、点を決めてからそれに合うように楕円曲線を選んでもよいので、そういう疑問も解決する。

さて、単純な改良は$p-1$法と同じように第2段階を適用することだ。
つまり、第1段階で$B_1$以下の素数$p$で、
\begin{align*}
Q = \left[\prod_{p} p^a\right] P
\end{align*}
と計算した$Q$を使って、$[p_1]Q, [p_2]Q, \ldots$を計算していく。
ここで、$p_1,p_2,\ldots$は$B_1$以上$B_2$以下の素数だ。
しかも素数の差はそんなにパターンがないので、$\Delta_2=[2]Q, \Delta_4=[4]Q,\ldots$を\ruby{予}{あらかじ}め計算しておき\Notes{2と3を除き、隣り合う素数の差は偶数になるので$\Delta_1$や$\Delta_3$は計算しなくて良い。}、
\begin{align*}
[p_i]Q = [p_{i-1}]Q + \Delta_{p_i-p_{i-1}}
\end{align*}
とすれば、(楕円曲線上での)加法1回で済む。
これは、$k$倍算するよりもはるかに高速だが、より高速化する方法を後に述べる。

\subsubsection{Montgomery座標での実装}
高速化には、アフィン座標ではなく、射影座標や修正射影座標で点を表すというのも1つの手だ。
しかし、更に突き詰めると、もはや$y$座標を考えなくても良いという結論に至る。
それがMontgomery座標である。
$y$座標を考えなくとも大丈夫なのかというと、実際、任意の加法ができないのだが、楕円曲線法で求められているのは$k$倍算であるので十分である。
しかも、楕円曲線のパラメータは$A=1,B=0$とした方が計算が簡略化される。
実際の計算は、$P_1=(X_1,Z_1), P_2=(X_2,Z_2), P_{\pm}=P_1{\pm}P_2=(X_{\pm},Z_{\pm}),[2]P_1=(X_d,Z_d)$とすると、
\begin{align*}
a_{24} &= \frac{C + 2}{4}\\
\alpha_{\pm} &= X_1 \pm Z_1\\
\beta_{\pm} &= X_2 \pm Z_2\\
X_+ &= Z_-(\alpha_-\beta_+ + \alpha_+\beta_-)^2\\
Z_+ &= X_-(\alpha_-\beta_+ - \alpha_+\beta_-)^2\\
X_d &= \alpha_+^2\alpha_-^2\\
Z_d &= (\alpha_+^2 - \alpha_-^2)(\alpha_-^2 + a_{24}(\alpha_+^2 - \alpha_-^2))
\end{align*}
で求められる\Notes{補助的に変数$a_{24},\alpha_{\pm},\beta_{\pm}$を導入した。特に$a_{24}$は$C$が決まった時点で以降は変わらないので、加算や2倍算時には定数として扱える。}。
見て分かる通り、$P_1+P_2$を計算するためには$P_1-P_2$を知っていなければならないので注意が必要だ。

次に、Montgomery座標で計算するとなったら、楕円曲線は$y^2=x^3+Cx^2+x$の中から選ぶことになるが、$C$をランダムに選ぶより良い方法について考えよう。
次の定理が知られている。

\begin{Theo}{}{}
楕円曲線$E:y^2=x^3+C_{\sigma}x^2+x$のパラメータ$C_{\sigma}$を$\sigma\neq0,1,5$に対して、
\begin{align*}
u &= \sigma^2 - 5\\
v &= 4\sigma\\
C_{\sigma} &= \frac{(v - u)^3(3u+v)}{4u^3v} - 2
\end{align*}
とすると、群$E(\mathbb{F}_p)$の位数は12で割り切れる。
さらに、Montgomery座標で$(u^3,v^3)$と表される点を楕円曲線法の開始点としてよい。
\end{Theo}

よって、$\sigma$を選べば、良い曲線のついでに最初の点まで得られる。

\Algo{最適化された楕円曲線法}{elliptic_curve_method_fast}{c.f., \rAlgo{inverse_mod}}
