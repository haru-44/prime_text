平方差法の原理は既に見た。
$n=9951$を素因数分解するとしたとき、
\begin{align*}
9951 &= 10000 - 49\\
&= 100^2 - 7^2\\
&= (100 - 7)(100 + 7)\\
&= 93 \times 107
\end{align*}
を見つけるのであった。

この原理を深く考察しよう。
一般に、$x\not\equiv\pm y\pmod{n}$で、$x^2\equiv y^2\pmod{n}$という関係を持つ$x,y$を見つければ、非自明な因数が$\gcd(x-y, n)$で求められる。
$9951$の場合は、$100^2\equiv7^2\pmod{9951}$であるから、$\gcd(100-7, 9951)=93$が得られる。

\IND{2次篩法}{2しふるいほう}(quadratic sieve algorithm)も同様の考え方に基づいた素因数分解法だが、詳細を一気に理解しようとすると大変なので、まずは全体の概要を説明した後、徐々に解像度を上げるように説明していきたい。

\subsubsection{2次篩法の考え方}
$x^2\equiv y^2\pmod{n}$となるような$x,y$を探すとき、運を天に任せるなら$x$をランダムに選び、$x^2\bmod{n}$が平方数になれば$x,y$が見つかったことになる。
もちろん、そのようなことは望むべくもないが、$x^2\bmod{n}$を調べるとき、次の点は重要である。

\begin{itemize}
 \item $x$は$\sqrt{n}$より大きい数を調べればよい。
 \item $x$が$\sqrt{n}$に近い範囲(具体的には$\sqrt{n}<x<\sqrt{2n}$)では、$x^2\bmod{n}$の計算を$x^2-n$で代用してよい。
\end{itemize}

そこで、次の2次関数を考える。
\begin{align*}
Q(x) = (x + \left \lfloor\sqrt{n}\right \rfloor)^2 - n
\end{align*}
この$Q(x)$に$x=1,2,\ldots$と次々に代入していき、$Q(x)$の値を求める。
$Q(x)$を導入した魂胆は、$x$が小さいと、$Q(x)$も小さくなると期待できるからである。
それはつまり、$Q(x)$を素因数分解しやすいと目論むことができるということだ。
例えば、$n=2201$のとき、
\begin{align*}
Q(1) &\equiv 47^2 \equiv 8 \equiv 2^3\pmod{2201}\\
Q(2) &\equiv 48^2 \equiv 103 \pmod{2201}\\
Q(3) &\equiv 49^2 \equiv 200 \equiv 2^3 \times 5^2 \pmod{2201}\\
Q(4) &\equiv 50^2 \equiv 299 \equiv 13 \times 23 \pmod{2201}\\
\end{align*}
となる。
この$Q(x)$が何の役に立つのか、疑問に感じるかもしれないが、既に素因数分解の下準備は完了している。
$Q(1)$と$Q(3)$をかけると、
\begin{align*}
47^2 \times 49^2 &\equiv 2^6 \times 5^2 \pmod{2201}\\
(47 \times 49)^2 &\equiv (2^3 \times 5)^2 \pmod{2201}
\end{align*}
が得られ、めでたく$x^2\equiv y^2\pmod{n}$を満たす$x=47\times49=2303$, $y=2^3\times5=40$が得られた。
よって、$\gcd(x-y,n) = \gcd(2303-40, 2201) = 31$という$2201$の素因数が求められる。

以上が、2次篩法の一連の流れである。
つまり、$Q(x)$の値を$x=1,2,\ldots$の順に計算し、素因数分解する。
この際に、適当な境界値$B$以下の素因数に完全に分解できるもののみを残す。
常に、完全な素因数分解が可能というわけではないし、その必要もないので、都合の悪い$Q(x)$は捨てるわけだ。
十分な数が集まれば$n$の素因数分解に成功する。

このアウトラインだけでは具体的な実装は不可能だ。
すぐさま沸き起こる疑問として、$Q(1), Q(2),\ldots$を計算して、素因数分解したはいいが、そのどれを組み合わせれば$x^2\equiv y^2 \pmod{n}$の形になるのだろうか。
問題は、右辺の組み合わせであるが、指数部分がすべて偶数になるようにすればよい。
よって、$(2$の指数$\bmod{2}, 3$の指数 $\bmod{2}, \ldots)$という$\mathbb{F}_2$のベクトルを考えればよい。
上記の例でいえば、
\begin{align*}
Q(1) &= (1, 0, 0, 0, 0, 0, 0, 0, 0)\\
Q(3) &= (1, 0, 0, 0, 0, 0, 0, 0, 0)\\
Q(4) &= (0, 0, 0, 0, 0, 1, 0, 0, 1)
\end{align*}
を考えるのである($Q(2)$は素因数が大きいため除外した)。
この中で、足してすべて$0$になるような組み合わせを見つければよい。

これは、$\mathbb{F}_2$上の連立一次方程式を解くことに他ならない。
そして、線形代数でよく知られるように、$k$次のときは$k+1$本の式が必要となる。
このためのアルゴリズムとして\IND{Gaussの消去法}{Gaussのしようきよほう}が知られている。

\subsubsection{篩とは何か?}
2次篩法の、「2次」の部分は2次関数$Q(x)$を考えるからということで理解できるが、「篩」はどこから来たのだろう。
それは、$Q(x)$が、$B$-smoothか判定するために篩を使用するためである。
2次篩法の速さは、篩が高速に実行できることに由来しており、結構重要な部分でもある。

さて、smoothな数を篩うアルゴリズムは、\rAlgo{b_smooth_sieve}で紹介した。
これを素直に適用するなら、$Q(x)$が取り得る値の範囲を篩うのかと思うかもしれないが、$Q(x)$の特徴を利用したより効果的な篩が構成できる。

それを考えるにあたって、$x^2 - n$が素数$p$で割り切れるとき$x,p$はどのような数か? という疑問を考察しよう。
$n$は$p$を法として平方剰余でなければならないことは明らかだ。
例えば、$n=101909=101\times1009$とすると、$\left(\frac{n}{3}\right)=-1$なので、どんな$x$でも$Q(x)$が3の倍数になることはない。
一方、$\left(\frac{n}{5}\right)=1$で解は$\pm2$なので、$x \equiv \pm2 \pmod{5}$のとき、$x^2 - n$は5の倍数となる。
これで、平方非剰余となる素数は\ruby{端}{はな}から考えなくてよいことが分かった。

