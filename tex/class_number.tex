$\mathcal{C}(D)$は前節まで単なる集合として扱ってきたが、適切な演算を入れることによって群になる。
前節までを振り返ると、2つの2次形式$(a,b,c),(a',b',c')$が同値であるとは、$f(x,y)=ax^2+bxy+cy^2$と$g(x,y)=a'x^2+b'xy+c'y^2$の生成する数が一致することであった。
そして、同じ数を生成するのに複数の表し方があるのは都合が悪いので、同値な2次形式たちを「代表」する2次形式を定めたかった。
それが簡約形式で、2つの2次形式が同値であることと、それぞれの簡約形式が一致することは必要十分条件であった。

最初に、途中で挫折しないように素因数分解までのアウトラインを示しておこう。
判別式$D<0$の2次形式全体を考えると無限に存在するが、簡約形式で考えれば有限個に収まって、しかもそれは群を成す。
この群は素因数分解に応用できて、奇数$n$を素因数分解したい数だとして、$D=-n$と置く(ただし、$n\equiv3\pmod{4}$)。
どんな群にも言えることだが、群の位数が分かると、位数2の元が比較的簡単に見つけられる。
さて、この群の位数2の元というのは、$(a,0,c),(a,a,c),(a,b,a)$のいずれかの形をしている。
例えば、位数2の元として$(a,a,c)$を見つけたとすると、実は$a$は$n$の約数である。

整理し直すと、
\begin{enumerate}
 \item 群の位数を求める。
 \item 位数2の元を求める。
 \item $n$の因数分解を得る。
\end{enumerate}
というステップである。
このアプローチによる最大の障害は、群の位数を求めることにあるから、この群について詳細に考えたい。
というのが、本節以降での流れだ。

では改めて、これが群になるということを説明する。
群であるということは、適切な演算が定義されなければならない。
それが合成と呼ばれる操作で、2つの2次形式から新たな2次形式を作る。
いま、$(a_1,b_1,c_1), (a_2,b_2,c_2)$という2つの判別式が$D$の2次形式を合成して新たな2次形式$(a_3,b_3,c_3)$を作ろう。
2つと同値な2次形式$(A_1,B,C_1), (A_2,B,C_2)$が存在するのだが\Notes{両者で$B$は共通であることに注意。}、$(A_1A_2,B,C_1/A_2)$もまた判別式$D$の2次形式である。
$(A_1A_2,B,C_1/A_2)$は一般には簡約形式ではないので、簡約したものを$(a_3,b_3,c_3)$とすれば合成は完了する。

このとき、単位元$1_D$となるのは
\begin{align*}
1_D = \begin{cases}
(1, 0, -D/4),& \mbox{if $D$は偶数}\\
(1, 1, (1-D)/4),& \mbox{if $D$は奇数}
\end{cases}
\end{align*}
である。

\Algo{2次形式の合成}{quadratic_form_composition}{c.f., \rAlgo{extended_gcd}, \rAlgo{quadratic_form_reduction}}

\begin{table}[htb]
\caption{まとめ}
\centering
\begin{tabular}{|c|c|c|c|c|c|}\hline
\multirow{2}{*}{$n \bmod{4}$} & \multirow{2}{*}{$D$}   & \multicolumn{2}{c|}{自明な分解を与える特異形式}                     & \multirow{2}{*}{特異形式} & \multirow{2}{*}{対応する分解} \\ \cline{4-4}
                              &                        &                                  & $1_D$                            &                           &                               \\ \hline\hline
\multirow{2}{*}{$3$}          & \multirow{2}{*}{$-n$}  &                                  & \multirow{2}{*}{$(1,1,(1+n)/4)$} & $(a,b,a)$                 & $2a + b \mid n$               \\ \cline{5-6}
                              &                        &                                  &                                  & $(a,a,c)$                 & $a \mid n$                    \\ \hline
\multirow{3}{*}{$1$}          & \multirow{3}{*}{$-4n$} & \multirow{3}{*}{$(2,2,(1+n)/2)$} & \multirow{3}{*}{$(1,0,n)$}       & $(a,b,a)$                 & $(2a + b)/2 \mid n$           \\ \cline{5-6}
                              &                        &                                  &                                  & $(a,a,c)$                 & $a/2 \mid n$                  \\ \cline{5-6}
                              &                        &                                  &                                  & $(a,0,c)$                 & $a \mid n$                    \\ \hline
\end{tabular}
\end{table}
