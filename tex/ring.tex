ここまで、Fibonacci数列とLucas数列が素数判定に利用できることを見た。
ここからは、なぜそうなるのか、ということを\ruby{詳}{つまび}らかにしたいところなのだが、このままでは「武器」が少なすぎる。
Fibonacci数列やLucas数列がどのような振る舞いをするのかを調べるために、「環」と呼ばれる構造を導入する。
群(\rDefi{group})には、(逆の演算として引き算や割り算は付随するが)足し算(あるいは掛け算)という1種類の演算しかなかった。
環は、足し算と引き算に加えて、掛け算もできる代数構造である。

\begin{Defi}{\IND{環}{かん}, ring}{ring}
集合$R$と二項演算$+,\times$について次の5つの条件が成り立つとき$(R,+,\times)$を環と呼ぶ。
\begin{enumerate}
 \item $(R,+)$は可換群である。
 \item 集合$R$は演算$\times$について閉じている。
 \item (\textbf{結合法則})任意の$a,b,c{\in}R$について$(a{\times}b){\times}c=a{\times}(b{\times}c)$が成り立つ。
 \item (\textbf{単位元})任意の$a{\in}R$について$a{\times}e_{\times}=e_{\times}{\times}a=a$となるような単位元$e_{\times}$が存在する。
 \item (\textbf{分配法則})任意の$a,b,c{\in}R$について$a{\times}(b{+}c)=a{\times}b{+}a{\times}c$と, $(b+c)\times{a}=b\times{a}+c\times{a}$が成り立つ。
\end{enumerate}
\end{Defi}

環のうち、ある$a,b{\in}R$について$a \times b \neq b \times a$であるとき、特に\IND{非可換環}{ひかかんかん}(noncommutative ring)と呼ぶ。
その対比として、任意の$a,b{\in}R$について$a \times b = b \times a$である環を\IND{可換環}{かかんかん}(commutative ring)と呼ぶ。

例えば、整数や$p$が素数であるときの$\mathbb{Z}_p$は環である。
一方で、行列は非可換環である。
多項式も環であり、特に断りなく一変数多項式を扱うが、多変数でも同様のことが言える。

\begin{Defi}{\IND{多項式環}{たこうしきかん}, polynomial ring}{polynomial ring}
環$R$に係数を持つ不定元$x$に関する多項式
\begin{align*}
a_mx^m + \cdots + a_2x^2 + a_1x^1 + a_0
\end{align*}
の集合$R[x]$は環を成す。
これを多項式環と呼ぶ。
\end{Defi}

つまり、既存の環を基にして新たな環を構成できるわけだ。
例えば、整数$\mathbb{Z}$は環であるから、整数係数多項式の集合$\mathbb{Z}[x]$は環である。
具体的な数値で見てみると、$3x+2$や$5x^2+3$などがその元だ。
加法や乗法は、通常の多項式の加法・乗法と変わりない。
$(3x+2)+(5x^2+3)=5x^2+3x+5$となるし、$(3x+2)(5x^2+3)=15x^3+10x^2+9x+6$となる。

確かに環であることが分かったが、それだけのことで、「嬉しさ」はまだ分からないだろう。
まず、都合のいいことに整数の場合と同様に剰余算が定義できる。
これが何を意味するかというと、整数$\mathbb{Z}$から剰余類$\mathbb{Z}_n$を得られたのと同様に、多項式環$\mathbb{Z}[x]$から剰余環$\mathbb{Z}[x]/(f(x))$が得られる\Notes{このような構成法を一般化するために\IND{イデアル}{いてある}(ideal)という概念が生まれ、環論を学ぶ上で避けては通れない道になっているが、ここでは無視して進む。}。

まず、多項式の剰余算についてだ。
つまり、多項式を多項式で割った余りを考える。
多項式を剰余算することに、違和感を覚えるかもしれない。
しかし、整数の剰余算とはどのように定義されるかを振り返ってみると、ある整数$q$が存在して、$a = nq + r$となるような$r$を剰余算の結果としている。
少し格式張った言い方をすると、除法の原理と呼ばれる。

\begin{Theo}{\IND{除法の原理}{しよほうのけんり}, division algorithm}{division_algorithm}
任意の$a\in\mathbb{Z},b\in\mathbb{N}$に対して、
\begin{align*}
a = bq + r, 0 \le r < b
\end{align*}
となる整数$q,r$の組がただ1つ存在する。
\end{Theo}

何のことはない、「8を3で割った余り」は2であって、1でも0でもない。
ただ1つ``2"に定まる、ということを言っているに過ぎない。
小学生に、いや普通の大人にさえも、除法の原理が超重要だと説いて、納得してくれる人は少ないだろう。
そういう話を開口一番にするのだから、数学者は煙たがられる。
「多項式の剰余」のような整数以外での剰余を考えたとき、そもそも一体「剰余」とは何者なのか明確にしたいという動機があって、除法の原理があるのである。

多項式でも除法の原理と同じようなことが言える。
多項式$a(x)$を$n(x)$で剰余算した結果$r(x)$とは、ある多項式$q(x)$が存在して、$a(x) = n(x)q(x) + r(x)$となる。
整数で$\mid r\mid < \mid n \mid$であったところは、$\deg r < \deg n$というように、次数の大小に置き換えられる。
このようにして、整数の剰余が環になったように、多項式の剰余も環を成す。

具体例で見てみよう。
実数係数多項式環$\mathbb{R}[x]$と多項式$(x^2+1)$から、新たな環$\mathbb{R}[x]/(x^2+1)$が得られる。
$x^2+1$で剰余算されるのだから、$x^2+1=0$である\Notes{整数においても$n$で剰余算するなら$n$は$0$と合同なのであった。}。
この関係式より$x^2=-1$が得られる。
例えば、$x^2+x+1$という多項式を$x^2+1$で剰余算するということは、$x^2=-1$で置き換えてやればいいことに外ならない。
ゆえに、$x^2+x+1$は$-1+x+1=x$と合同である。
ということは、$\mathbb{R}[x]/(x^2+1)$の元というのは、高々1次の多項式でしかないことが分かる。
また、加法や乗法も、通常の多項式の意味での加法・乗法の後に$x^2=-1$の変換を適用してやればよい。

これで多項式の剰余環を理解したつもりになられては困る。
$\mathbb{R}[x]/(x^2+1)$の元は既に、実数$a,b \in \mathbb{R}$を使って$ax+b$と表せることは確認した。
では、この$x$とは何者か? $x^2=-1$という関係しか利用してこなかったが、$x$について解けば$x=\sqrt{-1}=i$なのだ! 
つまり、$\mathbb{R}[x]/(x^2+1)$は複素数$\mathbb{C}$と一致する。
これまで複素数$\mathbb{C}$は、実数$\mathbb{R}$には存在しない元である虚数単位$i$を添加するして作る方法しか知らなった。
しかし今、多項式から複素数を作るという新しい方法を知った。
もっと言えば、新たな元を添加することと多項式の剰余は表裏一体の関係であるのだ。

この理屈は、何も実数に限らない。
整数$\mathbb{Z}$に整数ではない元、例えば$\sqrt{2}$を添加すると拡大する。
これを、記号的には$\mathbb{Z}[\sqrt{2}]$と書く。
具体的に$\mathbb{Z}[\sqrt{2}]$とは、$a+b\sqrt{2}$($a,b$は整数)と表される数の集合で、環を成す。
混乱を招く表現だが、多項式環$R[x]$と$R[\alpha]$とはまったく別物である。
前者は環$R$の元を係数とする多項式の集合が成す環であるのに対して、後者は環$R$に$\alpha$という元を添加した環である。
そして、$\mathbb{Z}[\sqrt{2}]$は$\mathbb{Z}[x]/(x^2-2)$まったく同じものである。
$\mathbb{Z}[x]/(x^2-2)$とは、$\mathbb{Z}[x]$の元を多項式$x^2-2$で剰余算した環である。
$x^2=2$という関係\Notes{$x^2-2=0$という関係式から求めた。}を用いると計算が簡単になる。
例えば、$\mathbb{Z}[x]/(x^2-2)$においては
\begin{itemize}
\item $4x^2+5x-6 = 4\cdot2 + 5x - 6 = 8 + 5x - 6 = 5x + 2 \pmod{x^2-2}$
\item $3x^3+x^2-3 = 3\cdot2\cdot x - 1 = 6x - 1 \pmod{x^2-2}$
\end{itemize}
というように、すべて1次以下の式となる。
実際、
\begin{itemize}
\item $4x^2 + 5x - 6 = 4(x^2 - 2) + (5x + 2)$
\item $3x^3+x^2-3 = (3x + 1)(x^2 - 2) + (6x - 1)$
\end{itemize}
というような関係があることが分かる。

Fibonacci数列やLucas数列を計算するとき、\rAlgo{fibonacci_sequence}や\rAlgo{lucas_sequence}では定義の通り再帰的に計算していた。
一方で、Fibonacci数列では
\begin{align*}
F_n = \frac{\phi^n - (1 - \phi)^n}{\phi - (1 - \phi)}
\end{align*}
Lucas数列では
\begin{align*}
U_n &= \frac{\alpha^n - \beta^n}{\alpha - \beta}\\
V_n &= \alpha^n - \beta^n
\end{align*}
というような一般式も知られているので、こちらを基にアルゴリズムを記述することも可能ではないかという発想が生まれる。
しかし、プログラミング経験のある諸兄なら、$\phi$のような無理数をそのまま扱って大丈夫かという危惧を覚えるだろう。
そこで、この精妙な代数学の成果を利用する。
計算すると厄介なものは、計算しない方がよい。
Lucas数列における$\alpha$は、普通、整数にならないが、計算の過程を含めても$U_n$は、整数$i,j$を使って$i + j\alpha$と表される。
再度の説明となるが、$\alpha$は$x^2-ax+b=0$の解の1つである。
ここで、掛け算をしたときに、$\alpha^2$の項が生じ得るのではないか、という疑問が沸き起こる。
$\alpha$は、$x^2-ax+b=0$の解であったことから、$\alpha^2-a\alpha+b=0$を満たす。
つまり、$\alpha^2=-b+a\alpha$と書き直せるのである。
以上より、$i+j\alpha$の形同士の足し算とかけ算は、やはり$i+j\alpha$の形になる。
\begin{align*}
(i_1 + j_1\alpha) + (i_2 + j_2\alpha) &= (i_1 + i_2) + (j_1 + j_2)\alpha\\
(i_1 + j_1\alpha) \times (i_2 + j_2\alpha) &= (i_1i_2 - bj_1j_2) + (i_1j_2 + i_2j_1 + aj_1j_2)\alpha
\end{align*}

今後、$i + j\alpha$を$(i,j)$と書く。
具体的な数は次のように表される。
\begin{align*}
0 &= (0, 0)\\
1 &= (1, 0)\\
\alpha &= (0, 1)\\
\alpha^2 &= (-b, a)\\
\beta &= (a, -1)\\
\beta^2 &= (a^2-b, -a)
\end{align*}

これを使えば、Lucas数列を一般式から計算するときでも、整数のみで計算できる。
