ここまで、Fibonacci数列とLucas数列が素数判定に利用できることを見た。
ここからは、なぜそうなるのか、ということを\ruby{詳}{つまび}らかにしたいところなのだが、このままでは「武器」が少なすぎる。
Fibonacci数列やLucas数列がどのような振る舞いをするのかを調べるために、「環」と呼ばれる構造を導入する。
群(\rDefi{group})には、(逆の演算として引き算や割り算は付随するが)足し算(あるいは掛け算)という1種類の演算しかなかった。
環は、足し算と引き算に加えて、掛け算もできる代数構造である。

\begin{Defi}{\IND{環}{かん}, ring}{ring}
集合$R$と二項演算$+,\times$について次の5つの条件が成り立つとき$(R,+,\times)$を環と呼ぶ。
\begin{enumerate}
 \item $(R,+)$は可換群である。
 \item 集合$R$は演算$\times$について閉じている。
 \item (\textbf{結合法則})任意の$a,b,c{\in}R$について$(a{\times}b){\times}c=a{\times}(b{\times}c)$が成り立つ。
 \item (\textbf{単位元})任意の$a{\in}R$について$a{\times}e_{\times}=e_{\times}{\times}a=a$となるような単位元$e_{\times}$が存在する。
 \item (\textbf{分配法則})任意の$a,b,c{\in}R$について$a{\times}(b{+}c)=a{\times}b{+}a{\times}c$と, $(b+c)\times{a}=b\times{a}+c\times{a}$が成り立つ。
\end{enumerate}
\end{Defi}

環のうち、ある$a,b{\in}R$について$a \times b \neq b \times a$であるとき、特に\IND{非可換環}{ひかかんかん}(noncommutative ring)と呼ぶ\Notes{非可換環の例として行列が挙げられる。}。
その対比として、任意の$a,b{\in}R$について$a \times b = b \times a$である環を\IND{可換環}{かかんかん}(commutative ring)と呼ぶ。

環は足し算・引き算・掛け算が可能であったが、体ではそれに加えて割り算も可能である。

\begin{Defi}{\IND{体}{たい}, field}{field}
集合$F$と二項演算$+,\times$について次の2つの条件が成り立つとき$(F,+,\times)$を\textbf{体}と呼ぶ。
\begin{enumerate}
 \item $(F,+,\times)$は環である。
 \item (\textbf{逆元}の存在)任意の$a\in{F}\setminus\{e_{+}\}$について$a{\times}a^{-1}=a^{-1}{\times}a=e_{\times}$となる逆元$a^{-1}$が存在する。
\end{enumerate}
\end{Defi}

\IND{斜体}{しやたい}(skew field)あるいは\IND{可除環}{かしよかん}(division ring)と呼ぶ流儀もあるが、これは積が可換かどうかに注目するときに登場する。
ここで体と言ったときは、暗黙の裡に可換であるとする。
さらに、要素が有限の体を\IND{有限体}{ゆうけんたい}(finite field)と呼ぶ。
ちなみに、体を表す文字としてしばしば$K$が用いられるが、これはドイツ語名``Körper"から来ている。
``Körper"は、「身体」という意味で日本語名の「体」は``Körper"の直訳である。

\begin{Exam}{}{ring_field}\;
\begin{itemize}
 \item 実数$\mathbb{R}$、有理数$\mathbb{Q}$は体である。
 \item 整数$\mathbb{Z}$は環である。
 \item $\mathbb{Z}_n$は環である。
 \item $p$が素数のとき、$\mathbb{Z}_p$は有限体である。
\end{itemize}
\end{Exam}

有限体の表記にはいくつかの種類があって、$\mathbb{Z}_p$以外にも$\mathbb{Z}/p\mathbb{Z}, \mathbb{F}_p, GF(p)$等がある。
ここでは\ruby{専}{もっぱ}ら$\mathbb{F}_p$の表記を用いる。

次に、既知の環や体から新たな環・体を作る方法として、多項式を導入する。
特に断りなく一変数多項式を扱うが、多変数でも同様のことが言える。

\begin{Defi}{\IND{多項式環}{たこうしきかん}, polynomial ring}{polynomial ring}
環$R$に係数を持つ不定元$x$に関する多項式
\begin{align*}
a_mx^m + \cdots + a_2x^2 + a_1x^1 + a_0
\end{align*}
の集合$R[x]$は環を成す。
これを多項式環と呼ぶ。
\end{Defi}

一応注意しておくが、体も環の一種なので体に係数として持つ多項式だって考えることができる。

例えば、整数$\mathbb{Z}$は環であるから、整数係数多項式の集合$\mathbb{Z}[x]$は環である。
具体的な数値で見てみると、$3x+2$や$5x^2+3$などがその元だ。
加法や乗法は、通常の多項式の加法・乗法と変わりない。
$(3x+2)+(5x^2+3)=5x^2+3x+5$となるし、$(3x+2)(5x^2+3)=15x^3+10x^2+9x+6$となる。

確かに環であることが分かったが、それだけのことで、「嬉しさ」はまだ分からないだろう。
まず、都合のいいことに整数の場合と同様に剰余算が定義できる。
これが何を意味するかというと、整数$\mathbb{Z}$から剰余類$\mathbb{Z}_n$を得られたのと同様に、多項式環$\mathbb{Z}[x]$から剰余環$\mathbb{Z}[x]/(f(x))$が得られる\Notes{このような構成法を一般化するために\IND{イデアル}{いてある}(ideal)という概念が生まれ、環論を学ぶ上で避けては通れない道になっているが、ここでは無視して進む。}。

まず、多項式の剰余算についてだ。
つまり、多項式を多項式で割った余りを考える。
多項式を剰余算することに、違和感を覚えるかもしれない。
しかし、整数の剰余算とはどのように定義されるかを振り返ってみると、ある整数$q$が存在して、$a = nq + r$となるような$r$を剰余算の結果としている。
少し格式張った言い方をすると、除法の原理と呼ばれる。

\begin{Theo}{\IND{除法の原理}{しよほうのけんり}, division algorithm}{division_algorithm}
任意の$a\in\mathbb{Z},b\in\mathbb{N}$に対して、
\begin{align*}
a = bq + r, \quad 0 \le r < b
\end{align*}
となる整数$q,r$の組がただ1つ存在する。
\end{Theo}

何のことはない、「8を3で割った余り」は2であって、1でも0でもない。
ただ1つ``2"に定まる、ということを言っているに過ぎない。
小学生に、いや普通の大人にさえも、除法の原理が超重要だと説いて、納得してくれる人は少ないだろう。
そういう話を開口一番にするのだから、数学者は煙たがられる。
「多項式の剰余」のような整数以外での剰余を考えたとき、そもそも一体「剰余」とは何者なのか明確にしたいという動機があって、除法の原理があるのである。

多項式でも除法の原理と同じようなことが言える。
多項式$a(x)$を$n(x)$で剰余算した結果$r(x)$とは、ある多項式$q(x)$が存在して、$a(x) = n(x)q(x) + r(x)$となる。
整数で$\mid r\mid < \mid n \mid$であったところは、$\deg r < \deg n$というように、次数の大小に置き換えられる。
このようにして、整数の剰余が環になったように、多項式の剰余も環を成す。

具体例で見てみよう。
実数係数多項式環$\mathbb{R}[x]$と多項式$(x^2+1)$から、新たな環$\mathbb{R}[x]/(x^2+1)$が得られる。
$x^2+1$で剰余算されるのだから、$x^2+1=0$である\Notes{整数においても$n$で剰余算するなら$n$は$0$と合同なのであった。}。
この関係式より$x^2=-1$が得られる。
例えば、$x^2+x+1$という多項式を$x^2+1$で剰余算するということは、$x^2=-1$で置き換えてやればいいことに外ならない。
ゆえに、$x^2+x+1$は$-1+x+1=x$と合同である。
ということは、$\mathbb{R}[x]/(x^2+1)$の元というのは、高々1次の多項式でしかないことが分かる。
また、加法や乗法も、通常の多項式の意味での加法・乗法の後に$x^2=-1$の変換を適用してやればよい。

これで多項式の剰余環を理解したつもりになられては困る。
$\mathbb{R}[x]/(x^2+1)$の元は既に、実数$a,b \in \mathbb{R}$を使って$ax+b$と表せることは確認した。
では、この$x$とは何者か? $x^2=-1$という関係しか利用してこなかったが、$x$について解けば$x=\sqrt{-1}=i$なのだ! つまり、$\mathbb{R}[x]/(x^2+1)$は複素数$\mathbb{C}$と一致する。
これまで複素数$\mathbb{C}$は、実数$\mathbb{R}$には存在しない元である虚数単位$i$を添加するして作る方法しか知らなった。
しかし今、多項式から複素数を作るという新しい方法を知った。
もっと言えば、新たな元を添加することと多項式の剰余は表裏一体の関係であるのだ。

この理屈は、何も実数に限らない。
整数$\mathbb{Z}$に整数ではない元、例えば$\sqrt{2}$を添加すると拡大する。
これを、記号的には$\mathbb{Z}[\sqrt{2}]$と書く。
具体的に$\mathbb{Z}[\sqrt{2}]$とは、$a+b\sqrt{2}$($a,b$は整数)と表される数の集合で、環を成す。
混乱を招く表現だが、多項式環$R[x]$と$R[\alpha]$とはまったく別物である。
前者は環$R$の元を係数とする多項式の集合が成す環であるのに対して、後者は環$R$に$\alpha$という元を添加した環である。
そして、$\mathbb{Z}[\sqrt{2}]$は$\mathbb{Z}[x]/(x^2-2)$まったく同じものである。
$\mathbb{Z}[x]/(x^2-2)$とは、$\mathbb{Z}[x]$の元を多項式$x^2-2$で剰余算した環である。
$x^2=2$という関係\Notes{$x^2-2=0$という関係式から求めた。}を用いると計算が簡単になる。
例えば、$\mathbb{Z}[x]/(x^2-2)$においては
\begin{itemize}
\item $4x^2+5x-6 = 4\cdot2 + 5x - 6 = 8 + 5x - 6 = 5x + 2 \pmod{x^2-2}$
\item $3x^3+x^2-3 = 3\cdot2\cdot x - 1 = 6x - 1 \pmod{x^2-2}$
\end{itemize}
というように、すべて1次以下の式となる。
実際、
\begin{itemize}
\item $4x^2 + 5x - 6 = 4(x^2 - 2) + (5x + 2)$
\item $3x^3+x^2-3 = (3x + 1)(x^2 - 2) + (6x - 1)$
\end{itemize}
というような関係があることが分かる。

$R[\alpha]$がどのような構造になるのか、ということは環論の重要な興味だが、特に体においては既約という概念によって整理される。

\begin{Defi}{\IND{既約多項式}{きやくたこうしき}, irreducible polynomial}{irreducible_polynomial}
$f(x)\in R[x]$が因数分解できない、つまり、$f(x)=g(x)k(x)$が成り立つのは、$\deg(g(x))=0$または$\deg(k(x))=0$となる場合に限るとき、$f(x)$を既約多項式と呼ぶ。
\end{Defi}

\begin{Theo}{}{}
体$K$上の多項式環$K[x]$で、$n$次多項式$f(x)\in K[x]$とし、$\alpha$を$f(x)$の根の1つとする。
\begin{enumerate}
 \item $f(x)$が既約多項式であるとき、かつそのときのみ、$K[x]/(f(x))$は体である。
 \item $f(x)$が既約多項式であるとき、かつそのときのみ、$\{a_0 + a_1\alpha + a_2\alpha^2 + \cdots + a_{n-1}\alpha^{n-1} : a_i \in K\}$は体である。
 \item 1と2で定めた体は同型である。
\end{enumerate}
\end{Theo}

有限体$\mathbb{F}_p$を考えよう。
$\mathbb{F}_p$の位数は$p$で素数である。
ここで、$k$次既約多項式$f(x)$が存在すると、$\mathbb{F}_p/(f(x))$は新たな体であり、$\mathbb{F}_{p^k}$や$q=p^k$と置いて$\mathbb{F}_q$と書く。

\begin{Theo}{}{}
素数$p$、自然数$k$、$q=p^k$とする。
位数$q$の有限体は、同型を除いて$\mathbb{F}_q$ただ1つである。
\end{Theo}

$k$次既約多項式と言ってもいくつもあるが、驚くべきことに、どの既約多項式でも$\mathbb{F}_p/(f(x))$は同型なので、$\mathbb{F}_q$という表記の中に$f(x)$が何であるかを示さなくとも基本的に問題にならないのだ。
もっとも、どのように表現されるかは既約多項式によって異なるので、具体的に実装しようとした場合は陽に現れる。
あまり抽象論をしていても実感がないだろうから、例を示して$\mathbb{F}_q$を説明しよう。
有限体$\mathbb{F}_2$において、$f(x)=x^4+x+1$は既約多項式である。
ゆえに、有限体$\mathbb{F}_{2^4}=\{a_0 + a_1\alpha + a_2\alpha^2 + a_3\alpha^3 : a_i \in \mathbb{F}_2\}$は$\alpha^4+\alpha+1=0$という関係式を用いて、掛け算で4次以上の項が現れた場合は$\alpha^4=\alpha+1$に置き換えてやればよい。
例えば、
\begin{align*}
(\alpha^3 + \alpha^2 + \alpha + 1) \times \alpha &= \alpha^4 + \alpha^3 + \alpha^2 + \alpha\\
 &= (\alpha + 1) + \alpha^3 + \alpha^2 + \alpha\\
 &= \alpha^3 + \alpha^2 + 1
\end{align*}
というふうに。
一方、$g(x)=x^4+x^3+1$も4次の既約多項式で、$\mathbb{F}_2/(g(x))$は$\mathbb{F}_{2^4}$と同型になるが、
\begin{align*}
(\alpha^3 + \alpha^2 + \alpha + 1) \times \alpha &= \alpha^4 + \alpha^3 + \alpha^2 + \alpha\\
 &= (\alpha^3 + 1) + \alpha^3 + \alpha^2 + \alpha\\
 &= \alpha^2 + \alpha + 1
\end{align*}
となり、見かけ上の計算は異なる。

$\mathbb{F}_q$の定数項のみを集めれば、$\mathbb{F}_p$に一致することを考えれば、$\mathbb{F}_p$は$\mathbb{F}_q$の部分体であることが分かる。

\begin{Defi}{\IND{拡大体}{かくたいたい}(extension field), \IND{部分体}{ふふんたい}(subfield)}{extension field}
2つの体$K, L$に、$K\subset L$という関係が成り立つとき、$K/L$と書く。
そして、$L$は$K$の拡大体であると言い、$K$は$L$の部分体であると言う。
\end{Defi}

\begin{Defi}{\IND{素体}{そたい}, prime field}{prime field}
部分体を持たない体を素体呼ぶ。
\end{Defi}

\begin{Prop}{}{prime field}
任意の素体は、$\mathbb{Q}$または$\mathbb{F}_p$に同型である。
\end{Prop}

素体は同型を除いて、有理数体$\mathbb{Q}$か$\mathbb{F}_p$しか存在しないことが分かった。
つまり、どんな体でも$\mathbb{Q}$か$\mathbb{F}_p$を部分として持っていることになる。
体の世界は、案外狭いように感じられただろうか。
標数という用語を定義すれば、素体のようすがもう少し詳しく描写できる。

\begin{Defi}{\IND{標数}{ひようすう}, characteristic}{characteristic}
体$\mathbb{F}$において、次を満たす$p$を$\mathbb{F}$の標数と呼ぶ。
\begin{align*}
\underbrace{1 + 1 + \cdots + 1}_{p \mbox{個}} = 0
\end{align*}
このような$p$がないときは$0$と定義する。
\end{Defi}

\begin{Prop}{}{prime field2}\;
\begin{itemize}
\item 体$\mathbb{F}$の標数が$0$のとき、$\mathbb{F}$は有理数体$\mathbb{Q}$と同型な部分体を持つ。
\item 体$\mathbb{F}$の標数$p$が$0$でないとき、それは素数であり、$\mathbb{F}$は有限体$\mathbb{F}_p$と同型な部分体を持つ。
\end{itemize}
\end{Prop}

つまり、標数が判明すれば素体が明らかになるわけだ。

Fibonacci数列やLucas数列を計算するとき、\rAlgo{fibonacci_sequence}や\rAlgo{lucas_sequence}では定義の通り再帰的に計算していた。
一方で、Fibonacci数列では
\begin{align*}
F_n = \frac{\phi^n - (1 - \phi)^n}{\phi - (1 - \phi)}
\end{align*}
Lucas数列では
\begin{align*}
U_n &= \frac{\alpha^n - \beta^n}{\alpha - \beta}\\
V_n &= \alpha^n - \beta^n
\end{align*}
というような一般式も知られているので、こちらを基にアルゴリズムを記述することも可能ではないかという発想が生まれる。
しかし、プログラミング経験のある諸兄なら、$\phi$のような無理数をそのまま扱って大丈夫かという危惧を覚えるだろう。
そこで、この精妙な代数学の成果を利用する。
整数$\mathbb{Z}$に$x^2-ax+b=0$の解の1つ$\alpha$を添加した環$\mathbb{Z}[\alpha]$を考え、この上で計算するのだ。
$\mathbb{Z}[\alpha]$は整数$i,j$を使って$i + j\alpha$と表される元を持つ。
掛け算するとき$\alpha$は、$x^2-ax+b=0$の解であったことから、$\alpha^2-a\alpha+b=0$を満たす。
つまり、$\alpha^2=-b+a\alpha$と書き直せるのである。
以上より、$i+j\alpha$の形同士の足し算とかけ算は、やはり$i+j\alpha$の形になる。
\begin{align*}
(i_1 + j_1\alpha) + (i_2 + j_2\alpha) &= (i_1 + i_2) + (j_1 + j_2)\alpha\\
(i_1 + j_1\alpha) \times (i_2 + j_2\alpha) &= (i_1i_2 - bj_1j_2) + (i_1j_2 + i_2j_1 + aj_1j_2)\alpha
\end{align*}

今後、$i + j\alpha$を$(i,j)$と書く。
具体的な数は次のように表される。
\begin{align*}
0 &= (0, 0)\\
1 &= (1, 0)\\
\alpha &= (0, 1)\\
\alpha^2 &= (-b, a)\\
\beta &= (a, -1)\\
\beta^2 &= (a^2-b, -a)
\end{align*}

これを使えば、Lucas数列を一般式から計算するときでも、整数のみで計算できる。
