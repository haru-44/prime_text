\subsubsection{Wieferich素数}
どんな素数でも成り立つ式があるとする。
例えばFermatの小定理は、$p$が奇素数なら
\begin{align*}
2^{p-1} \equiv 1 \pmod{p}
\end{align*}
を満たすのであった。
では、条件をもっと絞ったらどうなるだろうか。
例えば、
\begin{align*}
2^{p-1} \equiv 1 \pmod{p^2}
\end{align*}
を満たす素数$p$は存在するだろうか? このような素数をWieferich素数と呼ぶ。

\begin{Defi}{\IND{Wieferich素数}{Wieferichそすう}, Wieferich prime}{Wieferich_prime}
素数$p$が、
\begin{align*}
2^{p-1} \equiv 1 \pmod{p^2}
\end{align*}
を満たすとき、Wieferich素数と呼ぶ。
\end{Defi}

現在知られているWieferich素数は1093と3511のみである\Notes{\url{https://oeis.org/A001220}}。

他にも、\rTheo{lucas_sequence}を参考して、Lucas-Wieferich素数というものも考えることができる。

\begin{Defi}{\IND{Lucas-Wieferich素数}{Lucas-Wieferichそすう}, Lucas-Wieferich prime}{Lucas_Wieferich_prime}
$\Delta=a^2-4b$は平方数でないとする。
$p$が$\gcd(p, 2b\Delta)=1$を満たす素数で、
\begin{align*}
U_{p - \left(\frac{\Delta}{p}\right)} \equiv 0 \pmod{p^2}
\end{align*}
を満たすなら、$p$をパラメータ$(a,b)$に関するLucas-Wieferich素数と呼ぶ。
\end{Defi}

更に、\rTheo{fibonacci_prime}を参考にすれば、次のような素数を考えることができるが、Fibonacci-Wieferich素数は未だ見つかっていない。

\begin{Defi}{\IND{Fibonacci-Wieferich素数}{Fibonacci-Wieferichそすう}(Fibonacci-Wieferich prime), \IND{Wall-Sun-Sun素数}{Wall-Sun-Sunそすう}(Wall-Sun-Sun prime)}{Fibonacci_Wieferich_prime}
素数$p$がパラメータ$(1,-1)$に関するLucas-Wieferich素数であるとき、Fibonacci-Wieferich素数あるいはWall-Sun-Sun素数と呼ぶ。
\end{Defi}

Fibonacci-Wieferich素数が興味深いのは、Fermatの最終定理に関連していることだ。
(既にWilesによって証明されているが、)Fermatの最終定理が偽だったとしよう。
最初の反例の指数$p$は、Fibonacci-Wieferich素数でなければならない\cite{Zhi1992}。
つまり、未証明の状況下において、Fibonacci-Wieferich素数を探すことは、Fermatの最終定理の反例候補を探すことでもあったのだ。

\subsubsection{Wolstenholme素数}

\begin{align*}
{2p-1 \choose p-1} \equiv 1 \pmod{p^3}
\end{align*}
は、$p\ge5$なる素数で成り立つが、$p^4$に変更すると、合同式を満たす素数はぐっと減る。

\begin{Defi}{\IND{Wolstenholme素数}{Wolstenholmeそすう}, Wolstenholme prime \cite{RichardJ1995}}{Wolstenholme_prime}
素数$p$が、
\begin{align*}
{2p-1 \choose p-1} \equiv 1 \pmod{p^4}
\end{align*}
を満たすとき、Wolstenholme素数と呼ぶ。
\end{Defi}

\begin{Theo}{\cite{RichardJ1995}}{Wolstenholme_prime_equivalent}
$p\ge11$を素数とする。
次の3つは同値である。
\begin{enumerate}
 \item $p$はWolstenholme素数
 \item $p$は$B_{p-3}$の分子を割り切る。ここで、$B_n$は$n$番目のBernoulli数。
 \item $\sum_{p/6<k<p/4}\frac{1}{k^3} \equiv 0 \pmod{p}$
\end{enumerate}
\end{Theo}

2022年7月現在、知られているWolstenholme素数は、16843と2124679のみである。
これら以外のWolstenholme素数について、2007年には$10^9$以下には存在しないことが確認され\cite{McIntosh2007ASF}、2022年には$10^{11}$以下には存在しないことが確認されている\cite{Booker2022}\Notes{\url{https://oeis.org/A088164}}。

素数$p$がWolstenholme素数かどうかは、\rTheo{Wolstenholme_prime_equivalent}の3を使ってもよいが、逆元の計算が必要となる。
\cite{McIntosh2007ASF}で指摘されているように、多項式
\begin{align*}
\prod_{p/6<k<p/4} (x+k^3)
\end{align*}
の$x$の係数が$p$を法として0と合同かを調べればWolstenholme素数かが分かる。
念のため、根と係数の関係より$x$の係数は
\begin{align*}
\left(\prod_{p/6<k<p/4}k^3\right)\left(\sum_{p/6<k<p/4}\frac{1}{k^3}\right)
\end{align*}
となることに注意しておく。
よって、実装する際も単純に多項式を展開していくだけでよいが\Notes{2次以上の項は無視してよいし、係数も$\bmod{p}$で考えてよい。}、\rAlgo{is_wolstenholme_prime}では高速化のため$k^3$を足し算のみで計算するようにしている。

\Algo{Wolstenholme素数の判定}{is_wolstenholme_prime}{}
