\subsubsection{考え方}
平方剰余の基本的な事実であるが、改めて確認しよう。

\begin{Prop}{}{sqrt1}
$n$が素数ならば、$a^2 \equiv 1 \pmod{n}$を満たす$a$は、$1$と$-1$に限る。
\end{Prop}

普通の整数では経験的に明らかであるが、剰余の世界ではそうではない。
「$n$が素数ならば」という前提がある時点で察した方もいるだろうが、$n$が合成数の場合はその限りではない。
例えば、$n=15$のとき2乗して1になるのは、1, 14($\equiv-1$)の他に4,11($\equiv-4$)がある。

$n$が素数であったとして話を進めよう。
Fermatの小定理(\rTheo{Fermats-little-theorem})によれば、$n$が素数のとき
\begin{align*}
a^{n-1} \equiv 1 \pmod{n}
\end{align*}
が成り立つ。
この両辺の平方根を取る。
つまり、2乗して$a^{n-1}$や$1$になる数を考える。
左辺は、指数法則が剰余でも通用するため、$a^{(n-1)/2}$である。
右辺は、\rProp{sqrt1}より$\pm1$のどちらかである。
つまり、$n$が素数の場合、どんな$a$でも次の2つの内、少なくとも1つは満たす。
\begin{itemize}
 \item $a^{(n-1)/2} \equiv 1 \pmod{n}$
 \item $a^{(n-1)/2} \equiv -1 \pmod{n}$
\end{itemize}

次に、1つ目の$a^{(n-1)/2} \equiv 1 \pmod{n}$に着目する。
これも平方根を取ることによって、同じことが言える。
$a^{(n-1)/2} \equiv 1 \pmod{n}$になるのは、$a^{(n-1)/4}\bmod{n}$が$1$または$-1$のときである。

これを繰り返すことによって、直感的には次のように展開していく。
左上がFermatの小定理であり、右側が平方根が$-1$の場合、下側が平方根が$1$の場合を表している。
\begin{align*}
a^{n-1} &\equiv 1 \pmod{n} \qquad\to\qquad a^{(n-1)/2} \equiv -1 \pmod{n}\\
&\downarrow \\
a^{(n-1)/2} &\equiv 1 \pmod{n} \qquad\to\qquad a^{(n-1)/4} \equiv -1 \pmod{n} \\
&\downarrow \\
a^{(n-1)/4} &\equiv 1 \pmod{n} \qquad\to\qquad a^{(n-1)/8} \equiv -1 \pmod{n} \\
&\downarrow \\
&\vdots
\end{align*}

こう書くと、無限に続いてしまうように思われるかも知れないが、そうではない。
$a$の指数部が奇数のとき、平方根を取ることができなくなって、打ち止めとなるのだ。
具体的な数で試してみると、例えば$n=29$のとき、奇数の$7$に達した時点で止まる。
\begin{align*}
a^{28} &\equiv 1 \pmod{n} \qquad\to\qquad a^{14} \equiv -1 \pmod{n}\\
&\downarrow \\
a^{14} &\equiv 1 \pmod{n} \qquad\to\qquad a^{7} \equiv -1 \pmod{n} \\
&\downarrow \\
a^{7} &\equiv 1 \pmod{n}
\end{align*}

これら末端の数式をまとめると、次のような命題が得られる。
\begin{Prop}{}{}
$n=29$が素数ならば、どんな$a$でも次の3つの内、少なくとも1つを満たす。
\begin{itemize}
\item $a^{14} \equiv -1 \pmod{n}$
\item $a^{7} \equiv -1 \pmod{n}$
\item $a^{7} \equiv 1 \pmod{n}$
\end{itemize}
\end{Prop}

$n$に具体的な数値を与えて命題を書き下してみたが、この方法はどんな奇数$n$についても適用できることは明らかだ。
一般の$n$については次のように書ける。

\begin{Theo}{}{miller_rabin_a}
$m$を奇数とし、$n-1 = 2^s \cdot m$と表現する。
$n$が素数ならば、任意の自然数$a$($0<a<n$)について、次のうち、少なくともどちらか1つは満たす。
\begin{itemize}
\item $a^m \equiv 1 \pmod{n}$
\item ある$t$($0\le t<s$)が存在して、$a^{2^tm} \equiv -1 \pmod{n}$
\end{itemize}
\end{Theo}

素数判定法において、対偶を考えることは十八番だ。
この定理の対偶は次のようになり、\IND{Miller-Rabinテスト}{Miller-Rabinてすと}(Miller–Rabin primality test)はこの定理を利用する。

\begin{Theo}{}{miller_rabin_b}
$m$を奇数とし、$n-1 = 2^s \cdot m$と表現する。
次を同時に満たす自然数$a$($0<a<n$)が存在するならば、$n$は合成数である。
\begin{itemize}
\item $a^m \not\equiv 1 \pmod{n}$
\item 任意の$t$($0\le t<s$)について、$a^{2^tm} \not\equiv -1 \pmod{n}$
\end{itemize}
\end{Theo}

\subsubsection{Miller-Rabinテスト}
Miller-Rabinテストもやはり、確率的な素数判定法である。
「合成数」と判定されれば確実に合成数だが、「素数」と判定されても確実ではない。
なぜならば、合成数であるにも関わらずMiller-Rabinテストを突破するような$a$が存在するからだ。
例えば、$n=121=11^2$にMiller-Rabinテストを実施する場合
\begin{itemize}
\item $a^{60} \equiv -1 \pmod{121}$
\item $a^{30} \equiv -1 \pmod{121}$
\item $a^{15} \equiv -1 \pmod{121}$
\item $a^{15} \equiv 1 \pmod{121}$
\end{itemize}
の内、少なくとも1つを満たせば「素数」と判定される。
$a=3$を選ぶと4つ目の条件
\begin{align*}
3^{15} \equiv 1 \pmod{121}
\end{align*}
を満たし、$121$は素数と判定されることになるが、もちろん誤りだ。
しかし、本当の素数と違って、どんな$a$についてもテストをパスするわけではない。
実際、$a=2$の場合は
\begin{itemize}
\item $a^{60} \equiv 89 \pmod{121}$
\item $a^{30} \equiv 45 \pmod{121}$
\item $a^{15} \equiv 98 \pmod{121}$
\end{itemize}
となり、確かに合成数と判定される。
なので、「素数」と判定されても安心せず、確実性を上げるために複数回のチェックを行う必要がある。
ただし、$n$が十分小さければそのような心配をしなくてもいいかもしれない。
実際、$n<2047$の場合は、$a=2$のみで確実に素数かを判定できるし、$n<1373653$の場合は、$a=2$に加えて$a=3$でチェックするだけでよい。
となれば、より少ない$a$でなるべく大きな$n$を確実に素数判定できるような、巧い組み合わせを見つけれれば、嬉しいことになる。
有志による調査の結果が、\url{https://miller-rabin.appspot.com/}にまとめられているので参考にされたい。

\Algo{補助関数}{split_int}

\Algo{Miller-Rabinテスト}{miller_rabin_test}{c.f., \rAlgo{split_int}}

では、どのくらいの$a$が判定を誤らせるのだろうか。
1回の試行で間違う確率は、高々1/4と評価されている(Solovay-Strassenテストよりも良い)。

\begin{Prop}{}{miller_rabin_c}
奇数の合成数$n>9$に対して、Miller-Rabinテストが誤って素数と判定してしまうような$a$の個数は、$\frac{1}{4}\varphi(n)$以下である。
\end{Prop}

そのため、試行回数を増やすことによって、ほぼ確実に素数であることを判定できる。
ゆえに、RSA暗号の鍵生成に際して利用されるなど、広く実用に供されているアルゴリズムでもある。

\subsubsection{強擬素数}
Fermat擬素数などと同様に、Miller-Rabinテストをパスする合成数は底$a$の\IND{強擬素数}{きようきそすう}(strong pseudoprime)と呼ぶ。
これは、しばしば$\mbox{spsp}(a)$とも書かれる。

